%%%%%%
%BASE
%%%
\documentclass[12pt,a4paper]{article}
\usepackage[utf8]{inputenc}
\usepackage[T1]{fontenc}
\usepackage[francais]{babel}

%%%%%
%MATHS
%%%
\usepackage{amsmath}
\usepackage{array}

%%%%%
%COULEURS
%%%
\usepackage{xcolor}
\usepackage{color}

%%%%%
%PUCES
%%%
\usepackage{pifont}

\everymath{\displaystyle}
\usepackage{hyperref}
\setlength{\parindent}{10px}



%%%%%
%Haut de Page
%%%
\usepackage{fancybox}
\usepackage{fancyhdr}
\usepackage[left=1.3cm,right=1.2cm,top=2cm,bottom=1.5cm]{geometry}
\pagestyle{fancy}
\fancyhead[L]{Lespinasse Florentin}
\fancyhead[C]{}
\fancyhead[R]{1STI2D4~~~~~~~~\today}

\begin{document}
\begin{center}
        \shadowbox{\begin{large}
                \textcolor{black}{Jules Vernes}
        \end{large}}
    \end{center}
    \vspace{0.5 cm}

Jules Verne, naît à Nantes le 8 février 1828, est l'ainé de ces 4 frères et s\oe ur, il suit comme son père des études de droit. Jules Vernes se met à faire des écrits de tout genre, des romans de science fiction, des pièces de téhâtre...\par



\end{document}

