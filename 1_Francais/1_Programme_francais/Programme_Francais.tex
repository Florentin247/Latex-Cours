\documentclass[12pt,a4paper]{article}
\usepackage[utf8]{inputenc}
\usepackage[T1]{fontenc}
\usepackage{amsmath}
\usepackage{xcolor}
\usepackage{pifont}
\everymath{\displaystyle}
\usepackage[francais]{babel}
\usepackage{xcolor} 
\usepackage{hyperref}
\usepackage{pdfpages}
\usepackage{times}

\title{Fran\c cais}
\author{Florentin Lespinasse\\
	Miribel les Echelles\\
	France\\
	   \texttt{floles38@gmail.com}}
\date{\today}
\maketitle

\begin{document}
\begin{center}
\setcounter{tocdepth}{4}								% Montre seulement, ici jusqu'à : subsubsection
\renewcommand{\contentsname}{Sommaire}					% Change le nom du Sommaire	
\newpage
\tableofcontents										% Crée le Sommaire
\end{center}
\newpage
		\section[Les M\'emoires d'une \^ame]{Objet d'\'etude : La po\'esie du \textsc{\romannumeral 19}$^{e}$~siècle au \textsc{\romannumeral 21}$^{e}$~siècle}
		\textbf{Parcours :}\textit{\underline{Les M\'emoires d'une \^ame}}
			\subsection{\href{.extra/Bio/Hugo.pdf}{Victor Hugo}}
				\subsubsection{M\'elancholia 1 (1) }
\begin{dingautolist}{192}

\item \underline{Introduction} \par
Le titre est très explicite, il nous donne une vision de ce qui peut se trouver dans le texte. 
\underline{Mélancholia} qui nous viens du grec, la bible noire.
Le texte de Victor Hugo dénonce le travail des enfants à un trop jeune âge.

\item \underline{Probl\'ematique }\par
	\textcolor{blue}{Comment Victor Hugo d\'enonce le travail forc\'e des enfants?}
\item \underline{Plan du texte} \par
		$v_{1}~$\`a$~v_{16}$ : Un portrait path\'etique des enfants\\
        $v_{17}~$\`a$~v_{34}$ : Condamnation violente du travail 
	

\item \underline{D\'eveloppement} \par
	Voir cours.

\item \underline{Conclusion} \par
	De nos jours, le travail des enfants est toujours pr\'esent dans le monde comme en Chine ou en Afrique. Victor Hugo \`a utilis\'e une autre forme d'\'ecrits pour d\'enoncer le travail des enfants. Sous la forme romanesque. Il a aussi \'ecrit d'autres textes tels que \textit{Les Mis\'erables} en 1862. Ce texte lui aussi d\'enonce le travail forc\'e.
\end{dingautolist}
\href{.extra/Textes/Melancholia.pdf}{Voir le texte en entier}
 \newpage
				\subsubsection{Cr\'epuscule 2 (2)}

\begin{dingautolist}{192}

\item \underline{Introduction} \par
	Carpediem, du latin $carpe = cueille~et~diem = le jour$ cueille le jour, est une notion h\'erit\'e de la philosophie antique. Philosophie pour atteindre le bonheur 

\item \underline{Probl\'ematique }\par
		\textcolor{blue}{Comment Victor Hugo nous d\'elivre t-il sa conception de l'amour?}

\item \underline{Plan du texte} \par
		$v_{1}~$\`a$~v_{8}$ : Paysage nocturne\\
        $v_{9}~$\`a$~v_{16}$ : Dialogue entre la nature et la mort\\
        $v_{17}~$\`a$~v_{28}$ : Renaissance

\item \underline{D\'eveloppement} \par
	Voir cours.

\item \underline{Conclusion} \par
	Victor Hugo nous fait un rappel du Carpediem.
Une fusion entre les pa\"iens et les chr\'etiens est cr\'ee. Il y a une vision spirituelle et charnelle de l'amour. Deplus, on peut retrouver aussi une hymne \`a l'amour et \`a la vie.
Le message du po\`ete est donc de s'aimer, profiter de l'instant pr\'esent. 

\end{dingautolist}	
\href{.extra/Textes/Crepuscule.pdf}{Voir le texte en entier}
 \newpage
			\subsection{Solitude 1 (3)}
			\subsubsection{\href{.extra/Bio/La_Martine.pdf}{Alphonse de La Martine}}
\begin{dingautolist}{192}

\item \underline{Introduction} \par
		Grande figure de la po\'esie romantique mais aussi homme engag\'e politiquement, La Martine connut le succ\`es d\`es son 1$^{er}$ recueil intitul\'e \textit{M\'editations Po\'etique} en 1820. 
		En 1823, il publie un second recueil \textit{Nouvelles M\'editations Po\'etiques} d'o\`u est extra\^it $"La Solitude"$, 
		long po\`eme en alexandrins, en rimes suivies dans lequel il met en sc\`ene un personnage qui quitte la soci\'et\'e pour se fondre dans la nature. 
		Le po\`eme nous fait vivre la progression du personnage dasn la montagne et jusqu'au sommet.

\item \underline{Probl\'ematique }\par
		\textcolor{blue}{Comment \`a travers le r\'ecit d'une marche en montagne, le po\`ete rend-il compte d'une expérience spirituelle?}

\item \underline{Plan du texte} \par
		$v_{7}~$\`a$~v_{18}$ : Entre 2 mondes (quitte le mode des hommes et rejoins celui de la nature)\\
		$v_{19}~$\`a$~v_{32}$ : Ascension\\
		$v_{33}~$\`a$~v_{38}$ : \'Elevation (Arriv\'ee au sommet)

\item \underline{D\'eveloppement} \par
		Voir cours.

\item \underline{Conclusion} \par


\end{dingautolist}
\href{.extra/Textes/Solitude.pdf}{Voir le texte en entier}
 \newpage

			\subsection{Isolement 2 (4)}
			\subsubsection{\href{.extra/Bio/Borel.pdf}{P\'etrus Borel}}
\begin{dingautolist}{192}

\item \underline{Introduction} \par
		Po\`ete romantique qui n'a pas connu beaucoup de succ\`es de son vivant mais \`a \'et\'e reconnu comme l'un des \'ecrivains les plus originaux du romantisme. 
		Il a eu une vie tr\`es mouvement\'e avec des m\'etiers un peu partout. Sa vie est tr\`es instable vis \`a vis de sa famille. P\'etrus Borel prend un surnom : $"Lycantrophe"$ qui veut dire loup-garou. 
		Il \'ecrit un recueil lyrique \textit{Rhapsodies} en 1832. Le titre "Isolement" montre la solitude du po\`ete. On peut supposer qu'il recherche l'\^ame s\oe ur.

\item \underline{Probl\'ematique }\par
		\textcolor{blue}{Comment le po\`ete exprime-t-il sa vision de l'amour?}

\item \underline{Plan du texte} \par
		$v_{1}~$\`a$~v_{12}$ : Vision de la nature\\
		$v_{13}~$\`a$~v_{24}$ : Description de la femme id\'eale\\
		$v_{25}~$\`a$~v_{32}$ : La solitude du po\`ete

\item \underline{D\'eveloppement} \par
		Voir cours.

\item \underline{Conclusion} \par
		Il y a ici un po\`eme romantique par excellence, avec l'importance de la nature, le lyrisme, les r\'eff\'erences culturelles... Sa conception de l'amour est vital, vient combler un manque.\par 
		A. \textsc{Lamartine}
		"Un seul \^etre vous manques et tout est d\'epeupl\'e."
		
		Citation extraite du recueil \textit{M\'editations Po\'etiques} de La Martine dont le po\`eme s'intitule "Isolement"  		



\end{dingautolist}		
\href{.extra/Textes/Isolement.pdf}{Voir le texte en entier}
 \newpage

		\section[Voltaire, esprit des Lumi\`eres]{Objet d'\'etude : La litt\'eature d'id\'ee du \textsc{\romannumeral 16}$^{e}$~siècle au \textsc{\romannumeral 18}$^{e}$~siècle}
		\textbf{Parcours :}\textit{\underline{Voltaire, esprit des lumi\`eres}}
			\subsection{\href{.extra/Bio/Voltaire.pdf}{Voltaire}}
				\subsubsection[L'ing\'enu]{\textit{L'Ing\'enu}}
					\paragraph[Texte 1 (5)]{\textbf{\underline{Texte 1 : 1 (5)}}}
		Fin du premier chapitre de «~L'impitoyable bailli~» \`a «~chacun~s'alla~coucher~»

		\begin{dingautolist}{192}

		\item \underline{Introduction} \par
			On assiste à un débat sur la religion fait par Voltaire. C'est un apologue\footnote{Dévoile des idées par un divertissement}.\par
		1762 -- affaire Callas /époux sirveine.\par
		1765-66 -- Affaire du chevalier de la barre.\par
		1763 -- Traité de la tolérance (essai\footnote{Texte où l'on développe ses idées}).\par
		Le thème principal est la religion et le savoir. L'ingénu veut diffuser les idées par le compte philosophique (Candide -- 1759).

		\item \underline{Probl\'ematique }\par
			\textcolor{blue}{Comment à travers la mise en scène d'échange autour de la religion, Voltaire se livre t-il à une véritable défense de la tolérance? }

		\item \underline{Plan du texte} \par
			$l_{1}~$\`a$~l_{11}$ : Interrogation sur la religion du Huron\\ 
			$l_{12}~$\`a$~l_{21}$ : Décision de convertir le Huron
		\item \underline{D\'eveloppement} \par
				Voir cours.

		\item \underline{Conclusion} \par
			Voltaire fait une satire\footnote{Critique moqueuse} de l'étroitesse d'esprit des provinciaux qui pensent qu'il a le droit de devenir comme eux. Il critique l'intolérance des religions qui est son grand combat, en valorisant son personnage << L'Ingénu\footnote{Née libre} >> qui ici est défenseur de culte et de penser. 




		\end{dingautolist}
\href{.extra/Textes/Voltaire_Ingénu_Texte1.pdf}{Voir le texte en entier}
		 \newpage

							\paragraph[Texte 2 (6)]{\textbf{\underline{Texte 2 : 2 (6) }}}
		Fin du chapitre \textsc{\romannumeral 14} de << Les deux captifs ... un huron convertissait un janséniste. >>
\begin{dingautolist}{192}

\item \underline{Introduction} \par

\item \underline{Probl\'ematique }\par
	\textcolor{blue}{En quoi les valeurs de l'Ingénu s'affirment-elles ici?}\par
	Autre Problématique possible \par
	\textcolor{blue}{Quel portrait peut-on faire du Huron?}


\item \underline{Plan du texte} \par
	$l_{1}~$\`a$~l_{13}$ : Combat pour la liberté et critique de l'arbitraire du pouvoir\\
    $l_{14}~$\`a$~l_{24}$ : \'Eloge\footnote{Blâme} de l'amour \par
Autre Plan que l'on peut faire\par
	$l_{1}~$\`a$~l_{13}$ : Un Homme libre\\
    $l_{14}~$\`a$~l_{24}$ : Un Homme amoureux


\item \underline{D\'eveloppement} \par
        Voir cours.

\item \underline{Conclusion} \par
	Ainsi dans ce passage, l'Ingénu semble avoir achevé sa formation.
	Il a appris la philosophie, il a été initié à la recherche intellectuelle en lisant des livres, et l'amour joue également un rôle de guide dans son \'Education sentimentale. 
	Voltaire semble ici mettre sur le même plan l'amour et la philosophie donc le c\oe ur et la raison.
	L'Ingénu aura accomplit sa formation et deviendra un homme parfait.
	L'Ingénu philosophe intrépide, guerrier.
	Dans ce passage, le Huron affirme 2 valeurs fondamentales:\par
	\ding{223}La liberté : critiquant l'arbitraire du pouvoir\par
	\ding{223}L'amour : montrant qu'il peut être aussi noble que la philosophie ou que la religion \par

\end{dingautolist}
\href{.extra/Textes/Voltaire_Ingénu_Texte2.pdf}{Voir le texte en entier}
 \newpage

				\subsubsection{Article $"Guerre"$ 1 (7)}

\begin{dingautolist}{192}

\item \underline{Introduction} \par
	Voltaire philosophe des lumi\`eres ayant particip\'e \`a \textit{L'encyclop\'edie}. \'Ecrit en 1764 son propre \textit{dictionnaire phylosophique}. 
	Pour continuer son combat en faveur du progr\`es et de la tol\'erance. Et pour toucher un plus large publique. Exill\'e en Suisse (\`a Ferney) 1726-1729. Mais en 1761 et jusqu'en 1765, il \`a d\'effendu l'affaire Calas (Jean Calas).\par
	Dans L'article guerre de son dictionnaire, il d\'enonce l'absurdit\'e de la guerre en nous racontant une sorte de fable qui \`a valeure d'exemple.
	
\item \underline{Probl\'ematique }\par
	\textcolor{blue}{Par quels proc\'ed\'es Voltaire d\'enonce-t-il la guerre?}

\item \underline{Plan du texte} \par
	$l_{1}~$\`a$~l_{10}$ : Naissance d'une guerre \\
	$l_{11}~$\`a$~l_{23}$ : La guerre, un ph\'enom\`ene de contagion

\item \underline{D\'eveloppement} \par
        Voir cours.

\item \underline{Conclusion} \par
	\`A travers un r\'ecit divertissant, avec l'omnipr\'esence de l'ironie, Voltaire met en \'evidence l'absurdit\'ee de la guerre et donne \`a son jugement une dimension universelle. N'oublions pas qu'il s'agit d'une d\'efinition d'un article de l'encyclop\'edie. 

\end{dingautolist}
\href{.extra/Textes/Article_Guerre.pdf}{Voir le texte en entier}
 \newpage

				\subsection{Suppl\'ement au voyage de Bougainville 2 (8)}
			\subsubsection{\href{.extra/Bio/Diderot.pdf}{Diderot}}
\begin{dingautolist}{192}

\item \underline{Introduction} \par
	En 1772, il publie \textit{Supl\'ement au Voyage de Bougainville} un essai qui se pr\'esente sous la forme d'un dialogue entre 2 amis discutant du voyage du navigateur Bougainville, 
	lequel vient de faire para\^itre un r\'ecit relatant son tour du monde, et notamment sa d\'ecouverte de Tahiti.
	Les 2 amis en viennent \`a \'evoquer "Supl\'ement au Voyage de Bougainville''. En r\'ealit\'e invent\'e par Diderot pour apporter un autre regard sur le voyage de l'explorateur.
	Dans le passage, "Les adieux au vieillard'', Diderot met en sc\`ene un personnage fictif, un vieux tahitien, s'adressant \`a Bougainville pour lui demander de quitter son \^ile afin de la pr\'eserver.

\item \underline{Probl\'ematique }\par
	\textcolor{blue}{Comment Diderot remet-il en question la civilisation europ\'eenne?}

\item \underline{Plan du texte} \par
	$l_{1}~$\`a$~l_{10}$ : Reproche fait aux europ\'ens par rapport aux actes commis d\`es leur arriv\'e\\
	$l_{10}~$\`a$~l_{20}$ : Critique de la volont\'ee de conquête et de la colonisation\\
	$l_{20}~$\`a$~l_{24}$ : D\'eclaration d'\'egalit\' entre tahitiens et europ\'ens 

\item \underline{D\'eveloppement} \par
        Voir cours.

\item \underline{Conclusion} \par
	Diderot, \`a travers le vieux tahitien, d\'enonce la notion de propri\'et\'e qui s'installe \`a cause d'eux.
	On retrouves toutes les caract\'eristiques d'une vision Utopique.
	Il prône la Libert\'e, l'\'Egalit\'e, la Fraternit\'e des Tahitiens contre la soci\'et\'e des europ\'ens, et au service de l'esclavage.
	Ce qu'on appelle "le Mythe du bon sauvage'', se retrouve dans plusieurs textes de Diderot. Mais aussi dans d'autres tels que \textit{L'Ing\'enu} de Voltaire.

\end{dingautolist}
\href{.extra/Textes/Bougainville.pdf}{Voir le texte en entier}
 \newpage

		\section[Sciences et fiction]{Objet d'\'etude : Le roman et le r\'ecit du Moyen\^Age au \textsc{\romannumeral 21}$^{e}$~siècle}
		\textbf{Parcours :}\textit{\underline{Science et fiction}}
			\subsection{\href{.extra/Bio/Jules.pdf}{Jules Vernes}Voyage au centre de la Terre}
				\subsubsection[Texte 1 (9)]{\textbf{\underline{Texte 1 : 1 (10)}}}
			Le portrait du professeur Lindenbrock	
\begin{dingautolist}{192}

\item Introduction \par
Nous allons étudier un passage du livre \textit{Voyage au centre de la Terre} écrit en 1864. Axel se représente son oncle comme un savant, mais avec des défauts.
Il commence par nous faire une description du savoir de son oncle puis il nous explique qui le connaît et enfin, Axel nous montre la face caché du Professeur.
\item Probl\'ematique \par
	\textcolor{blue}{Quel portrait, Axel fait-il de son oncle?}
\item Plan du texte \par	
	$l_{1}~$\`a$~l_{6}$ : Un grand savant\\
    $l_{7}~$\`a$~l_{15}$ : Un savant de grande renommée\\
    $l_{16}~$\`a$~l_{24}$ : Portrait physique et moral qui met en lumière les défauts du Professeur
\item D\'eveloppement \par
        Voir cours.

\item Conclusion \par
Un portrait physique et moral élogieux mais contrasté par certains points du caractère du Professeur Lindenbrock. Le portrait d'un scientifique talentueux et passionné mais décalé dans sa vie sociale.
On peut le rapprocher à la description du célèbre Frankenstein, savant-fou ayant donné vie a un monstre.


\end{dingautolist}
\href{.extra/Textes/J.Verne_texte_1.pdf}{Voir le texte en entier}
\newpage
				\subsubsection[Texte 2 (10)]{\textbf{\underline{Texte 1 : 2 (11)}}}
		Chapitre 18 au~c\oe ur du volcan
\begin{dingautolist}{192}

\item Introduction \par
Dans ce passage du livre \textit{Voyage au centre de la Terre}, nous allons parler du début d'une nouvelle aventure, un nouveau décor. 


\item Probl\'ematique \par
	\textcolor{blue}{Comment à travers ce passage, l’auteur parvient-il à nous faire vivre l’intensité de ces premières découvertes souterraines?}
\item Plan du texte \par
	$l_{1}~$\`a$~l_{15}$ : Au seuil de l'aventure\\
    $l_{16}~$\`a$~l_{27}$ : Un spectacle impressionnant\\
    $l_{28}~$\`a$~l_{32}$ : La fascination d'Axel

\item D\'eveloppement \par
        Voir cours.

\item Conclusion \par
Les trois aventuriers, viennent de franchir la première grotte, celle qu'ils décrivent d'une beauté sans nom. Cette déscente dans un autre univers nous montre bien l'image d'un nouveau monde, celui qui n'a jamais été abîmé par l'homme. 

\end{dingautolist}
\href{.extra/Textes/J.Verne_texte_2.pdf}{Voir le texte en entier}

\newpage
\subsection{Frankenstein ou le Prom\'eth\'e moderne 1 (11) }
				\subsubsection{\href{.extra/Bio/Shelly.pdf}{Mary Shelly}}
\begin{dingautolist}{192}

\item Introduction \par
Histoire qui raconte la création d'un monstre. La créature tue le frère du D$^r$ Frankenstein, elle tue aussi la fiancée du D$^r$. Pour se venger du refus de créer une fiancé au monstre. La créature met fin à ses jours par le feu car il a trop de remords envers ce qu'il à fait. En 1818, l'époque des Lumières se termine tout juste. C'est un romans visionnaire qui met en garde contre la science. Quand l'auteur écrit ce texte, elle a seulement 18 ans.
\item Probl\'ematique \par
	\textcolor{blue}{Comment ce passage met-il en scène les dangers de la science?}
\item Plan du texte \par
	$l_{1}~$\`a$~l_{6}$ : Mise en scène lugubre\\
    $l_{7}~$\`a$~l_{12}$ : Description de la créature\\
    $l_{13}~$\`a$~l_{18}$ : Remors du créateur


\item D\'eveloppement \par
        Voir cours.

\item Conclusion \par
Mise en scène lugubre pour raconter une naissance, ce qui ne représente rien de bon.\par
Labeur, révélation immédiate, résultat.\par
Paradox $\Rightarrow$ misérable --- pitiée --- apres tant de soins.


\end{dingautolist}
\href{.extra/Textes/Frankenstein.pdf}{Voir le texte en entier}


\newpage
\subsection{Germinal 2 (12)}
				\subsubsection{\href{.extra/Bio/Zola.pdf}{\'Emile Zola}}
\begin{dingautolist}{192}

\item Introduction \par
\item Probl\'ematique \par
	\textcolor{blue}{Comment l’auteur parvient-il à nous faire ressentir la souffrance des mineurs ?}
\item Plan du texte \par
	$l_{1}~$\`a$~l_{6}$ :   		\\
    $l_{7}~$\`a$~l_{12}$ :  		\\
    $l_{13}~$\`a$~l_{18}$ : 


\item D\'eveloppement \par
        Voir cours.

\item Conclusion \par


\end{dingautolist}
\href{.extra/Textes/Germinal_Texte_2_12.pdf}{Voir le texte en entier}



\end{document}

