\documentclass[12pt,a4paper]{article}
\usepackage[utf8]{inputenc}
\usepackage[T1]{fontenc}
\usepackage{amsmath}
\usepackage{xcolor}
\usepackage{pifont}
\everymath{\displaystyle}
\usepackage[francais]{babel}
\usepackage{xcolor} 
\usepackage{hyperref}
\usepackage[framed]{ntheorem}
\usepackage[np]{numprint}
\usepackage{array}
\usepackage{colortbl}
\usepackage{color}

\renewcommand{\arraystretch}{1.5}
\setlength{\tabcolsep}{1cm}


\newtheorem{chap}{Chapitre}
\newtheorem{li}{Livre}

\begin{document}
\setcounter{tocdepth}{3}								% Montre seulement, ici jusqu'à : subsubsection
\renewcommand{\contentsname}{Sommaire}					% Change le nom du Sommaire	
\tableofcontents										% Crée le Sommaire
\\ \newpage

\section {\textit{Les Contemplations}}
	\subsection{Victor Hugo}
\begin{li} \\

~~
\\ \textsc{\romannumeral 1}
{\fontfamily{pzc}\selectfont 
\`A ma fille\par
	Il parle de sa fille.
	Il lui explique comment ne plus être de ce monde est fabuleux.
	Il montre que personne n'est bien dans ce monde.
	$v_{21-24}$ Leçon de vie à sa fille , vivre est difficile, il lui demande de suivre ses pas --> chemin, Ne rien haîr, tout aimer.
	Le bonneur est dans les petites choses.\\ \\
}
	\textsc{\romannumeral 3}
	{\fontfamily{pzc}\selectfont
	Mes deux filles\par
	Poème dédié à ses 2 filles, "enfants'' qui symbolise la purtée, l'innocence de l'enfant.
	Il décrit la photographie d'un instant.\\ \\
}
	\textsc{\romannumeral 11}
	{\fontfamily{pzc}\selectfont
	Lise\par
	Il parle de son amour de jeunesse avec Lise.
	Qu'il était prêt à l'épouser et qu'ils s'aimaient mais que étant si jeune il ne pouvait pas se marier. Ils étaient épanouis.
	Lise est comparée a un ange, fée, princesse.
	Il est admiratif envers elle. Le soir --> vieillesse.\\ \\
}
	\textsc{\romannumeral 15}
	{\fontfamily{pzc}\selectfont
	La coccinelle\par
	Victor Hugo met en avant la bêtise de l'homme.
	Thème qui renvois à Lise \ding{115}.\\ \\
}
	\textsc{\romannumeral 16}
	{\fontfamily{pzc}\selectfont
	Vers 1820\par
	Il donne l'impression que le maris de Denise est un trompeur. \\ \\
}
	\textsc{\romannumeral 17}
	{\fontfamily{pzc}\selectfont
	\`A M$^r$ Fromant Meurice\par
	Il explique que si un personne fait quelquechose, ce sera pour elle une oeuvre d'art mais pas forcément pour la personne d'enface. \\ \\
}
	\textsc{\romannumeral 25}
	{\fontfamily{pzc}\selectfont
	Unité\par
	Margueurite innocente, regarde le soleil, dialogue entre eux.
	Unité de la Nature, Victor Hugo, passe l'idée que tout les élements se vallent. (Petite fleure --> ce gran soleil).



}
\end{li}
\begin{li}
{\fontfamily{pzc}\selectfont 


}
\end{li}
\begin{li}
{\fontfamily{pzc}\selectfont 


}
\end{li}
\begin{li}
{\fontfamily{pzc}\selectfont 


}
\end{li}



\\ \newpage



\section {\textit{Paroles}}
	\subsection{Jaques Pr\'evert}
{\fontfamily{pzc}\selectfont 



}
\\ \newpage


\section {\textit{L'ing\'enu}}
	\subsection{Voltaire}
\begin{chap}\par
		{\fontfamily{pzc}\selectfont \par
Comment le prieur de Notre-Dame  de la Montagne et M$^l^l^e$sa s\oe ur rencontrèrent un Huron\par
	M$^r$ et M$^{me}$ de Kerkabon se promènent, --incipit -->  mise en situation. 
	Arrive un homme Huron (habillé comme un indien) qui vient de débarquer en Basse bretagne (Saint Malo).
	Les anglais l'ont fait prisonier et il s'est habitué à la tibue.
	Les Kerkabon l'invitent à manger chez eux. 
	Il fait connaissance avec les amis de M$^r$ et M$^{me}$ de Kerkabon.
	L'abbé de Saint-Yves et sa soeur M$^{me}$ de Saint-Yves.
	Le baillis, Individu courtois, etonnement que ce "Sauvage'' soit aussi bien civilisé.
	Tellement admiratif, ils décident de l'integrer dans leur famille, et veulent le baptiser.

	\begin{table}[ht]
   		\centering
   		\begin{tabular}{|c|c|}
  			\hline
 	M$^{lle}$ de Kerkabon & M$^r$ L'abbé de Kerkabon \\
		   \hline 
	45 ans & assé agé\\
			\hline
	Vieille fille\footnote{Femme vierge} & bon éclésistique\footnote{homme d'église} \\
		   \hline
	Bon coeur, sensible & aimé par ses voisins\\
		   \hline
   		\end{tabular}
	\end{table}




}

\end{chap}
\begin{chap}
{\fontfamily{pzc}\selectfont \par
Le Huron, nommé l'Ingénu, reconnu de ses parents\par
	Les Anglais repartent mais le Huron reste.
	Il y a une scène de reconnaissance grâce à son petit pendentif qu'il porte autour de son coup. --> (Coup de téhâtre)


}

\end{chap}
\begin{chap}
{\fontfamily{pzc}\selectfont \\ \par \\
Le Huron nomé l'Ingénu\par
	M$^r$ le prieur compte le bâptiser pour le convertir dans la religion et lui donner son nom (M$^r$ le prieur).
	Lors du bâpteme, le Huron est introuvable.
	M$^{lle}$ de Kerkabon et M$^{lle}$ de Saint-Yves se promènent et découvrent le Huron nu dans la petite rivière.


}

\end{chap}
\begin{chap}
{\fontfamily{pzc}\selectfont \par
L'ingénu bâptisé\par
	Le Huron ne veux pas se faire bâptisé aurement que dans la rivière.
	M$^{lle}$ de Saint-Yves est contente d'être la marraine du Huron.
	Il fût bâptisé puis prends le nom de \textbf{Hercule}, pour ses 12 miracles...
	Dumoins 13, qui à été de changer 50 filles en femmes en une seule nuit.

}

\end{chap}
\begin{chap}
{\fontfamily{pzc}\selectfont \par
L'ingénu amoureux \par
M$^r$ Hercule (son nouveau nom), veux époser M$^{lle}$ de Saint-Yves mais étant son neuveux, il ne peut épouser sa marraine, c'est un crime fondamental pour la religion chrétienne.
Il décide donc soit de se débâptiser, soit de demander une faveur au pape.



}

\end{chap}
\begin{chap}
{\fontfamily{pzc}\selectfont mon bout de texte}

\end{chap}
\begin{chap}
{\fontfamily{pzc}\selectfont mon bout de texte}

\end{chap}
\begin{chap}
{\fontfamily{pzc}\selectfont mon bout de texte}

\end{chap}
\begin{chap}
{\fontfamily{pzc}\selectfont mon bout de texte}

\end{chap}
\begin{chap}
{\fontfamily{pzc}\selectfont mon bout de texte}

\end{chap}
\begin{chap}
{\fontfamily{pzc}\selectfont mon bout de texte}

\end{chap}
\begin{chap}
{\fontfamily{pzc}\selectfont mon bout de texte}

\end{chap}
\begin{chap}
{\fontfamily{pzc}\selectfont mon bout de texte}

\end{chap}
\begin{chap}
{\fontfamily{pzc}\selectfont mon bout de texte}

\end{chap}
\begin{chap}
{\fontfamily{pzc}\selectfont mon bout de texte}

\end{chap}
\begin{chap}
{\fontfamily{pzc}\selectfont mon bout de texte}

\end{chap}
\begin{chap}
{\fontfamily{pzc}\selectfont mon bout de texte}

\end{chap}
\begin{chap}
{\fontfamily{pzc}\selectfont mon bout de texte}

\end{chap}
\begin{chap}
{\fontfamily{pzc}\selectfont mon bout de texte}

\end{chap}
\begin{chap}
{\fontfamily{pzc}\selectfont mon bout de texte}

\end{chap}
\\ \newpage

\end{document}

