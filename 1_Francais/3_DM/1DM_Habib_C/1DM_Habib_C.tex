%%%%%%
%BASE
%%%
\documentclass[12pt,a4paper]{article}
\usepackage[utf8]{inputenc}
\usepackage[T1]{fontenc}
\usepackage[francais]{babel}

%%%%%
%MATHS
%%%
\usepackage{amsmath}
\usepackage{array}

%%%%%
%COULEURS
%%%
\usepackage{xcolor}
\usepackage{color}

%%%%%
%PUCES
%%%
\usepackage{pifont}

\everymath{\displaystyle}
\usepackage{hyperref}
\setlength{\parindent}{10px}



%%%%%
%Haut de Page
%%%
\usepackage{fancybox}
\usepackage{fancyhdr}
\usepackage[left=1.3cm,right=1.2cm,top=2cm,bottom=1.5cm]{geometry}
\pagestyle{fancy}
\fancyhead[L]{Lespinasse Florentin}
\fancyhead[C]{DM de Français}
\fancyhead[R]{1STI2D4~~~~~~~~le 14/02/2020}

\begin{document}
\vspace{0.5 cm}

Il faut repensser la tolérence tant que cela reste en rapport aux mœurs autant qu’aux croyances. Le siècle des lumieres, ont réorganisés le rapport à la croyances.
Les philosophes ont pris cette avancée en tant qu’hypothèse et jeux visant l’esprit c’est devenu un quotidient. 
Certains se demandent comment les parisiens peuvent rester français à Paris. Tout homme  ne peut pas être parfait. 
L’indulgeance est seulement un autre nom de la tolérance. 
Elle même peut être comme le fair-play, c’est toujours mieu avec dans les règles mais pas tout le monde aquière cette règle. 
Dans la démocatie actuelle, il est de plus en plus utilisé dans les règles du jeu. 
La tolérance est un mode de vie choisie par la nation. 
C’est comme une extention, elle n’est pas obligatoire, elle permet d’approfondir l’opinion publique, son inverse est l’étroitesse d’esprit. 
Cela permet aussi des discutions sans fin, il y aura toujours une personne pour relancer le débat. 
L’indulgence, demande de reflechir, observer avant d’ommettre un jugement sur une personne. 
Le meilleur argument est celui qui clôt une discution. 
L’indulgence est inspiré de la sympatie ou du savoir vivre d’autruit, réduite à ce sens élementaire, elle sera valable dans la société. 
Tolérons-nous les uns les autres, est un message de l’époque du régime de Louis XIV. 
Il est resté dans la démocatie car il ne régit pas la société, au conrtaire, il l’assouplit. 
Les croyances et les mœurs vont avoir tendence à s’entraider, à se vivifier l’un envers l’autre et non pas se détruire.\par
\begin{center}
Claude Habib, \textit{Comment peut-on être tolérant?} "Repenser la tolérance."
\end{center}


\end{document}

