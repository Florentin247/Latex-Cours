%%%%%%
%BASE
%%%
\documentclass[12pt,a4paper]{article}
\usepackage[utf8]{inputenc}
\usepackage[T1]{fontenc}
\usepackage[francais]{babel}

%%%%%
%MATHS
%%%
\usepackage{amsmath}
\usepackage{array}

%%%%%
%COULEURS
%%%
\usepackage{xcolor}
\usepackage{color}

%%%%%
%PUCES
%%%
\usepackage{pifont}

\everymath{\displaystyle}
\usepackage{hyperref}
\setlength{\parindent}{10px}



%%%%%
%Haut de Page
%%%
\usepackage{fancybox}
\usepackage{fancyhdr}
\usepackage[left=1.3cm,right=1.2cm,top=2cm,bottom=1.5cm]{geometry}
\pagestyle{fancy}
\fancyhead[L]{Lespinasse Florentin}
\fancyhead[C]{\underline{Voyage au centre de la Terre} de Victor Hugo}
\fancyhead[R]{1STI2D4~~~~12 Mars 2020}

\begin{document}
\begin{center}
        \shadowbox{\begin{large}
                \textcolor{black}{DM de Fran\c cais}
        \end{large}}\\
    \end{center}
    \vspace{0.5 cm}
\sffamily{
Hambourg, 24 mai 1863.
Le professeur Lidenbrock rentre brusquement chez lui pour se réfugier dans son cabinet où il appelle son neveu Axel avec insistance.
La source de cet empressement est l’achat d’un manuscrit islandais du xiie siècle écrit en runique.
Cet évènement requiert toute l’attention de l’oncle, qui en perd même l’appétit.\par
Un parchemin s’est détaché du manuscrit.
Il est de la main de l’alchimiste Arne Saknussemm et  daterait  au  plus  tôt  du  xive siècle.
Axel cherche à décoder cette écriture étrange du parchemin caché, et perce finalement le secret du code : un voyage dangereux et fascinant s’offre à eux, mais il veut détruire le document pour l’éviter.
L’oncle apparait mais Axel garde le silence. 
L’oncle met tout le monde à la diète tant que le code ne sera pas découvert. 
Sous la pression, Axel lui révèle le secret.\\

Le message fait état d’un voyage au centre de la Terre, en entrant par un volcan islandais, le Sneffels.
Axel fait part de ses inquiétudes à son oncle et maitre.
La petite amie d’Axel, Graüben, le convainc finalement de partir avec son oncle.
Arrivés à Copenhague, le professeur se met en quête d’un bateau pour l’Islande.
En attendant le départ, il fait faire à son neveu des exercices pour vaincre le vertige.
Onze jours plus tard, ils rencontrent des personnalités de Rejkjavik : le gouverneur Trampe, le maire Finsen et le professeur Fridriksson.
Les deux Allemands sont conviés à la table de Fridriksson qui leur apprend l’histoire d’Arne Saknussemm et du volcan éteint.
Fridriksson leur présentera un bon guide.\\

Le lendemain, ils rencontrent le guide Hans Bejlke, qui les accompagnera tout au long du voyage.
Le 16 juin au matin, l’expédition est lancée.
Le paysage islandais est pauvre et désolé.
Ils traversent le pays à cheval et parfois en barque, notamment pour franchir un fjord.
Après avoir gouté à nouveau à l’hospitalité islandaise, les voyageurs arrivent aux premières coulées de lave.
Le 22 juin, ils sont au pied du Sneffels et logent chez l’habitant dans la bourgade de Stapi.
L’ascension du Sneffels commence, lente et silencieuse.
Hans montre le chemin et prévient le danger.
Le soleil de minuit les attend au sommet.\\

Les aventuriers descendent dans le cratère.
Ils découvrent l’inscription « Arne Saknussemm », puis descendent dans la cheminée du volcan par un maigre escalier.
Dix heures plus tard, le fond est atteint.
Ils s’enfoncent alors vers les entrailles de la Terre par un petit couloir.
La chaleur est supportable, malgré la profondeur.
Cependant, Axel est inquiet.
Arrivés à la fin de la galerie, ils se trouvent face à deux chemins.
L’oncle choisit d’aller vers l’est, mais il semble que les voyageurs remontent vers la surface.
L’eau se fait rare.
Ils traversent une houillère et Axel observe les couches géologiques témoignant de l’évolution de la Terre.
Un mur obstrue finalement le passage.\\

Il faut rebrousser chemin.
Ils décident malgré tout de poursuivre leur périple.
Assoiffé et fatigué par la marche, Axel s’évanouit alors qu’ils sont dans une prison de granit.
À son réveil, Hans, qui a entendu au loin le bruit d’un torrent, fore un trou dans la roche : il en jaillit une eau bouillante et ferrugineuse.
Le ruisseau est baptisé « Hans-bach ».
Cet évènement redonne courage au jeune Axel.
Ils sont désormais à cinq lieues de profondeur, sous l’Atlantique.
Arrivés à une grotte à seize lieues de profondeur, les voyageurs se reposent. 
Le professeur et son neveu s’adonnent à des considérations scientifiques.\\

Peu à peu, le mutisme de Hans les gagne.
Le 7 aout, Axel perd de vue ses deux compagnons et le ruisseau qui les accompagnait dans leur descente.
Axel sombre dans le désespoir.
Sa lampe s’éteint peu à peu.
Paniqué, il se cogne la tête et perd connaissance.
L’obscurité est alors totale.
Il se réveille enfin et perçoit au loin un bruit.
La paroi conduit sa voix et celle de son oncle.
Ils sont à une lieue de distance.
Axel suit le chemin du son et trébuche.
À son réveil, il se trouve près de ses compagnons. Heureux de les revoir!!
Il découvre la mer Lidenbrock, recluse dans une immense excavation.
D’énormes champignons et des os gigantesques, témoins d’une nature d’un autre temps, les émerveillent.\\

L’exploration de ce nouveau monde se poursuit sur la mer Lidenbrock. 
Hans pêche un poisson aveugle d’une autre époque, ce qui provoque chez Axel une rêverie au sujet de l’évolution et des animaux préhistoriques. 
L’absence de terre en vue suscite l’impatience du professeur. 
Soudain, un terrible combat éclate entre deux animaux marins préhistoriques, puis le calme revient. 
On aperçoit alors au loin une masse monstrueuse qui est en fait un geyser. 
L’obstacle est contourné et la navigation se poursuit.
Le temps est à l’orage et le professeur d’humeur massacrante. 
C’est alors qu’un disque de feu les menace, emporte le mât, puis disparait. 
Ils touchent enfin la terre ferme.\\

Le professeur Lidenbrock se rend compte que le vent les a ramenés sur le rivage qu’ils avaient quitté. 
Hans répare les avaries tandis que le professeur et son neveu explorent la région. 
Ils rencontrent un immense cimetière d’animaux préhistoriques et, soudain, un crâne humain, puis plusieurs squelettes entiers. 
Le professeur se met à disserter comme s’il était dans un amphithéâtre.
Ils découvrent des arbres colossaux, des mastodontes effrayants et un être géant qui les commande. 
Faisant demi-tour, Axel trouve un poignard rouillé. 
Peu après, ils découvrent deux lettres gravées dans un tunnel : «   A.S.  » comme Arne Saknussemm. 
Les investigations se poursuivent, mais les deux Allemands se heurtent à un rocher qui obstrue le passage : il faut le faire sauter.\\

Axel met le feu à la mèche. 
Suite à l’explosion, un abime s’ouvre qui aspire la mer et les emporte avec le radeau sur lequel ils s’étaient retirés. 
La chute est vertigineuse. 
Puis, arrivé dans un puits, le radeau remonte progressivement. 
La chaleur devient soudain insupportable et l’eau bouillante : une éruption volcanique se prépare. 
Ils étaient en fait dans la cheminée d’un volcan en activité. 
Axel s’évanouit sous la chaleur.
Au réveil d’Axel, les voyageurs sont à flanc de montagne et à moitié nus, dans une magnifique région. 
C’est le Stromboli. Ils se font passer pour des naufragés.
Le 9 septembre, ils sont de retour à Hambourg. 
Le professeur raconte son voyage aux scientifiques incrédules. 
De ce voyage, un livre est tiré, qui leur vaut la célébrité dans le monde entier. 
Heureux et fier de son oncle, Axel épouse Graüben.\\
}

\end{document}

