\documentclass[10pt,a4paper]{article}
\usepackage[utf8]{inputenc}
\usepackage[french]{babel}
\frenchbsetup{StandardLists=true}
\usepackage[T1]{fontenc}
\usepackage{amsmath}
\usepackage{amsthm}
\usepackage{amsfonts}
\usepackage{amssymb}
\usepackage{graphicx}
\usepackage{framed}
\usepackage{fancyhdr}
\usepackage[left=1.3cm,right=1.3cm,top=1.8cm,bottom=1.2cm]{geometry}
\usepackage{array} 
\usepackage{tabularx}
\usepackage{fancybox}
\usepackage{pst-tree}
\usepackage[framed]{ntheorem}
\usepackage{tabularx}
\usepackage{pstricks-add}
\usepackage{graphicx}
\usepackage{eurosym}
%\usepackage{pst-tree}
\usepackage[np]{numprint}
\usepackage{pifont}
\usepackage{mathrsfs}
\usepackage{amssymb}
\usepackage{amsthm}
\usepackage{pgf,tikz}
\usepackage[tikz]{bclogo}
\usepackage{pgfkeys}
\usepackage{mathrsfs}
\usepackage{multicol}
\usetikzlibrary{arrows}


\rfoot{\small -\thepage-}
\cfoot{}

\def\R{{\mathbb R}}
\def\Q{{\mathbb Q}}
\def\Z{{\mathbb Z}}
\def\D{{\mathbb D}}
\def\N{{\mathbb N}}
\def\C{{\mathbb C}}

\pagestyle{fancy}

\renewcommand{\thesection}{\Roman{section}}
\renewcommand{\thesubsection}{\arabic{subsection}}
\renewcommand{\thesubsubsection}{\alph{subsubsection}}
\renewcommand{\labelitemi}{$\bullet$}
\newcommand{\VE}[1]{\overrightarrow{#1}}
%\renewcommand{\r}{($O$ ; $\vec{i}$ , $\vec{j}$)}
\renewcommand{\arraystretch}{1}
\tikzstyle{mybox} = [draw=black, very thick, rectangle, rounded corners, inner sep=20pt, inner ysep=20pt] 
\tikzstyle{fancytitle} =[draw=black, very thick, rectangle, rounded corners, fill=white, text=black] % fill obligé sinon ne recouvre pas boite du dessous
\usepackage{array,multirow,makecell}
\setcellgapes{1pt}
\makegapedcells
\newcolumntype{R}[1]{>{\raggedleft\arraybackslash }b{#1}}
\newcolumntype{L}[1]{>{\raggedright\arraybackslash }b{#1}}
\newcolumntype{C}[1]{>{\centering\arraybackslash }b{#1}}
\theoremstyle{break}
\theorembodyfont{\upshape}
\newtheorem{Prop}{Propri\'et\'e}
\newtheorem{Def}{D\'efinition}
\newtheorem{Rem}{Remarque}
\newtheorem{exo}{Exercice}
\newtheorem{Meth}{Methode}
\newtheorem{cpreuve}{Preuve}
\newtheorem{Th}{Théorème}
\newtheorem{Act}{Activité}
\theorembodyfont{\small \sffamily }
\newtheorem{Ex}{Exemple}
\newtheorem{Preu}{Preuve}
\everymath{\displaystyle}
\pagestyle{fancy}
\lhead[]{\small{06 avril 2020} }
\chead[]{\textsc{\shadowbox{\begin{large}
				\textcolor{black}{QUESTIONS du JOUR }
\end{large}}}}
\rhead[]{\small {06 avril 2020}}
\fancyhead[R]{2019/2020}
\lfoot{}
\rfoot{\small -\thepage-}

\begin{document}





	\begin{tabular}{|p{2cm}|p{14cm}| }
		\hline
		&Enoncé\\
		\hline
		1&Soit $f(x)=-2(-3x+6)(x-7)$. Déterminer le tableau de signes de $f(x)$
		\newline 
		\newline 
		\newline

	\\
		\hline
		2&Soit $g(x)=3x^2-3x-18$ définie sur $\R$.
		\begin{enumerate}
			\item Calculer $g(3)$.\newline
			\item Déterminer l'image de -2 par $g$.\newline
			\item En déduire une factorisation de $g(x)$
			
		\end{enumerate}\\
		
		\hline
	
		3&La courbe ci-dessous est la représentation graphique d'une fonction $f$.
		
		
			\begin{center}
				\psset{unit=0.8cm,algebraic=true}
				\def\xmin {-3}
				\def\xmax {4}
				\def\ymin {-4.5}
				\def\ymax {6}
				\begin{pspicture*}(\xmin,\ymin)(\xmax,\ymax)
				\psgrid[subgriddiv=2,gridlabels=3pt,gridwidth=0.5pt,griddots=10,subgriddots=10](\xmin,\ymin)(\xmax,\ymax)
				\psaxes{->}(0,0)(\xmin,\ymin)(\xmax,\ymax)
				
				\psplot[linewidth=1pt,linestyle=dashed]{-2.8}{3.5}{2*x*x*x-6*x*x-x+6}
				
				\uput[r](3,2){$\mathcal{C}_f$}
				\psdot[dotstyle=+](1,1)
				
				\end{pspicture*}
			\end{center}
			\begin{enumerate}
				\item 	Quelle est l'image de 1 par la fonction $f$ ?
				\item On a $f'(1)=-7$. Déterminer l'équation de  la tangente en 1 à la courbe.\newline
				\item Donner le tableau de signe de $f'(x)$ sur $\R$\newline \newline
			\end{enumerate}
		
	\\
		\hline
		
	
		4&Soit la fonction $g(x)=3x^3-4x$ définie et dérivable sur $\R$
		\begin{enumerate}
			\item Quelle est la dérivée $g'$\newline
			\item Factoriser $g'(x)$ puis étudier le signe de $g'(x)$\newline
			\item En déduire le tableau de variations de $g$.\newline \newline
		\end{enumerate}
	\\
		\hline
	
\end{tabular}

\end{document}
