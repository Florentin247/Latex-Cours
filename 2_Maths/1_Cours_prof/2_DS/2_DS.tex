\documentclass[10pt,a4paper]{article}
\usepackage[utf8]{inputenc}
\usepackage[french]{babel}
\frenchbsetup{StandardLists=true}
\usepackage[T1]{fontenc}
\usepackage{amsmath}
\usepackage{amsthm}
\usepackage{amsfonts}
\usepackage{amssymb}
\usepackage{graphicx}
\usepackage{framed}
\usepackage{fancyhdr}
\usepackage[left=1.3cm,right=1.3cm,top=1.8cm,bottom=1.2cm]{geometry}
\usepackage{array} 
\usepackage{fancyhdr} 
\usepackage{fancybox}
\usepackage{pst-tree}
\usepackage[framed]{ntheorem}
\usepackage{tabularx}
\usepackage{pstricks-add}
\usepackage{eurosym}
%\usepackage{pst-tree}
\usepackage[np]{numprint}
\usepackage{pifont}
\usepackage{mathrsfs}
\usepackage{amssymb}
\usepackage{amsthm}
\usepackage{pgf,tikz}
\usepackage[tikz]{bclogo}
\usepackage{pgfkeys}
\usepackage{mathrsfs}
\usepackage{multicol}
\usetikzlibrary{arrows}


\rfoot{\small -\thepage-}
\cfoot{}

\def\R{{\mathbb R}}
\def\Q{{\mathbb Q}}
\def\Z{{\mathbb Z}}
\def\D{{\mathbb D}}
\def\N{{\mathbb N}}
\def\C{{\mathbb C}}

\pagestyle{fancy}

\renewcommand{\thesection}{\Roman{section}}
\renewcommand{\thesubsection}{\arabic{subsection}}
\renewcommand{\thesubsubsection}{\alph{subsubsection}}
\renewcommand{\labelitemi}{$\bullet$}
\newcommand{\VE}[1]{\overrightarrow{#1}}
%\renewcommand{\r}{($O$ ; $\vec{i}$ , $\vec{j}$)}
\renewcommand{\arraystretch}{1}
\tikzstyle{mybox} = [draw=black, very thick, rectangle, rounded corners, inner sep=20pt, inner ysep=20pt] 
\tikzstyle{fancytitle} =[draw=black, very thick, rectangle, rounded corners, fill=white, text=black] % fill obligé sinon ne recouvre pas boite du dessous
\usepackage{array,multirow,makecell}
\setcellgapes{1pt}
\makegapedcells
\newcolumntype{R}[1]{>{\raggedleft\arraybackslash }b{#1}}
\newcolumntype{L}[1]{>{\raggedright\arraybackslash }b{#1}}
\newcolumntype{C}[1]{>{\centering\arraybackslash }b{#1}}
\theoremstyle{break}
\theorembodyfont{\upshape}
\newtheorem{Prop}{Propri\'et\'e}
\newtheorem{Def}{D\'efinition}
\newtheorem{Rem}{Remarque}
\newtheorem{exo}{Exercice}
\newtheorem{Meth}{Methode}
\newtheorem{cpreuve}{Preuve}
\newtheorem{Th}{Théorème}
\newtheorem{Act}{Activité}
\theorembodyfont{\small \sffamily }
\newtheorem{Ex}{Exemple}
\newtheorem{Preu}{Preuve}
\everymath{\displaystyle}
\pagestyle{fancy}
\lhead[]{\small NOM : .......}
\chead[]{\textsc{\shadowbox{\begin{large}
				\textcolor{black}{DEVOIR SURVEILLE n°2}
\end{large}}}}
\rhead[]{\small {}}
\fancyhead[R]{2019/2020}
\lfoot{}
\rfoot{\small -\thepage-}

\begin{document}
	\textbf{Il sera tenu compte du soin apporté à la rédaction et à la présentation dans l'évaluation des copies.
		Toutes les réponses devront être justifiées.}
\begin{exo}[AUTOMATISMES. 10 min] 
\textbf{SANS CALCULATRICE}. A Compléter sur le sujet (aucun détail de calcul n'est imposé)\\
{\renewcommand{\arraystretch}{2}
\begin{tabular}{|p{1cm}|p{10cm}|p{5.5cm}| }
	\hline
&Enoncé&Réponse\\
\hline
1&Mettre sous forme d'une fraction irréductible $A= 2-\dfrac74$&\\
\hline
2&Mettre sous forme d'une fraction irréductible $B= \dfrac{12}5\times\dfrac{20}9$&\\
\hline
3&Compléter $\dfrac25\times \cdots=3$&\\
\hline
4&Développer $-3x(1-2x)$&\\
\hline
5&Factoriser $(-3x+1)(1-2x)-(1-2x)^2$&\\
\hline
6&On sait que $f(x)=x^2-4x$ .Calculer l'image de $-2$&\\
\hline
\end{tabular}}

\end{exo}
\begin{exo}
	On considère la suite définie sur $\N^*$ par  :$u_n= -n^2+2n+15$.
	\begin{enumerate}
	\item A l'aide de votre calculatrice, émettre une conjecture sur les variations de la suite $(u_n)$.
	\item 
	\begin{enumerate}
		\item Pour tout $n>0$, exprimer $u_{n+1}$ en fonction de $n$.  
		\item Montrer que pour tout $n>0$,  $u_{n+1}-u_n=1-2n$. 
		\item Démontrer la conjecture sur les variations faite à la question 1.
	\end{enumerate}
	\end{enumerate}
\end{exo}
\begin{exo}
		Au premier janvier 2 010, Bob a reçu 200 000 $\geneuro$ en héritage. Il décide de placer cette somme   et trouve un placement au taux de $6\%$. Mais chaque année il doit retirer 9 000 $\geneuro$ pour payer les impôts dus à ce placement. \\
	On appelle $C_n$ le capital acquis au 1er janvier de l'année $2010+n$  avec  $C_0=200~000$ 
	\begin{enumerate}
		\item Quel est le montant du capital au premier janvier 2011? 
		\item Expliquer pourquoi la suite $(C_n)$ vérifie la relation de récurrence  : $C_{n+1}=1,06C_n-9000$
		\item Soit l'algorithme suivant:\\
		\begin{tabular}{|l|}
			\hline
			$n \leftarrow 0$\\
			$u \leftarrow 200000$\\
			Tant que $u<240000$\\
			\hspace{0.3cm}
			\begin{tabular}{|l}
			$n \leftarrow n+1$\\
			$u \leftarrow1,06u-9000$
			\end{tabular}\\
			Fin Tant que\\
			afficher $n$\\
			\hline
		\end{tabular}\\
	
	Expliquer ce que permet de trouver cet algorithme ?
		\item Bob veut acheter une maison à 280 000$\geneuro$. En quelle année, aura-t-il la somme nécessaire à cet achat ?
	\end{enumerate}
\end{exo}
\begin{exo}
	
	On prend une feuille de papier dont l' épaisseur est de l'ordre de 0,1 mm.
	On plie une première fois cette feuille de papier en deux, puis de nouveau  en deux et l' on continue ainsi.\\
	On suppose qu'on peut la plier autant de fois que l'on veut.\\
Combien de pliages seront nécessaires pour que l'épaisseur de la feuille pliée dépasse la hauteur de la statue de la liberté de New York, sachant qu'elle mesure $93$m.
\end{exo}
\begin{exo}
	Déterminer la forme algébrique des nombres complexes suivants:\textbf{(détails des calculs attendus)}
	\begin{multicols}{3}
			\begin{enumerate}
				\item $z_A=(2+3i)-5(4-5i)$
				\item $z_B=(4-5i)^2$
				\item $z_C=\dfrac1{2-3i}$
			\end{enumerate}
	\end{multicols}

\end{exo}

\newpage
\begin{exo}[AUTOMATISMES. 10 min] 
	\textbf{SANS CALCULATRICE}. A Compléter sur le sujet (aucun détail de calcul n'est imposé)\\
	{\renewcommand{\arraystretch}{2}
		\begin{tabular}{|p{1cm}|p{10cm}|p{5.5cm}| }
			\hline
			&Enoncé&Réponse\\
			\hline
			1&Mettre sous forme d'une fraction irréductible $A= 2-\dfrac74$&$A=\dfrac14$\\
			\hline
			2&Mettre sous forme d'une fraction irréductible $B= \dfrac{12}5\times\dfrac{20}9$&$B=\dfrac16{4}$\\
			\hline
			3&Compléter $\dfrac25\times \cdots=3$&$\dfrac25\times \dfrac{15}{2}=3$\\
			\hline
			4&Développer $E=-3x(1-2x)$&$E=6x^2-3x$\\
			\hline
			5&Factoriser $F=(-3x+1)(1-2x)-(1-2x)^2$&$F=-x(1-2x)$\\
			\hline
			6&On sait que $f(x)=x^2-4x$ .Calculer l'image de $-2$&$f(-2)=12$\\
			\hline
		\end{tabular}}
		
	\end{exo}
	\begin{exo}
		On considère la suite définie sur $\N^*$ par  :$u_n= -n^2+2n+15$.
		\begin{enumerate}
			\item A l'aide de votre calculatrice, $u_1=16, u_2=15, u_3=12$....il semble que la suite soit décroissante. On peut également utiliser le nuage de points.
			\item 
			\begin{enumerate}
				\item Pour tout $n>0$,  $u_{n+1}=-(n+1)^2+2(n+1)+15=-n^2+16$.
				\item Pour tout $n>0$,  $u_{n+1}-u_n=-n^2+16-(-n^2+2n+15)=-2n+1$. 
				\item Pour tout $n>0$,  $u_{n+1}-u_n=-2n+1$. Or $n \geq 1$ donc $-2n+1 \leq -1$ donc $-2n+1$ est négatif, donc la suite $(u_n)$ est décroissante.
			\end{enumerate}
		\end{enumerate}
	\end{exo}
	\begin{exo}
		Au premier janvier 2 010, Bob a reçu 200 000 $\geneuro$ en héritage. Il décide de placer cette somme   et trouve un placement au taux de $6\%$. Mais chaque année il doit retirer 9 000 $\geneuro$ pour payer les impôts dus à ce placement. \\
		On appelle $C_n$ le capital acquis au 1er janvier de l'année $2010+n$  avec  $C_0=200~000$ 
		\begin{enumerate}
			\item Le montant du capital au premier janvier 2011 est égal à $C_1 = 200000*1.06-9000=203~000 $$\geneuro$
			\item Chaque année le placement rapporte $6\%$ il faut donc multiplier la somme acquise par 1.06. Dans le même temps, Bob retire chaque année $9000 \geneuro$; On a donc pour tout  $n >0$, $C_{n+1}=1,06C_n-9000$
			\item Soit l'algorithme suivant:\\
			\begin{tabular}{|l|}
				\hline
				$n \leftarrow 0$\\
				$u \leftarrow 200000$\\
				Tant que $u<240000$\\
				\hspace{0.3cm}
				\begin{tabular}{|l}
					$n \leftarrow n+1$\\
					$u \leftarrow1,06u-9000$
				\end{tabular}\\
				Fin Tant que\\
				afficher $n$\\
				\hline
			\end{tabular}\\
			
			Cet algorithme affiche le rang $n$ pour lequel $C_n$ sera supérieur ou égal à 240000.
			\item Il s'agit de trouver $n$ pour lequel à $C_n > 280 000$. Pour $n=16$ on a $C_n=277018$ et pour $n=17$ on obtient $C_n=284639$. Bob pourra donc acheter sa maison en $2010+17=2037$.
		\end{enumerate}
	\end{exo}
	\begin{exo}
		
		On prend une feuille de papier dont l' épaisseur est de l'ordre de 0,1 mm.
		On plie une première fois cette feuille de papier en deux, puis de nouveau  en deux et l' on continue ainsi.\\
		On suppose qu'on peut la plier autant de fois que l'on veut.\\
		On pose $e_0=0.1$; Au premier pliage $e_1=2*0.1$. Au deuxième pliage $e_2=2^2 *0.1$. Au nième pliage $e_n=2^n * 0.1$.
		On cherche à savoir pour quel $n$, $e_n > 93 *10^3$ avec la calculatrice on trouve $n=18$.
		Il faudra donc 18 pliages  pour que l'épaisseur de la feuille pliée dépasse la hauteur de la statue de la liberté de New York.
	\end{exo}
	\begin{exo}
		Déterminer la forme algébrique des nombres complexes suivants:\textbf{(détails des calculs attendus)}
		
			\begin{enumerate}
				\item $z_A=(2+3i)-5(4-5i)=-18+28i$
				\item $z_B=(4-5i)^2=16-40i-25=-9-40i$
				\item $z_C=\dfrac1{2-3i}=\dfrac{2+3i}{(2-3i)(2+3i)}=\dfrac{2+3i}{(4-9i^2)=\dfrac{2+3i}{(13)=\dfrac{2}13+\dfrac3{13}i$
			\end{enumerate}
		
		
	\end{exo}
\end{document}
