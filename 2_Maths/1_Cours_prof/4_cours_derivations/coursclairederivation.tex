\documentclass[10pt,a4paper]{article}
\usepackage[utf8]{inputenc}
\usepackage[french]{babel}
\frenchbsetup{StandardLists=true}
\usepackage[T1]{fontenc}
\usepackage{amsmath}
\usepackage{amsthm}
\usepackage{amsfonts}
\usepackage{amssymb}
\usepackage{graphicx}
\usepackage{framed}
\usepackage{fancyhdr}
\usepackage[left=1.3cm,right=1.3cm,top=1.8cm,bottom=1.2cm]{geometry}
\usepackage{array} 
\usepackage{fancyhdr} 
\usepackage{fancybox}
\usepackage{pst-tree}
\usepackage[framed]{ntheorem}
\usepackage{tabularx}
\usepackage{pstricks-add}
\usepackage{eurosym}
%\usepackage{pst-tree}
\usepackage[np]{numprint}
\usepackage{pifont}
\usepackage{mathrsfs}
\usepackage{amssymb}
\usepackage{amsthm}
\usepackage{pgf,tikz}
\usepackage[tikz]{bclogo}
\usepackage{pgfkeys}
\usepackage{mathrsfs}
\usepackage{multicol}
\usetikzlibrary{arrows}
\usepackage{listingsutf8}
\lstset{%
	language=python,%
	basicstyle=\ttfamily,%
}

\rfoot{\small -\thepage-}
\cfoot{}

\def\R{{\mathbb R}}
\def\Q{{\mathbb Q}}
\def\Z{{\mathbb Z}}
\def\D{{\mathbb D}}
\def\N{{\mathbb N}}
\def\C{{\mathbb C}}

\pagestyle{fancy}

\renewcommand{\thesection}{\Roman{section}}
\renewcommand{\thesubsection}{\arabic{subsection}}
\renewcommand{\thesubsubsection}{\alph{subsubsection}}
\renewcommand{\labelitemi}{$\bullet$}
\newcommand{\VE}[1]{\overrightarrow{#1}}
%\renewcommand{\r}{($O$ ; $\vec{i}$ , $\vec{j}$)}
\renewcommand{\arraystretch}{0.7}
\tikzstyle{mybox} = [draw=black, very thick, rectangle, rounded corners, inner sep=20pt, inner ysep=20pt] 
\tikzstyle{fancytitle} =[draw=black, very thick, rectangle, rounded corners, fill=white, text=black] % fill obligé sinon ne recouvre pas boite du dessous
\usepackage{array,multirow,makecell}
\setcellgapes{1pt}
\newcommand{\Syst}[2]{\left\{\begin{array}{ccc} #1\\ #2 \end{array}\right.}

\makegapedcells
\newcolumntype{R}[1]{>{\raggedleft\arraybackslash }b{#1}}
\newcolumntype{L}[1]{>{\raggedright\arraybackslash }b{#1}}
\newcolumntype{C}[1]{>{\centering\arraybackslash }b{#1}}
\theoremstyle{break}
\theorembodyfont{\upshape}
\newtheorem{Prop}{Propri\'et\'e}
\newtheorem{Def}{D\'efinition}
\newtheorem{Rem}{Remarque}
\newtheorem{exo}{Exercice}
\newtheorem{Meth}{Methode}
\newtheorem{cpreuve}{Preuve}
\newtheorem{Th}{Théorème}
\newtheorem{Act}{Activité}
\theorembodyfont{\small \sffamily }
\newtheorem{Ex}{Exemple}
\newtheorem{Preu}{Preuve}
\everymath{\displaystyle}
\pagestyle{fancy}
\fancyhead[L]{1STI2D}
\fancyhead[C]{COURS SUITES }
\fancyhead[R]{2019/2020}
\rfoot{\small -\thepage-}
\cfoot{}

\fancyhead[C]{\shadowbox{\begin{large}
			\textcolor{black}{COURS DÉRIVATION1}
		\end{large}}}
\fancyhead[R]{2019/2020}
\rfoot{\small -\thepage-}
\cfoot{}

\begin{document}

	
	\begin{center}
		\shadowbox{\begin{large}
				\textcolor{black}{DÉRIVATION: Première partie}
		\end{large}}
	\end{center}
\tableofcontents

\section{Un peu d'histoire}
Regarder les sites suivants:\\
\textit{https://maths.discip.ac-caen.fr/IMG/pdf/histoire des  derivees.pdf}\\
\textit{https://youtu.be/X6S9R8sN6do}
\section{Nombre dérivé}
\subsection{Taux de variation d'une fonction }
	\begin{bclogo}[couleur = yellow!30, arrondi = 0.1,logo=\bcbook]{Définition}
	On considère une fonction $f$ définie sur un intervalle $I$ non vide ainsi que deux réels $a$ et $h$ avec $h\neq 0$ tels que $a\in I$ et $a+h\in I$.
	Le {taux de variations} de $f$ en $a$  est le nombre défini par :
	\begin{center}
		\boldmath
		$\tau_a(h)=\frac{f(a+h)-f(a)}{h}$
		\unboldmath
	\end{center}
	
	\end {bclogo}
	\begin{Rem}[Interprétation graphique]
		Soit $\mathscr{C}_f$ la courbe représentative de $f$ et soit $A$ et $M$ les points de $\mathscr{C}_f$ de coordonnées respectives :$A(a;f(a))$ et $M(a+h;f(a+h))$.\\
	Le taux de variation de $f$ entre $a$ et $a+h$ correspond au \textbf{coefficient directeur} de la droite $(AM)$.\\
	\end{Rem}

Illustration:Graphique 1
	\begin{center}
		\psset{xunit=1cm,yunit=1cm,algebraic=true}
		\def\xmin {-1.5}
		\def\xmax {5}
		\def\ymin {-1}
		\def\ymax {5}
		\begin{pspicture*}(\xmin,\ymin)(\xmax,\ymax)
		\psgrid[subgriddiv=2,gridlabels=3pt,gridwidth=0.5pt,griddots=10,subgriddots=10](\xmin,\ymin)(\xmax,\ymax)
		\psaxes[labels=none]{->}(0,0)(\xmin,\ymin)(\xmax,\ymax)
		\pcline[linewidth=1pt]{->}(0,0)(1,0) \uput[d](0.2,0){\small $\vec i$}
		\pcline[linewidth=1pt]{->}(0,0)(0,1) \uput[l](0,0.5){\small $\vec j$}
		\psplot[linewidth=1pt]{\xmin}{\xmax}{-0.2*(x+1)*(x-7)}
		\psline[linecolor=red](-2.75,0)(7.3,6.03)
	\psline[linestyle=dashed,linecolor=red](0.5,1.95)(0.5,0)
	\psline[linestyle=dashed,linecolor=red](2.5,3.15)(2.5,0)
	\psline[linestyle=dashed,linecolor=red](0.5,1.95)(0,1.95)
	\psline[linestyle=dashed,linecolor=red](2.5,3.15)(0,3.15)
	\psline[linecolor=red]{->}(0.5,1.95)(2.5,1.95)
	\psline[linecolor=red]{->}(2.5,1.95)(2.5,3.15)
	\psdots[dotstyle=+](0.5,1.95)(2.5,3.15)
	\uput[u](0.5,1.95){A}
	\uput[u](2.5,3.15){M}
	\uput[l](0,1.95){$f(a)$}
	\uput[l](0,3.15){$f(a+h)$}
	\uput[d](0.5,0){$a$}
	\uput[d](2.5,0){$a+h$}
		\uput[d](1.5,1.95){$h$}
			\uput[r](2.5,2.6){$f(a+h)-f(a)$}
		\end{pspicture*}
	\end{center}
		\begin{Meth}[ Savoir calculer un taux de variation ]
		\begin{enumerate}
				\item Soit la fonction f définie sur $\R$ par $f(x)=2x^2+1$.  Calculer le taux de variation de $f$ en $a = 1$
		\item Soit la fonction g définie sur $\R$ par $f(x)=3x+2$.
		Calculer le taux de variation de $f$ en $a = -1$
	\end{enumerate}
	\end{Meth}
\subsection{Nombre dérivé }
	\begin{bclogo}[couleur = yellow!30, arrondi = 0.1,logo=\bcbook]{Définition}
	On dit que \textbf{$f$ est dérivable en $a$} si la limite du taux de variation lorsque $h$ tend vers 0 est un réel un réel $l$.\\
	Cette limite lorsqu'elle existe, est notée $f'(a)$ et est appelée \textbf{nombre dérivé de $f$ en $a$}
On a :\boldmath\[ f'(a)=\lim_{h\to0}\tau_a(h)=\lim_{h\to0}\frac{f(a+h)-f(a)}{h}=l \]\unboldmath
\end{bclogo}
\begin{Rem}
	En physique, la vitesse $v(t)$ en un point de date $t$ est le nombre dérivé de la fonction $x =f(t)$. On note $v(t) =f'(t) =\dfrac{\text{d}x}{\text{d}t}.$
\end{Rem}
\begin{Meth}[Savoir montrer qu'une fonction est dérivable en a et déterminer le nombre dérivé $f'(a)$ ]
\begin{itemize}
	\item On calcule le taux de variations de la fonction en $a$;
	\item On détermine sa limite lorsque $h$ tend vers 0.
	\item Si la limite est finie alors la fonction est dérivable et on obtient $f'(a)$.\\
	Sinon la fonction n'est pas dérivable en $a$.
\end{itemize}
	\begin{enumerate}
		\item Soit la fonction $f$ définie sur $\R $ par $f(x)=3x^2-5$.\\Démontrer que $f$ est dérivable en 3 et calculer $f'(3)$.
		\item Soit la fonction $g$ définie sur $\R$ par $f(x)=x^2+x+1$. Déterminer le nombre dérivé de $g$ en 1.
	\end{enumerate}
\end{Meth}
\subsection{Nombre dérivé et tangente à une courbe}
En reprenant le graphique 1 , lorsque $h$ va tendre vers $0$ le point $M$ se rapproche du point $A$ et la droite $(AM)$ tend vers une \textbf{position limite qui s'appelle la tangente à la courbe au point d'abscisse $a$}.\\
Illustration:
\begin{center}
	\psset{xunit=1cm,yunit=1cm,algebraic=true}
	\def\xmin {-1.5}
	\def\xmax {5}
	\def\ymin {-1}
	\def\ymax {5}
	\begin{pspicture*}(\xmin,\ymin)(\xmax,\ymax)
	\psgrid[subgriddiv=2,gridlabels=3pt,gridwidth=0.5pt,griddots=10,subgriddots=10](\xmin,\ymin)(\xmax,\ymax)
	\psaxes[labels=none]{->}(0,0)(\xmin,\ymin)(\xmax,\ymax)
	\pcline[linewidth=1pt]{->}(0,0)(1,0) \uput[d](0.2,0){\small $\vec i$}
	\pcline[linewidth=1pt]{->}(0,0)(0,1) \uput[l](0,0.5){\small $\vec j$}
	\psplot[linewidth=1pt]{\xmin}{\xmax}{-0.2*(x+1)*(x-7)}
	\psplot[linecolor=red]{\xmin}{\xmax}{x+1.45}
		\psplot[linecolor=red,linestyle=dashed]{\xmin}{\xmax}{0.6*x+1.65}
			\psplot[linecolor=red,linestyle=dashed]{\xmin}{\xmax}{0.37*x+1.76}
				\psplot[linecolor=red,linestyle=dashed]{\xmin}{\xmax}{0.16*x+1.87}
	\psline[linestyle=dashed,linecolor=red](0.5,1.95)(0.5,0)
	\psline[linestyle=dashed,linecolor=red](0.5,1.95)(0,1.95)

	\psdots[dotstyle=+](0.5,1.95)
	\uput[u](0.5,1.95){A}
	\uput[l](0,1.95){$f(a)$}
	\uput[d](0.5,0){$a$}
	\end{pspicture*}
\end{center}
	\begin{bclogo}[couleur = yellow!30, arrondi = 0.1,logo=\bcbook]{Définition}
		Si $f$ est dérivable en $a$ , la tangente $\mathscr{T}_a$ à la courbe représentative $\mathscr{C}_f$ de $f$ en $a$ est la droite:\begin{enumerate}
			\item qui a pour coefficient directeur $f'(a)$
			\item qui passe par le point $A$ de coordonnées $(a;f(a))$.
		\end{enumerate} 
\end{bclogo}
\begin{center}
	\psset{xunit=1cm,yunit=1cm,algebraic=true}
	\def\xmin {-1.5}
	\def\xmax {5}
	\def\ymin {-1}
	\def\ymax {5}
	\begin{pspicture*}(\xmin,\ymin)(\xmax,\ymax)
	\psgrid[subgriddiv=2,gridlabels=3pt,gridwidth=0.5pt,griddots=10,subgriddots=10](\xmin,\ymin)(\xmax,\ymax)
	\psaxes[labels=none]{->}(0,0)(\xmin,\ymin)(\xmax,\ymax)
	\pcline[linewidth=1pt]{->}(0,0)(1,0) \uput[d](0.2,0){\small $\vec i$}
	\pcline[linewidth=1pt]{->}(0,0)(0,1) \uput[l](0,0.5){\small $\vec j$}
	\psplot[linewidth=1pt]{\xmin}{\xmax}{-0.2*(x+1)*(x-7)}
	\psplot[linecolor=red]{\xmin}{\xmax}{x+1.45}

	\psline[linestyle=dashed,linecolor=red](0.5,1.95)(0.5,0)
	\psline[linestyle=dashed,linecolor=red](0.5,1.95)(0,1.95)
	
	\psdots[dotstyle=+](0.5,1.95)
	\uput[u](0.5,1.95){A}
	\uput[l](0,1.95){$f(a)$}
	\uput[d](0.5,0){$a$}
	\end{pspicture*}
\end{center}
	\begin{bclogo}[couleur = yellow!30, arrondi = 0.1,logo=\bcbook]{Proposition}
	Soit $f$ une fonction dérivable en $a$.\\
	La tangente à  $\mathcal{C}$ au point d'abscisse $a$ a pour équation : $$y=f'(a)(x-a)+f(a)$$
\end{bclogo}
\begin{Meth}
[ Calculer le coefficient directeur d'une tangente et trouver l'équation de la tangente]
	Soit $f$ la fonction définie sur $\mathbb{R}$ par $f(x)=x^2$. Déterminer le coefficient directeur de la tangente à la courbe de $f$ au point d'abscisse 4 puis donner l'équation de la tangente à la courbe au point d'abscisse 4.
\end{Meth}
\begin{Meth}[Lire un nombre dérivé sur un graphique.]
Sur le dessin ci dessous on a représenté la courbe d'une fonction $f$ ainsi que 3 tangentes à la courbe de $f$. Déterminer graphiquement $f'(1)$, $f'(-1)$  et $f'(2)$\\
\begin{center}
		\psset{xunit=1cm,yunit=1cm,algebraic=true}
	\def\xmin {-1.5}
	\def\xmax {5}
	\def\ymin {-1}
	\def\ymax {5}
	\begin{pspicture*}(\xmin,\ymin)(\xmax,\ymax)
	\psgrid[subgriddiv=2,gridlabels=3pt,gridwidth=0.5pt,griddots=10,subgriddots=10](\xmin,\ymin)(\xmax,\ymax)
	\psaxes[labels=none]{->}(0,0)(\xmin,\ymin)(\xmax,\ymax)
	\pcline[linewidth=1pt]{->}(0,0)(1,0) \uput[d](0.2,0){\small $\vec i$}
	\pcline[linewidth=1pt]{->}(0,0)(0,1) \uput[l](0,0.5){\small $\vec j$}
	\psplot[linewidth=1pt]{\xmin}{\xmax}{0.5*(x^3-3*x^2+6)}
	\psplot[linecolor=red]{\xmin}{\xmax}{4.5*x+5.5}
		\psplot[linecolor=red]{\xmin}{\xmax}{-1.5*x+3.5}
			\psplot[linecolor=red]{\xmin}{\xmax}{1}
				\psdots[dotstyle=+](1,2)(-1,1)(2,1)
		
	\end{pspicture*}
\end{center}	
\end{Meth}

\section{Fonction dérivée}
\subsection{Définition}
	\begin{bclogo}[couleur = yellow!30, arrondi = 0.1,logo=\bcbook]{Définition}
Soit $f$ une fonction définie sur un intervalle I de $\R$.
Lorsque le nombre dérivé de $f$ existe pour toutes les valeurs $x$ de I, on dit que $f$ est dérivable sur I.
La fonction qui à $x$ associe son nombre dérivé $f'(x)$ s’appelle la fonction dérivée de $f$ (ou la dérivée
de $f$ ) et elle se note $f’$.
	\begin{center}
		\begin{tabular}{lccl}
			$f':$&$I $&$\longrightarrow$&$\mathbb{R}$\\	
			&$x$&$\mapsto$&$f'(x)$\\
		\end{tabular}
	\end{center}	
\end{bclogo}
\subsection{Dérivées des fonctions usuelles}
Les fonctions dérivées des fonctions usuelles sont données dans le tableau ci dessous:
\vspace*{0.5cm}\\
\renewcommand{\arraystretch}{1.5}
\begin{tabular}{|p{4cm}|p{4cm}|p{4cm}|p{4cm}|}
	\hline
	Fonction $f$&Domaine de dérivabilité&Fonction dérivée $f'$&Exemples\\
	\hline
	&&&\\
	$f(x)=k$ \newline avec $k$ réel&$\R$&$f'(x)=0$&Si $f(x)=-\dfrac{5}{3}$ alors $f$ est dérivable sur $\R$ et $f'(x)=0$ \\
		&&&\\
	\hline
		&&&\\
		$f(x)=mx+p$ \newline avec $m$ et $p$ réels&$\R$&$f'(x)=m$&Si $f(x)=2x-5$ alors $f$ est dérivable sur $\R$ et $f'(x)=2$ \\
			&&&\\
		\hline
			&&&\\
			$f(x)=x^2$ \newline  &$\R$&$f'(x)=2x$& \\
			\hline			
				$f(x)=x^3 $ \newline  &$\R$&$f'(x)=3x^2$& \\
				\hline
\end{tabular}
\subsection{Dérivée et opérations}
Soient u et v deux fonctions dérivables sur un même intervalle I. Alors :
\vspace*{0.5cm}\\
\begin{center}
\renewcommand{\arraystretch}{1.5}
\begin{tabular}{|p{4cm}|p{4cm}|p{8cm}|}
	\hline
	Fonction $f$&Fonction dérivée $f'$&Exemples\\
	\hline
	$ku$ \par avec k réel&$ku'$& si $f(x)=4x^2$ alors $f'(x)=4\times(2x)=8x$\\
	\hline
	$u+v$&$u'+v'$&Si $f(x)=-2x^3+6x^2-3x$ alors\newline $f'(x)=-2\times(3x^2)+6\times(2x)-3\times1=-6x^2+12x-3$\\

	\hline
\end{tabular}
\end{center}
\begin{Meth}[Déterminer la fonction dérivée d'un polynôme]
	\begin{enumerate}
		\item Soit la fonction $f$ définie sur $\R$ par  $f(x)=3x^2+2x+1$ par:\\
				$f$ est un polynôme du second degré donc dérivable sur $\R$ et on a  :\\
		$f'(x) =\cdots\cdots\cdots\cdots$
		\item Soit la fonction $g$ définie sur $\R$ par $g(x)=-5x+2$. g est une fonction affine donc dérivable sur $\R$ et on a $g'(x)=\cdots\cdots\cdots\cdots$
		\item Soit la fonction $h$ définie sur $\R$ apr $h(t)=2t^3+5t^2+3$. $h$ est un polynôme du troisième dégré donc dérivable sur $\R$ et on a $h'(t)=\cdots\cdots\cdots\cdots$
	\end{enumerate}
	\end{Meth}
\begin{Meth}[Déterminer l'équation de la tangente à une courbe en un point donné]
		 $f$ est la fonction définie sur $\R$ par $f(x) =-2x^2+4x$, et $\mathcal{C}_f$ est sa courbe représentative. 
		 \begin{enumerate}
		 	\item Donner une équation de la tangente T à $\mathcal{C}_f$ au point A d’abscisse 0.
		 	\item Étudier la position relative de $\mathcal{C}_f$ par rapport à T.
		 		 \end{enumerate}
\end{Meth}

\section{Du signe de la dérivée aux variations}
~\\
\begin{bclogo}[couleur = yellow!30, arrondi = 0.1,logo=\bcbook]{Proposition}
	Soit $f$ une fonction dérivable sur une intervalle  $I$, alors :
		\begin{itemize}
		\item Si, pour tout $x$ de $I$, $f'(x)  \geq 0 $  alors $f$ est \textbf{ croissante} sur $I$ .
		\item Si, pour tout $x$ de $I$, $f'(x) \leq 0 $  alors $f$ est \textbf{ décroissante} sur $I$ .
		\item Si, pour tout $x$ de $I$, $f'(x) = 0 $ , alors $f$ est \textbf{ constante } sur $I$ .
	\end{itemize}  
	\end{bclogo}
\begin{Meth}[Étudier les variations d'une fonction]
	~\\
	\begin{bclogo}[couleur=green!30,arrondi=0.1,logo=\bcbook]{MÉTHODE}	
		\begin{itemize}
				\item On calcule la fonction dérivée de $f$.
		\item On factorise $f'(x)$ afin d'étudier son signe en utilisant ses connaissances sur les fonctions affines ou les polynômes du second et troisième degré.
		\item On en déduit les variations de $f$ et on résume les résultats dans un tableau de variations.
				\item On vérifie  la concordance entre le tableau et la courbe sur la calculatrice.
	\end{itemize}
\end{bclogo}
	Étudier les variations de la fonction $g$   définie sur $\R$ par $g (x) =-3x^2+12x+1$
\end{Meth}	
\section{Extrema d'une fonction}
\subsection{Définitions}
\begin{bclogo}[couleur = yellow!30, arrondi = 0.1,logo=\bcbook]{Proposition}
	Soit $f$ une fonction définie sur un intervalle $I$ et soit $c$ un réel de $I$ :
	\begin{itemize}
		\item Dire que $f(c)$ est un maximum\index{maximum} de $f$ signifie que pour tout  $x\in I$, $f(x)\leqslant f(c)$;
		\item dire que $f(c)$ est un minimum\index{minimum} de $f$ signifie que pour tout $x\in I$, $f(x)\geqslant f(c)$.
	\end{itemize}
\end{bclogo}
\begin{Ex}
\begin{multicols}{2}
	
Ci-contre $\mathscr C$ est la représentation graphique d'une fonction définie sur $I=[-2;4]$. \newline
		On constate que $f(-1)=0,7$ est un extremum local de $f$  (maximum local) car pour tout $x\in \left[ -2;0\right] $ on a $f(x)\leqslant 0,7=f(-1)$ . Par contre 0,7 n'est pas le maximum de $f$ sur $\left[ -2;4\right] $\newline
	Par contre  $f(2)=-2$ est le minimum de $f$ sur $\left[ -2;4\right] $ car pour tout $x\in I$ on a $f(x)\geqslant -2$
	
	\begin{center}
	\psset{xunit=1cm,yunit=1cm,algebraic=true}
	\def\xmin {-3}
	\def\xmax {5}
	\def\ymin {-3}
	\def\ymax {4}
	\begin{pspicture*}(\xmin,\ymin)(\xmax,\ymax)
	\psgrid[subgriddiv=2,gridlabels=3pt,gridwidth=0.5pt,griddots=10,subgriddots=10](\xmin,\ymin)(\xmax,\ymax)
	\psaxes[labels=none]{->}(0,0)(\xmin,\ymin)(\xmax,\ymax)
	\pcline[linewidth=1pt]{->}(0,0)(1,0) \uput[d](0.2,0){\small $\vec i$}
	\pcline[linewidth=1pt]{->}(0,0)(0,1) \uput[l](0,0.5){\small $\vec j$}
		\def\F{0.2*x^3-0.3*x^2-1.2*x}
		\psplot[linestyle=solid,plotpoints=1000]{-2}{4}{\F}
		\uput[ur](3.4,1){$\mathscr C$}
				\end{pspicture*}
	\end{center}
\end{multicols}
\end{Ex}

\begin{Meth}
	Soit $g$ la fonction définie sur $\R$ par $g (x) =-x^3-x^2+x+1$.
	\begin{enumerate}
		\item Calculer $g'(x)$
		\item Vérifier que -1 et $\dfrac13$ sont racines de $g'(x)$
		\item En déduire le signe de $g'(x)$
		\item Déterminer le tableau de variations de $g$.
		\item Déterminer les extrema de la fonction $g$
	\end{enumerate}
	\end{Meth}

	\begin{bclogo}[couleur = yellow!30, arrondi = 0.1,logo=\bcbook]{Définition}On considère une fonction $f$ dérivable sur un intervalle ouvert $I$ et $a$ est un réel de $I$.\newline
	Si $f(a)$ est un extremum local de $f$, alors $f'(a)=0$\end{bclogo}

\begin{multicols}{2}
	\begin{Rem}
		La réciproque de ce théorème est fausse. \newline 
		Considérons par exemple le cas de la fonction cube, dont la dérivée s'annule en $0$ qui n'est pourtant pas un extremum local. 
		\newline
		Il faut ajouter une hypothèse pour avoir le résultat réciproque, comme suit :
	\end{Rem}
	\begin{center}
		\psset{xunit=0.7cm , yunit=0.7cm}
		\begin{pspicture*}(-2.6,-3.1)(4,3.1)
		\def\xmin{-2.5} \def\xmax{4} \def\ymin{-3} \def\ymax{3}
		\newrgbcolor{couleurcadre}{0.99 0.99 0.99}
		\psframe[linewidth=0.3pt,linecolor=couleurcadre](-2.6,-3.1)(4,3.1)
		\def\pshlabel#1{\psframebox*[framesep=1pt]{\small #1}}
		\def\psvlabel#1{\psframebox*[framesep=1pt]{\small #1}}
		\psclip{%
			\psframe[linestyle=none](\xmin,\ymin)(\xmax,\ymax)
		}
		\newrgbcolor{couleur1}{0.6549 0.6549 0}
		\newrgbcolor{couleur2}{0 0.3176 0.4745}
		\newrgbcolor{couleur3}{0.0941 0.647 0.0196}
		\def\F{x 3 exp}
		\psplot[linecolor=blue,linestyle=solid,plotpoints=1000]{-1.5}{1.5}{\F}
		\uput[ur](1,1){$\mathscr C : y=x^3$}
		\endpsclip
		\psaxes[labels=none,labelsep=1pt,Dx=1,Dy=1,Ox=0,Oy=0]{-}(0,0)(\xmin,\ymin)(\xmax,\ymax)
		\uput[dl](0,0){0}
		\pcline[linewidth=1pt]{->}(0,0)(1,0) \uput[d](0.5,0){\small $\vec i$}
		\pcline[linewidth=1pt]{->}(0,0)(0,1) \uput[l](0,0.5){\small $\vec j$}
		\end{pspicture*}
	\end{center}
\end{multicols}
	\begin{bclogo}[couleur = yellow!30, arrondi = 0.1,logo=\bcbook]{Proposition}
On considère une fonction $f$ dérivable sur un intervalle ouvert $I$ et $a$ un réel de $I$.\newline
	Si $f'$ s'annule en $a$ \textbf{en changeant de signe}, alors $f(a)$ Nous admettrons ces deux propriétés.
\end{bclogo}
\begin{Meth}[Savoir étudier les variations d'une fonction]
	Soit $g$ la fonction définie sur $\R$ par $g (x) =-x^3-x^2+x+1$.
	\begin{enumerate}
		\item Calculer $g'(x)$
		\item Vérifier que -1 et $\dfrac13$ sont racines de $g'(x)$
		\item En déduire le signe de $g'(x)$
		\item Déterminer le tableau de variations de $g$.
		\item Déterminer les extrema de la fonction $g$
	\end{enumerate}
\end{Meth}
\begin{Meth}[Avec une courbe représentative]
	
	Soit $f$ une fonction définie et dérivable sur  $\R$. On note $f'$ la dérivée de la fonction $f$. 
	
	On donne ci-dessous la courbe $\mathcal{C}_{f}$ représentant la fonction $f$.
	
	\begin{itemize}
		\item La courbe $\mathcal{C}_{f}$ passe par les points $A (- 2 ; 0)$, $B (1 ; 1)$, $C (4 ; 3,2)$ et $D \left( \dfrac{11}{2} ; \dfrac{25}{16} \right)$. 
		\item L'axe des abscisses est tangent en $A$ à la courbe $\mathcal{C}_{f}$.
		\item La courbe $\mathcal{C}_{f}$ admet une deuxième tangente parallèle à l'axe des abscisses au point $C$.  
		\item La tangente à la courbe au point $B$ passe par le point $M \left( -4 ; -3 \right)$.
	\end{itemize}
	
	\begin{center}	
		\psset{unit=.5cm}
		\begin{pspicture}(-12,-6)(14,8) 
		\newrgbcolor{bleu}{0.1 0.05 .5}
		\newrgbcolor{prune}{.6 0 .48}
		\newrgbcolor{grisb}{.89 .88 .96}
		\def\pshlabel#1{\footnotesize #1}
		\def\psvlabel#1{\footnotesize #1}
		\def\f{(x+2)^3*(6-x)/135}
		\psgrid[gridwidth=0.25pt,gridcolor=darkgray,subgriddiv=0,gridlabels=0](0,0)(-12,-6)(14,8) 
		\psset{unit=1cm}
		\psaxes[labelsep=.8mm,linewidth=.75pt,ticksize=-2pt 2pt]{->}(0,0)(-6,-3)(7,4) 
		\uput[dl](0,0) {\footnotesize {0}}
		\uput[dl](7,0) {$x$}
		\uput[dl](0,4) {$y$}
		\psplot[algebraic=true,plotpoints=1000,linewidth=1.5pt, linecolor=bleu]{-5.3}{6.63}{\f}
		\psline[linewidth=1pt, linecolor=prune]{<->}(-.5,-.2)(2.5,2.2)
		\psline[linewidth=1pt, linecolor=prune]{<->}(3.25,3.2)(4.75,3.2)
		\psdots [linewidth=1pt, linecolor=bleu,dotscale=.8](-2,0)(1,1)(4,3.2)(5.5,1.5625)
		\uput[ul](-5.4,-3){\bleu{$\mathcal{C}_{f}$}}
		\uput[u](-2,0){\bleu{$A$}}
		\uput[ul](1,1){\bleu{$B$}}
		\uput[u](4,3.2){\bleu{$C$}}
		\uput[ur](5.5,1.5){\bleu{$D$}}
		\end{pspicture}
	\end{center}
	
	\`A partir du graphique et des données de l'énoncé, répondre aux questions suivantes.
	\begin{enumerate}
		\item Dresser sans justification le tableau de variations de la fonction $f$ sur $\R$.
		\item Déterminer $f'(-2)$, $f'(4)$ et $f'(1)$.
		\item Quel est l'ensemble solution de l'inéquation $f'(x) \geqslant 0$ ?
		\item On donne $f'(5,5)= -2,5$. Calculer les coordonnées du point d'intersection de la tangente à la courbe 	$\mathcal{C}_{f}$ au point $D$ avec l'axe des ordonnées.
		\item  Une des trois courbes ci-dessous est la représentation graphique de la fonction $f'$ . Déterminer laquelle.
		
		\medskip
		\psset{unit=0.5cm}
		\begin{pspicture}(-4,-5)(6,3) 
		\newrgbcolor{bleu}{0.1 0.05 .5}
		\def\pshlabel#1{\tiny #1}
		\def\psvlabel#1{\tiny #1}
		\def\f{(4-x)^2*(2+x)/32}
		\psgrid[gridwidth=0.25pt,gridcolor=darkgray,subgriddiv=0,gridlabels=0](0,0)(-4,-4)(6,3) 
		\psaxes[labelsep=.8mm,linewidth=.75pt,ticksize=-2pt 2pt,Dx=2,Dy=2]{->}(0,0)(-4,-4)(6,3) 
		\uput[dl](0,0){\footnotesize{0}}
		\uput[dl](6,0){\footnotesize{$x$}} \uput[dl](0,3){\footnotesize{$y$}}
		\psplot[algebraic=true,plotpoints=500,linewidth=1.25pt, linecolor=bleu]{-4}{6}{\f}
		\rput(1,-5){\footnotesize{Courbe $\mathcal{C}_{1}$}}
		\end{pspicture}
		\hfill	
		\psset{unit=0.5cm}
		\begin{pspicture}(-4,-5)(6,3) 
		\newrgbcolor{bleu}{0.1 0.05 .5}
		\def\pshlabel#1{\tiny #1}
		\def\psvlabel#1{\tiny #1}
		\def\f{(4-x)*(2+x)/4}
		\psgrid[gridwidth=0.25pt,gridcolor=darkgray,subgriddiv=0,gridlabels=0](0,0)(-4,-4)(6,3) 
		\psaxes[labelsep=.8mm,linewidth=.75pt,ticksize=-2pt 2pt,Dx=2,Dy=2]{->}(0,0)(-4,-4)(6,3) 
		\uput[dl](0,0){\footnotesize{0}}
		\uput[dl](6,0){\footnotesize{$x$}} \uput[dl](0,3){\footnotesize{$y$}}
		\psplot[algebraic=true,plotpoints=500,linewidth=1.25pt, linecolor=bleu]{-4}{6}{\f}
		\rput(1,-5){\footnotesize{Courbe $\mathcal{C}_{2}$}}
		\end{pspicture}
		\hfill
		\psset{unit=0.5cm}
		\begin{pspicture}(-4,-5)(6,3) 
		\newrgbcolor{bleu}{0.1 0.05 .5}
		\def\pshlabel#1{\tiny #1}
		\def\psvlabel#1{\tiny #1}
		\def\f{(16-4*x)*(x+2)^2/135}
		\psgrid[gridwidth=0.25pt,gridcolor=darkgray,subgriddiv=0,gridlabels=0](0,0)(-4,-4)(6,3) 
		\psaxes[labelsep=.8mm,linewidth=.75pt,ticksize=-2pt 2pt,Dx=2,Dy=2]{->}(0,0)(-4,-4)(6,3) 
		\uput[dl](0,0){\footnotesize{0}}
		\uput[dl](6,0){\footnotesize{$x$}} \uput[dl](0,3){\footnotesize{$y$}}
		\psplot[algebraic=true,plotpoints=500,linewidth=1.25pt, linecolor=bleu]{-4}{6}{\f}
		\rput(1,-5){\footnotesize{Courbe $\mathcal{C}_{3}$}}
		\end{pspicture}
	\end{enumerate}
	
	\bigskip
\end{Meth}
\end{document}






 
