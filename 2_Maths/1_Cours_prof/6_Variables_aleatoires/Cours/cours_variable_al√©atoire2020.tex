\documentclass[10pt,a4paper]{article}
\usepackage[utf8]{inputenc}
\usepackage[french]{babel}
\frenchbsetup{StandardLists=true} % à inclure si on utilise \usepackage[french]{babel}
\usepackage[T1]{fontenc}
\usepackage{amsmath}
\usepackage{amsthm}
\usepackage{amsfonts}
\usepackage{amssymb}
\usepackage{graphicx}
\usepackage{framed}
\usepackage{fancyhdr}
\usepackage[left=1.2cm,right=1.2cm,top=1.8cm,bottom=1.3cm]{geometry}
\usepackage{array} 
\usepackage{fancyhdr} 
\usepackage{fancybox}
\usepackage{pst-tree}
\usepackage[framed]{ntheorem}
\usepackage{tabularx}
\usepackage{pstricks,pstricks-add}
\usepackage{eurosym}
\usepackage{multicol}
\usepackage{tikz,tkz-tab}
\usepackage[tikz]{bclogo}
\usepackage{mathrsfs}
\usepackage{numprint}
\usepackage{pifont}
\rfoot{\small -\thepage-}
\cfoot{}
%papier millimétré
\newcommand{\mili}[4]{\psgrid[subgriddiv=10, gridlabels=0, gridwidth=0.4pt, subgridwidth=0.4pt,gridcolor=brown!80,subgridcolor=brown!40](#1,#2)(#3,#4)}

% Raccourcis diverses:
\newcommand{\nwc}{\newcommand}
\nwc{\dsp}{\displaystyle}
\nwc{\ct}{\centerline}
\nwc{\bgar}{\begin{array}}\nwc{\enar}{\end{array}}
\nwc{\bgit}{\begin{itemize}}\nwc{\enit}{\end{itemize}}
\nwc{\bgen}{\begin{enumerate}}\nwc{\enen}{\end{enumerate}}

\nwc{\la}{\left\{}\nwc{\ra}{\right\}}
\nwc{\lp}{\left(}\nwc{\rp}{\right)}
\nwc{\lb}{\left[}\nwc{\rb}{\right]}

\def\R{{\mathbb R}}
%\def\Q{{\mathbb Q}}
\def\Z{{\mathbb Z}}
%\def\D{{\mathbb D}}
\def\N{{\mathbb N}}
%\def\C{{\mathbb C}}


\nwc{\bgsk}{\bigskip}
\nwc{\vsp}{\vspace{0.1cm}}
\nwc{\vspd}{\vspace{0.2cm}}
\nwc{\vspt}{\vspace{0.3cm}}
\nwc{\vspq}{\vspace{0.4cm}}


\def\epsi{\varepsilon}
\def\vphi{\varphi}
\def\lbd{\lambda}

\def\Cf{\mathcal{C}_f}


\pagestyle{fancy}
\newcommand{\Oij}{$\left( {{\mathrm{O}};\vec i,\vec j} \right)$}
\renewcommand{\thesection}{\Roman{section}}
\renewcommand{\thesubsection}{\arabic{subsection}}
\renewcommand{\thesubsubsection}{\alph{subsubsection}}
\newcommand{\VE}[1]{\overrightarrow{#1}}
\newcolumntype{M}[1]{>{\centering\arraybackslash}m{#1}}
\renewcommand{\arraystretch}{}

\theoremstyle{break}
\theorembodyfont{\upshape}
\newtheorem{Prop}{Propri\'et\'e}
\newtheorem{Def}{D\'efinition}
\newtheorem{Rem}{Remarque}
\newtheorem{corr}{Correction}


\newtheorem{Th}{Théorème}
\theorembodyfont{\small \sffamily }
\newtheorem{Ex}{Exemple}
\newtheorem{Preu}{Preuve}
\theorembodyfont{\small	 \sffamily }
\newtheorem{Meth}{\underline{Methode}}

\pagestyle{fancy}
\fancyhead[L]{1STI2D}
\fancyhead[C]{COURS: PROBABILITES}
\fancyhead[R]{2019/2020}
\rfoot{\small -\thepage-}
\cfoot{}

\begin{document}

	
	\begin{center}
		\shadowbox{\begin{large}
				\textcolor{black}{VARIABLE ALÉATOIRE, LOI de Bernoulli}
			\end{large}}
		\end{center}
	\tableofcontents

\section{Un peu d'histoire}
Le mot hasard vient de l'arabe , le mot aléatoire vient du latin aléa  et le mot probabilité du latin et s'oppose au mot certitude .\\
 probabilitas «vraisemblance», du lat. probabilis, v. probable.\\
Les probabilités sont aujourd'hui l'une des branches les plus importantes et les plus pointues des mathématiques. Pourtant,c'est en cherchant à résoudre des problèmes posés par les jeux de hasard que les mathématiciens donnent naissance aux probabilités.\\

Les archéologues ont montré que ces jeux ont été pratiqués dans de nombreuses sociétés antiques. Pourtant, on ne trouve nul part trace de leur étude.\\

En 1654, Blaise Pascal(1623 ; 1662) entretient avec Pierre de Fermat(1601 ; 1665) des correspondances sur le thème des jeux de hasard et d'espérance de gain qui les mènent à exposer une théorie nouvelle : les calculs de probabilités.Ils s’intéressent à la résolution de problèmes de dénombrement comme par exemple celui du Chevalier de Méré:« Comment distribuer équitablement la mise à un jeu de hasard interrompu avant la fin ? »\\
Un autre traité, plus complet, sur les probabilités, est l'oeuvre d'un mathématicien suisse, Jacques Bernoulli (1662-1716).Son œuvre majeure est Ars Conjectandi publiée après sa mort à Bâle en 1713, avec une préface de son neveu Nicolas Bernoulli. Il y pose les principes du calcul des probabilités et introduit les nombres de Bernoulli.\\
Cet ouvrage aborde un aspect nouveau, le lien entre probabilités et fréquences en cas de tirages répétés (d'un jeu de pile ou face). Il énonce et démontre \\

On peut résumer ce théorème en le simplifiant : Pour un grand nombres de lancers de pièces (jeu de pile ou face) l'écart entre fréquence et probabilité tend vers zéro. C'est à dire que si l'on effectue beaucoup de lancers, la fréquence d'apparition du pile va tendre vers 1/2, c'est assez intuitif.\\
\newpage
\section{Variable aléatoire et loi de probabilité}
\subsection{Variable aléatoire} 
\subsubsection{Activité}

Voir fiche 1 cahier exercices

\subsubsection{Variable aléatoire : définition}
\noindent On lance une pièce trois fois de suite. On note P ou F suivant que \og pile \fg{} ou \og face \fg{} apparaît.\\
En considérant les issues équiprobables on a $\Omega=\left\lbrace FFF ; FFP ; FPF ; FPP ; PFF ; PFP ; PPF ; PPP\right\rbrace $.



On associe à chaque issue un gain \og algébrique \fg{} (positif ou négatif) d'argent.
\begin{itemize}
	\item un gain de 5 \euro si l'on tire exactement 3 \og pile \fg{} ;
	\item un gain de 3 \euro si l'on tire exactement 2 \og pile \fg{} ;
	\item une perte de 4 \euro sinon.
\end{itemize}
On définit alors une autre fonction $Y$ de $\Omega$ dans $\mathbb{R}$ qui à tout issue de $\Omega$ associe le gain algébrique du joueur.\\
$Y$ peut prendre 3 valeurs possibles : $y_1=-4$ ; $y_2=3$ et $y_3=5$.
\begin{center}
	\includegraphics[scale=1]{cours7.png} 
\end{center}
et la loi de probabilité de $Y$ est donnée par le tableau suivant :
\begin{center}
	\begin{tabular}{|c|c|c|c|}
		\hline
		$\text{~~~~}x_{i}\text{~~~~}$&$\text{~~~~}-4\text{~~~~}$&$\text{~~~~}3\text{~~~~}$&$\text{~~~~}5\text{~~~~}$\\
		\hline
		$P(Y=y_{i})$&&&\\
		\hline
	\end{tabular}
\end{center}
	\begin{bclogo}[couleur = yellow!30, arrondi = 0.1,logo=\bcbook]{Définition variable aléatoire}
	Soit $\Omega$ l'univers fini d'une expérience aléatoire (ensemble des issues de l'expérience aléatoire) et une loi de probabilité $P$ sur $\Omega$.
	\begin{itemize}
		\item On appelle  \textbf{variable aléatoire} toute fonction $X$ de $\Omega$  dans $\mathbb{R}$ qui , à toute issue de $\Omega$ fait correspondre un réel $x_i$
		\item L'événement \og $X$ prend la valeur $x_{i}$ \fg{} noté $(X=x_{i})$, est l'ensemble des éléments de $\Omega$ qui ont pour image $x_i$ par $X$.
		\item $X(\Omega)=\left\lbrace x_{1} ; ..... x_{n}\right\rbrace $ l'ensemble des valeurs prises par $X$.\\
	\end{itemize}
		\end{bclogo}
	\begin{Rem}
		Etant donné que la variable aléatoire $X$ prend un \textbf{nombre fini de valeurs} puisque $Car(\Omega)$ est fini, on parlera de variable aléatoire \textbf{discrète.}.\\
		Plus tard, vous étudierez des variables aléatoires sur des univers infini . On parlera de variable aléatoire continue.
	\end{Rem}
\newpage
	\begin{Ex}
		Soit l'expérience aléatoire : "On lance un dé à six faces et on regarde le résultat et on considère le jeu suivant:
		\begin{itemize}
			\item Si le résultat est pair, on gagne 2$\geneuro$
			\item Si le résultat est 1, on gagne 3$\geneuro$
			\item Si le résultat est 3 ou 5, on perd 4$\geneuro$
		\end{itemize}.
		\begin{enumerate}
			\item Quel est l'univers de cette expérience?
			\item On appelle $X$ la variable aléatoire correspondant au gain possible (il peut-être aussi négatif). Quelles sont les valeurs prises par $X$?
		\end{enumerate}
	\end{Ex}
	\subsection{Loi de probabilité d'une variable aléatoire}
		\begin{bclogo}[couleur = yellow!30, arrondi = 0.1,logo=\bcbook]{Définition loi de probabilité}
		On considère un ensemble fini $\Omega$ et une loi de probabilité $P$ sur $\Omega$. \\
		Soit $X$ une variable aléatoire définie sur $\Omega$.\\
		Soit $\left\lbrace x_{1} ; ..... x_{n}\right\rbrace $ l'ensemble des valeurs prises par $X$.\\
		Lorsqu'à chaque valeur $x_{i}$  on associe la probabilité de l'événement $(X=x_{i})$, on définit 
		la \textbf{loi de probabilité de la variable aléatoire} $\boldsymbol{X}$\\
		On représente en général cette loi à l'aide d'un tableau du type :\\\begin{center}
			\begin{tabular}{|>{\centering}p{3cm}|>{\centering}p{1cm}|>{\centering}p{1cm}|>{\centering}p{1cm}|>{\centering}p{1cm}|>{\centering}p{2cm}|}
				\hline
				\rule[-1ex]{0pt}{4ex} Valeurs $x_i$ & $x_1$ & $x_2$ & $\dots$ & $x_n$  & Total \tabularnewline\hline
				\rule[-1ex]{0pt}{4ex} $P(X=x_i)$ & $ p_1$ & $ p_2 $ & $ \dots $ & $ p_n $  & $1$\tabularnewline\hline
			\end{tabular}
		\end{center}
		\end{bclogo}
\begin{Ex}
	Reprendre l'exemple 1 et donner la loi de probabilité de $X$\\
	\renewcommand{\arraystretch}{2}
		\begin{tabular}{|p{3cm}|p{3cm}|p{3cm}|p{3cm}|}
		\hline
		$x_i$&&&\\
		\hline
		$P(X=x_i)$&&&\\
		\hline
	\end{tabular}
\end{Ex}
	
	
\subsection{Espérance}
		\begin{bclogo}[couleur = yellow!30,arrondi =0.1,logo =\bcbook]{Définition espérance} 
		Soit $X$ une variable aléatoire définie sur un univers $\Omega$ muni d'une loi de probabilité $P$.\\
		La loi de probabilité de $X$ est donnée par le tableau ci dessous:
		\begin{center}
			\renewcommand{\arraystretch}{1.5}
			\begin{tabular}{|c|c|c|c|c|}
				\hline
				$x_i$&$x_1$&$x_2$&$\cdots\cdots$&$x_n$\\
				\hline
				$P(X=x_i)$&	$p_1$&	$p_2$&	$\cdots\cdots$&$p_n$\\
				\hline
			\end{tabular}
		\end{center}
		
		\textbf{	L'espérance mathématique de $X$} est le nombre $E(X)$ définie par : 
		$$ E(X)=\sum_{i=1}^nx_i\times p(X=x_i)= \sum_{i=1}^nx_ip_i= x_1p_1+x_2p_2+\dots +x_np_n $$
		
	\end{bclogo}
	\begin{Ex}
		Calculer l'espérance de $X$ avec les données de l'exemple 1.
	\end{Ex}
	\begin{Rem}
		\begin{itemize}
			
			\item Lors d'un grand nombre d'expériences, le gain moyen d'un joueur se stabilise aux environs de $E(X)$.
			\item On interprète l'espérance comme le gain moyen que peut espérer un joueur par partie, 
			s'il joue un grand nombre de fois. On parlera donc de jeu favorable, défavorable ou équitable en 
			fonction du signe de $E(X)$.
		\end{itemize}
	\end{Rem}
		\begin{Meth}[Savoir étudier une variable aléatoire et déterminer sa loi de probabilité et son espérance]
	Soit l'expérience suivante: \\
		\begin{minipage}{0.6\textwidth}
			Une roue de loterie est partagée en dix secteurs de quatre couleurs différentes (bleu, jaune, vert et rose), comme représenté sur la figure ci-contre.\\ Quand on lance cette roue, elle tourne, puis s'arrête librement devant le repère . On suppose que tous les secteurs ont la même probabilité de s'arrêter devant le repère.\\
			Si la couleur de sortie est le bleu, on perçoit 15 \euro, si c'est le rose, on perçoit 10 \euro, si c'est le jaune, on perçoit 2 \euro et si c'est le vert, on ne perçoit rien.On appelle $X$ la variable aléatoire correspondant au gain possible. 
		\end{minipage} 
		\begin{minipage}{0.4\textwidth}
			\includegraphics[scale=0.6]{cours5.png}
		\end{minipage}
	\begin{enumerate}
		\item 	Quel est  l'univers $\Omega$ de cette expérience.
		\item Donner $X(\Omega)$ l'ensemble des valeurs possibles prises par $X$ sa loi de probabilité.
		\item Donner sous forme de tableau la loi de probabilité de $x$
		\item Calculer $E(X)$ et interpréter le résultat.
	\end{enumerate}	

	
\end{Meth}
	\section{Expérience aléatoire à deux épreuves indépendantes}
	\subsection{Activité}
Voir Activité 3 page 147 cahier exercices
	\subsection{Expériences indépendantes}
		\begin{bclogo}[couleur = yellow!30,arrondi =0.1,logo =\bcbook]{Définition expériences indépendantes} 
			Deux expériences sont dites indépendantes si le résultat de l’une n’a aucune influence sur le résultat de l’autre.
			
			\end{bclogo}
		Préciser dans chacun des cas si les 2 expériences sont indépendantes:
		\begin{Ex}
			Léa tente l’expérience suivante avec ses vêtements:Elle dépose dans un panier 4 chemisiers indiscernables au toucher: 1 blanc, 1 rouge et 2 verts.Dans un autre panier, elle y dépose 2 jupes également indiscernables au toucher: 1 blanche et 1 noire.Elle tire successivement et au hasard, un chemisier du premier panier et une jupe du deuxième panier.	
		\end{Ex}
	\begin{Ex}
		Un sac contient 10 boules jaunes et 5 boules vertes . Alice tire une première boule dans le sac  et note sa couleur puis une seconde et note aussi sa couleur.
	\end{Ex}
	\begin{Ex}
	Un sac contient 10 boules jaunes et 5 boules vertes . Pierre tire une première boule et note sa couleur puis la remet puis une seconde et note aussi sa couleur.
\end{Ex}
\subsection{Modélisation par un arbre pondéré}
\begin{Ex} 
Lire exemple 3 page 151 du livre
\end{Ex}
	\begin{bclogo}[couleur = red!30,arrondi =0.1,logo =\bcbook]{Rappel Propriétés Arbre pondéré} 
Soit une expérience aléatoire que l'on modélise par un arbre pondéré:\textbf{Un arbre pondéré se construit et se lit de gauche à droite.}
Les traits partant de la racine sont appelés branches primaires de l’arbre ; elles mènent à des nœuds.
Les branches joignant deux nœuds sont dites secondaires.
\begin{itemize}
	\item Règle 1:
	La somme des probabilités issues d’un même noeud est égale à 1.
	\item Règle 2:Principe multiplicatif:
	La probabilité d’un événement correspondant à un chemin est égale au produit des 
	probabilités  portées par les branches de ce chemin.
	\item Règle 3: 
	La probabilité d’un événement est égale à la somme des probabilités des chemins qui aboutissent 
	à sa réalisation.
\end{itemize}

\end{bclogo}
\begin{Meth}[Savoir traduire une situation par un arbre et calculer des probabilités]
	Une urne contient 2 boules gagnantes et 8 boules perdantes. Une expérience consiste à tirer au hasard 2 fois de suite une boule en la remettant à chaque fois dans l’urne.
		Soit X la variable aléatoire égale au nombre de boules gagnantes.
	\begin{enumerate}
		\item Traduire la situation par un arbre pondéré
		\item Calculer la probabilité $P(X=2)$ d’obtenir 2 boules gagnantes.
		\item Calculer la probabilité $P(X\geqslant1)$ d’obtenir au moins 1 boules gagnantes.
	\end{enumerate}
\end{Meth}
\section{Loi de Bernoulli}
\subsection{Activité}
Voir activité 4 p 147 cahier exercices
\subsection{Épreuve de Bernoulli}
\begin{bclogo}[couleur = yellow!30, arrondi = 0.1,logo=\bcbook]{Définition épreuve de Bernoulli}
	On appelle \textbf{épreuve de Bernoulli} toute expérience aléatoire qui n’a que \textbf{2 issues possibles. }
	L’une est appelée \textbf{« succès »} noté S et l’autre \textbf{« échec }» noté E ou $\bar{S }$
\end{bclogo}
\begin{Ex}
	\begin{itemize}
		\item On lance une pièce biaisée. La probabilité de faire pile est égale à 1/4 ; celle de faire face est égale à 3/4
		On décide d’appeler succès l’événement S : « obtenir pile ». Comme il n’y a que 2 issues possibles : « pile » et « face », cette expérience aléatoire est une épreuve de Bernoulli. $p (S) = \dfrac{1}{4}$ et $p( \bar{S}) =\dfrac{3}{4}$
		\item On lance un dé parfait à 6 faces et on s’intéresse à l’obtention du nombre 5. Bien qu’il y ait a priori 6 issues possibles à cette expérience, on peut considérer comme univers $\Omega = \{S ; \bar{S}\}$ , où S désigne l’événement « obtenir 5 ». On a $P( S)=\dfrac{1}{6}$  et $P(\bar{S} )=1-\dfrac{1}{6}=\dfrac{5}{6}$ .
	\end{itemize}
\end{Ex}
\subsection{Schéma de Bernouilli.}
Considérons une épreuve de Bernoulli. Notons $p$ la probabilité de l’événement « succès ». La probabilité de l’événement « échec » est donc $q=1-p$.
\begin{bclogo}[couleur = yellow!30,arrondi =0.1,logo =\bcbook]{ Définition schéma de Bernoulli}
	\textbf{Un schéma de Bernoulli est la répétition, dans les mêmes conditions, d’une même épreuve de Bernoulli ; ces épreuves étant identiques et indépendantes. }
	
\end{bclogo}

\begin{Ex}
	\begin{itemize}
		\item Lancer n fois un dé ou une pièce.
		\item Les tirages successifs, et avec remise dans une urne.
		\item Un seul lancer simultané de n pièces (ou de n dés).
		
	\end{itemize}
\end{Ex}

\begin{Rem}
	Une façon de représenter un schéma de Bernoulli est d’utiliser un arbre pondéré.\\
	Exemple d’arbre représentant un schéma de Bernoulli pour n = 2\\
	%:-+-+-+- Engendré par : http://math.et.info.free.fr/TikZ/Arbre/
	\begin{center}
		% Racine à Gauche, développement vers la droite
		\begin{tikzpicture}[xscale=1,yscale=1]
		% Styles (MODIFIABLES)
		\tikzstyle{fleche}=[->,>=latex,thick]
		\tikzstyle{noeud}=[fill=yellow,circle,draw]
		\tikzstyle{feuille}=[fill=yellow,circle,draw]
		\tikzstyle{etiquette}=[midway,fill=white,draw]
		% Dimensions (MODIFIABLES)
		\def\DistanceInterNiveaux{3}
		\def\DistanceInterFeuilles{2}
		% Dimensions calculées (NON MODIFIABLES)
		\def\NiveauA{(0)*\DistanceInterNiveaux}
		\def\NiveauB{(1)*\DistanceInterNiveaux}
		\def\NiveauC{(2)*\DistanceInterNiveaux}
		\def\InterFeuilles{(-1)*\DistanceInterFeuilles}
		% Noeuds (MODIFIABLES : Styles et Coefficients d'InterFeuilles)
		\node[noeud] (R) at ({\NiveauA},{(1.5)*\InterFeuilles}) {$\Omega$};
		\node[noeud] (Ra) at ({\NiveauB},{(0.5)*\InterFeuilles}) {$S$};
		\node[feuille] (Raa) at ({\NiveauC},{(0)*\InterFeuilles}) {$S$};
		\node[feuille] (Rab) at ({\NiveauC},{(1)*\InterFeuilles}) {$\overline{S}$};
		\node[noeud] (Rb) at ({\NiveauB},{(2.5)*\InterFeuilles}) {$\overline{S}$};
		\node[feuille] (Rba) at ({\NiveauC},{(2)*\InterFeuilles}) {$S$};
		\node[feuille] (Rbb) at ({\NiveauC},{(3)*\InterFeuilles}) {$\overline{S}$};
		% Arcs (MODIFIABLES : Styles)
		\draw[fleche] (R)--(Ra) node[etiquette] {$p$};
		\draw[fleche] (Ra)--(Raa) node[etiquette] {$p$};
		\draw[fleche] (Ra)--(Rab) node[etiquette] {$1-p$};
		\draw[fleche] (R)--(Rb) node[etiquette] {$1-p$};
		\draw[fleche] (Rb)--(Rba) node[etiquette] {$p$};
		\draw[fleche] (Rb)--(Rbb) node[etiquette] {$1-p$};
		\end{tikzpicture}
	\end{center}
	%:-+-+-+-+- Fin
		On considère un schéma de Bernoulli avec n =3. La représentation de la situation avec un arbre est:\\	
	
	%:-+-+-+- Engendré par : http://math.et.info.free.fr/TikZ/Arbre/
	\begin{center}
		% Racine à Gauche, développement vers la droite
		\begin{tikzpicture}[xscale=1,yscale=1]
		% Styles (MODIFIABLES)
		\tikzstyle{fleche}=[->,>=latex,thick]
		\tikzstyle{noeud}=[fill=yellow,circle,draw]
		\tikzstyle{feuille}=[fill=yellow,circle,draw]
		\tikzstyle{etiquette}=[midway,fill=white,draw]
		% Dimensions (MODIFIABLES)
		\def\DistanceInterNiveaux{3}
		\def\DistanceInterFeuilles{2}
		% Dimensions calculées (NON MODIFIABLES)
		\def\NiveauA{(0)*\DistanceInterNiveaux}
		\def\NiveauB{(1)*\DistanceInterNiveaux}
		\def\NiveauC{(2)*\DistanceInterNiveaux}
		\def\NiveauD{(3)*\DistanceInterNiveaux}
		\def\InterFeuilles{(-1)*\DistanceInterFeuilles}
		% Noeuds (MODIFIABLES : Styles et Coefficients d'InterFeuilles)
		\node[noeud] (R) at ({\NiveauA},{(3.5)*\InterFeuilles}) {$\Omega$};
		\node[noeud] (Ra) at ({\NiveauB},{(1.5)*\InterFeuilles}) {$S$};
		\node[noeud] (Raa) at ({\NiveauC},{(0.5)*\InterFeuilles}) {$S$};
		\node[feuille] (Raaa) at ({\NiveauD},{(0)*\InterFeuilles}) {$S$};
		\node[feuille] (Raab) at ({\NiveauD},{(1)*\InterFeuilles}) {$\overline{S}$};
		\node[noeud] (Rab) at ({\NiveauC},{(2.5)*\InterFeuilles}) {$\overline{S}$};
		\node[feuille] (Raba) at ({\NiveauD},{(2)*\InterFeuilles}) {$S$};
		\node[feuille] (Rabb) at ({\NiveauD},{(3)*\InterFeuilles}) {$\overline{S}$};
		\node[noeud] (Rb) at ({\NiveauB},{(5.5)*\InterFeuilles}) {$\overline{S}$};
		\node[noeud] (Rba) at ({\NiveauC},{(4.5)*\InterFeuilles}) {$S$};
		\node[feuille] (Rbaa) at ({\NiveauD},{(4)*\InterFeuilles}) {$S$};
		\node[feuille] (Rbab) at ({\NiveauD},{(5)*\InterFeuilles}) {$\overline{S}$};
		\node[noeud] (Rbb) at ({\NiveauC},{(6.5)*\InterFeuilles}) {$\overline{S}$};
		\node[feuille] (Rbba) at ({\NiveauD},{(6)*\InterFeuilles}) {$S$};
		\node[feuille] (Rbbb) at ({\NiveauD},{(7)*\InterFeuilles}) {$\overline{S}$};
		% Arcs (MODIFIABLES : Styles)
		\draw[fleche] (R)--(Ra) node[etiquette] {$p$};
		\draw[fleche] (Ra)--(Raa) node[etiquette] {$p$};
		\draw[fleche] (Raa)--(Raaa) node[etiquette] {$p$};
		\draw[fleche] (Raa)--(Raab) node[etiquette] {$1-p$};
		\draw[fleche] (Ra)--(Rab) node[etiquette] {$1-p$};
		\draw[fleche] (Rab)--(Raba) node[etiquette] {$p$};
		\draw[fleche] (Rab)--(Rabb) node[etiquette] {$1-p$};
		\draw[fleche] (R)--(Rb) node[etiquette] {$1-p$};
		\draw[fleche] (Rb)--(Rba) node[etiquette] {$p$};
		\draw[fleche] (Rba)--(Rbaa) node[etiquette] {$p$};
		\draw[fleche] (Rba)--(Rbab) node[etiquette] {$1-p$};
		\draw[fleche] (Rb)--(Rbb) node[etiquette] {$1-p$};
		\draw[fleche] (Rbb)--(Rbba) node[etiquette] {$p$};
		\draw[fleche] (Rbb)--(Rbbb) node[etiquette] {$1-p$};
		\end{tikzpicture}
	\end{center}
	%:-+-+-+-+- Fin	Il y a 8 chemins possibles au total.
	
	
\end{Rem}
\begin{Meth}[Reconnaitre un schéma de Bernoulli]
	Pour chacune des expériences aléatoires suivantes dites si elle constitue une épreuve de Bernoulli? Si c'est le cas proposez un succès $S$ et sa probabilité $\mathbb{P}(S)$.
	
	\begin{enumerate}
		\item Un stock contient $1\ \%$ de pièces défectueuses. On y prélève une pièce et on regarde si elle présente un défaut.
		\item Selon l'INSEE, $45\ \%$ des familles française ont un seul enfant, $38\ \%$ en ont deux et $17\ \%$ en ont trois ou plus. On interroge au hasard un élève du lycée et on lui demande le nombre d'enfants de sa famille.
	\end{enumerate}
\end{Meth}
\end{document}
