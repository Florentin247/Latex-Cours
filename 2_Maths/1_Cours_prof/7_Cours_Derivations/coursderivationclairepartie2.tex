\documentclass[11pt,a4paper]{article}
\usepackage[utf8]{inputenc}
\usepackage[francais]{babel}
\frenchbsetup{StandardLists=true} % à inclure si on utilise \usepackage[french]{babel}
\usepackage[T1]{fontenc}
\usepackage{amsmath}
\usepackage{amsthm}
\usepackage{amsfonts}
\usepackage{amssymb}
\usepackage{graphicx}
\usepackage{framed}
\usepackage{fancyhdr}
\usepackage[left=1.2cm,right=1.2cm,top=1.8cm,bottom=1.3cm]{geometry}
\usepackage{array} 
\usepackage{fancyhdr} 
\usepackage{fancybox}
\usepackage{pst-tree}
\usepackage[framed]{ntheorem}
\usepackage{tabularx}
\usepackage{pstricks,pstricks-add}
\usepackage{eurosym}
\usepackage{multicol}
\usepackage{tikz,tkz-tab}
\usepackage[tikz]{bclogo}
\usepackage{mathrsfs}
\usepackage{numprint}
\usepackage{pifont}


%%%%%%%%%%%%
%Tableaux
%%%%
\usepackage{color,colortbl} 
\usepackage[table]{xcolor}


\rfoot{\small -\thepage-}
\cfoot{}
%papier millimétré
\newcommand{\mili}[4]{\psgrid[subgriddiv=10, gridlabels=0, gridwidth=0.4pt, subgridwidth=0.4pt,gridcolor=brown!80,subgridcolor=brown!40](#1,#2)(#3,#4)}

% Raccourcis diverses:
\newcommand{\nwc}{\newcommand}
\nwc{\dsp}{\displaystyle}
\nwc{\ct}{\centerline}
\nwc{\bgar}{\begin{array}}\nwc{\enar}{\end{array}}
\nwc{\bgit}{\begin{itemize}}\nwc{\enit}{\end{itemize}}
\nwc{\bgen}{\begin{enumerate}}\nwc{\enen}{\end{enumerate}}

\nwc{\la}{\left\{}\nwc{\ra}{\right\}}
\nwc{\lp}{\left(}\nwc{\rp}{\right)}
\nwc{\lb}{\left[}\nwc{\rb}{\right]}

\def\R{{\mathbb R}}
%\def\Q{{\mathbb Q}}
\def\Z{{\mathbb Z}}
%\def\D{{\mathbb D}}
\def\N{{\mathbb N}}
%\def\C{{\mathbb C}}


\nwc{\bgsk}{\bigskip}
\nwc{\vsp}{\vspace{0.1cm}}
\nwc{\vspd}{\vspace{0.2cm}}
\nwc{\vspt}{\vspace{0.3cm}}
\nwc{\vspq}{\vspace{0.4cm}}


\def\epsi{\varepsilon}
\def\vphi{\varphi}
\def\lbd{\lambda}

\def\Cf{\mathcal{C}_f}


\pagestyle{fancy}
\newcommand{\Oij}{$\left( {{\mathrm{O}};\vec i,\vec j} \right)$}
\renewcommand{\thesection}{\Roman{section}}
\renewcommand{\thesubsection}{\arabic{subsection}}
\renewcommand{\thesubsubsection}{\alph{subsubsection}}
\newcommand{\VE}[1]{\overrightarrow{#1}}
\newcolumntype{M}[1]{>{\centering\arraybackslash}m{#1}}
\renewcommand{\arraystretch}{}

\theoremstyle{break}
\theorembodyfont{\upshape}
\newtheorem{Prop}{Propri\'et\'e}
\newtheorem{Def}{D\'efinition}
\newtheorem{Rem}{Remarque}
\newtheorem{corr}{Correction}
% Tableaux
\newcolumntype{C}[1]{>{\centering\arraybackslash}p{#1cm}}

\newtheorem{Th}{Théorème}
\theorembodyfont{\small \sffamily }
\newtheorem{Ex}{Exemple}
\newtheorem{Preu}{Preuve}
\theorembodyfont{\small	 \sffamily }
\newtheorem{Meth}{\underline{Méthode}}

\pagestyle{fancy}
\fancyhead[L]{1STI2D}
\fancyhead[C]{COURS}
\fancyhead[R]{2019/2020}
\rfoot{\small -\thepage-}
\cfoot{}

\begin{document}

	
	\begin{center}
		\shadowbox{\begin{large}
				\textcolor{black}{DÉRIVÉE : Deuxième partie}
			\end{large}}
		\end{center}
	\tableofcontents

\section{Rappel cours dérivation première partie}
	\begin{bclogo}[couleur = yellow!30, arrondi = 0.1,logo=\bcbook]{Rappel}
		\begin{itemize}
			\item \boldmath	Le {taux de variations} de $f$ en $a$  est le nombre défini par 
				$\tau_a(h)=\frac{f(a+h)-f(a)}{h}$
			\item Le nombre dérivé s'il existe est le nombre défini par  :$ f'(a)=\displaystyle\lim_{h\to0}\tau_a(h)=\displaystyle\lim_{h\to0}\frac{f(a+h)-f(a)}{h}=l $
			\item 	Si $f$ est dérivable en $a$ , la \textbf{tangente} $\mathscr{T}_a$ à la courbe représentative $\mathscr{C}_f$ de $f$ en $a$ est la droite:\begin{enumerate}
				\item qui a pour coefficient directeur $f'(a)$
				\item qui passe par le point $A$ de coordonnées $(a;f(a))$.
				\item elle a pour équation \fbox{$y=f'(a)\left( x-a\right) +f(a)$}
			\end{enumerate}
		\item Lorsque le nombre dérivé de $f$ existe pour toutes les valeurs $x$ de I, on dit que $f$ est dérivable sur I.
		La fonction qui à $x$ associe son nombre dérivé $f'(x)$ s’appelle la fonction dérivée de $f$ (ou la dérivée
		de $f$ ) et elle se note $f’$.
		\item Premières formules de dérivation (voir paragraphes II) et III) dérivée de $x\mapsto x^2$, $x\mapsto x^3$ de $u+v$ et de $\lambda u$ )
\item L'étude du signe de la dérivée de la fonction $f$ permet d'étudier les variations de $f$\\
Soit $f$ une fonction dérivable sur une intervalle  $I$, alors :
\begin{itemize}
	\item Si, pour tout $x$ de $I$, $f'(x)  \geq 0 $  alors $f$ est \textbf{ croissante} sur $I$ .
	\item Si, pour tout $x$ de $I$, $f'(x) \leq 0 $  alors $f$ est \textbf{ décroissante} sur $I$ .
	\item Si, pour tout $x$ de $I$, $f'(x) = 0 $ , alors $f$ est \textbf{ constante } sur $I$ .
\end{itemize} 
		\end{itemize}
	\end{bclogo}
\newpage

%%%%%%%%%%%%%%%%%%%%%%%%%%%%%%%%ù
%	Méthode 1
%%%%%%%%%%%%%%%%%%%

\begin{Meth}[Rappel:Étude d'une fonction polynôme de degré 3]
Soit la fonction $f$ définie sur $\R$ par $f(x)=x^3-3x-4$
\begin{enumerate}
	\item Déterminer la dériver de $f$ \par
		$f'(x) = 3x^2-3$ ~~~~$\forall x \in \R $ 
	\item Étudier le signe de $f'(x)$ \par
	Signe de $f'(x)$ : $a=3>0$
	\item En déduire les variations de$f$\par
	On factorise $f'(x) = 3(x^2-1) = 3(x-1)(x+1)$
	$f'$ est une fonction du 2$^{nd}$ degré qui a 2 racines, $1$ et $-1$ donc $f'(x)$ est du signe de $a=3>0$ à l'exterieur des racines.\par
		\begin{center}
				\underline{Tableau de variations}

			\setlength{\columnseprule}{0.4pt}
            \begin{tikzpicture}
            \tkzTabInit[espcl=2.5]{$x$/1,$f'(x)$/1,$f(x)$/2.5}{$-\infty$,$-1$,$1$,$+\infty$}
			\tkzTabLine{,+,z, -, z,+, }
            \tkzTabVar{-/,+/$-2$,-/$-6$,+/}
            \end{tikzpicture}
    \end{center}
	$f(-1) = (-1)^3 - 3 \times (-1) - 4 = -1 + 3 - 4 = -2$
	$f(1) = (1)^3 - 3 \times (1) - 4 = 1 - 3 - 4 = -6$
	
	\item Donner l'équation de la tangente $T$ à la courbe de $f$ au point d'abscisse -2 \par
	Equation tengeante à $C_f$ au point d'abscice \par
	$y = f'(a)(x-a)+f(a)$ \par
	\textbf{\underline{Ici}} \par
	$T_{-2}: y = f'(-2)(x-(-2))+f(-2)$\par
	$f'(-2) = 3 \times (-2)^2 - 3 = 3 \times 4 - 3 = 9 $\par
	$f(-2) =  (-2)^3 - 3 \times (-2) -4 = -8 + 6 - 4 = -6 $\par
	\textbf{\underline{Donc}}\par
	 $T_{-2}: y = 9(x+2)-6$
	 $T_{-2}: y = 9x+12$
\end{enumerate}
\end{Meth}
\newpage
\section{Fonction dérivée de fonctions de référence} 


	\begin{bclogo}[couleur = yellow!30, arrondi = 0.1,logo=\bcbook]{Fonctions de référence}
	\begin{minipage}{0.9\textwidth}
	$f$ désigne une fonction dérivable sur $I$ et $f'$ est la fonction dérivée de f. On a : 
	\begin{center}
	\renewcommand{\arraystretch}{1.5}
	\begin{tabular}{|C{2.5}|C{2.5}|C{2.5}|}
	\hline 
	Fonction $f$ & Fonction $f'$ & Intervalle I \\ 
	\hline  
	$k$ & $0$ & $\R$ \\ 
	\hline  
	$ax+b$ & $a$ & $\R$ \\ 
	\hline
	$x^2$ & $2x$ & $\R$ \\ 
	\hline
	$x^3$ & $3x^2$ & $\R$ \\ 
	\hline
	$x^n$ & $n x^{n-1}$ & $\R$, avec $n\in\N^*$ \\
	\hline    
	\rule{0cm}{0.65cm}
	$\dfrac{1}{x}$ & $ -\dfrac{1}{x^2}$ & $\R^*$ \\ [0.2cm]
	\hline
	$\cos(x)$ & $ -\sin(x)$ & $\R$ \\
	\hline
	$\sin(x)$ & $\cos(x)$ & $\R$ \\
	\hline
	
	\end{tabular}
	\end{center}
	
	\end{minipage}
	
\end{bclogo}


\begin{Rem}
Les quatre premières lignes sont des rappels de la partie 1 sur la dérivation. La 5ème est une généralisation des résultats obtenus avec $x^2$ et $x^3$ .\\
\begin{Ex}
	Si $f$ est définie par $f(x)=x^5$ , $f$ est dérivable sur $\R$ et $f'(x)=5x^4$.
\end{Ex}
Les quatre dernières lignes sont admises.	
\end{Rem} 
%%%%%%%%%%%%%%%%%%%%%%%%%%%%%%%%%%%%%%%%%%%%%%%%%%%%%%%%%%%%%%%%%%%%%%%%%%%%%%%%%%%%%%%%%%%%%%%%%%%%%%%%%%%%%%%%%%%%%%
\section{Fonction dérivée et opérations}
	\subsection{RAPPEL: Somme de deux fonctions, multiplication par un réel}
\begin{bclogo}[couleur = yellow!30, arrondi = 0.1,logo=\bcbook]{Fonction $u+v$ et $\lambda u$ avec $\lambda$ réeel}
	Si $u$ et $v$ sont deux fonctions dérivables sur un même intervalle $I$ de $\R$ et $\lambda$ un réel non nul alors :
	\begin{itemize}
		\item $(u+v)$ est dérivable sur $I$ et $(u+v)'=u'+v'$.
		\item $\lambda u$ est dérivable sur  $I$ et $(\lambda u)'= \lambda u'$
	\end{itemize} 
\end{bclogo}


%%%%%%%%%%%%%%%%%%%%%%%%%%%%%%%%ù
%	Méthode 2
%%%%%%%%%%%%%%%%%%%


\begin{Meth}
	\begin{enumerate}
		\item Déterminer la dérivée de la fonction $f$ définie sur $]0;+\infty[$ par $f(x)=-3x^4-\dfrac5x$\par
		$f'(x) = -3 \times (4x^3) -5 \times	(-\dfrac{1}{x^2}) = -12x^3 + \dfrac{5}{x^2}$
		\item Déterminer la dérivée de la fonction $g$ définie sur $\R$ par $g(x)=3\cos(x)-5\sin(x)$\par
		$g'(x) = -3 (\sin(x)) -5 (\cos(x)) = 3\sin(x) - 5\cos(x)$
	\end{enumerate}
	
\end{Meth} 
\newpage
\subsection{Produit de deux fonctions dérivables}

	\begin{bclogo}[couleur = yellow!30, arrondi = 0.1,logo=\bcbook]{Fonction $u\times v$}

	Si $u$ et $v$ sont deux fonctions dérivables sur un même intervalle $I$ de $\R$, alors $u\times v$ est dérivable sur $I$ et $(u\times v)'=u'v+uv'$.
\end{bclogo}
\begin{Ex}[Rédaction]
		Soit la fonction $f$définie sur $\R$ par :\\ $f(x)=(4x^3-3)(-\dfrac{3}{2}x^2-4x)$\\
	$f$ est de la forme $uv$ avec: \\
	\begin{tabular}{m{5cm}|m{0.2cm}m{8cm}}
		\ding{172} $u(x)=4x^3-3$&&\ding{172} $v(x)=-\dfrac{3}{2}x^2-4x$ \\
	\ding{173} $u$ est dérivable sur $\R$ &&\ding {173} $v$ est dérivable sur $\R$\\
		\ding{174} $u'(x)=12x^2$&&\ding{174} $v'(x)=-\dfrac{3}{2}\times2x-4=-3x-4$\\	
	\end{tabular}\\
	$f$ est donc dérivable sur $\R$ et pour tout $x\in \R$ \newline  $f'(x)=(12x^2)(-\dfrac{3}{2}x^2-4) +(-3x-4)(4x^3-3)$\\
	$\Leftrightarrow f'(x)=-16x^4-48x^3-12x^4+9x-16x^3+12$\\$\Leftrightarrow f'(x)=-28x^4-64x^3+9x+12$\\
\end{Ex}

%%%%%%%%%%%%%%%%%%%%%%%%%%%%%%%%ù
%	Méthode 3
%%%%%%%%%%%%%%%%%%%
\begin{Meth}
	Déterminer la fonction dérivée de la fonction $f$ définie sur $\R$ par $f(x)=x \cos(x)$.


\begin{multicols}{2}
	$f$ est définie sur $\R$~~~~$f$ est de la forme $uv$ avec 
\centering
\setlength{\arrayrulewidth}{2pt}
\definecolor{gris1}{gray}{0.85} 
\definecolor{gris2}{gray}{0.65}
\renewcommand{\arraystretch}{2}
    \begin{tabular}{|C{2.5}|C{2.5}|}
    \hline
     \cellcolor{gris1} $u(x) = x $ &\cellcolor{pink!70} $v(x) = \cos$ \\ 
	\hline
	$u$ dérivable sur $\R$ & $v$ dérivable sur $\R$\\
    \hline
     \cellcolor{pink!200}$u'(x)=1$&\cellcolor{gris2}$v'(x)=-\sin(x)$  \\
    \hline
    \end{tabular}

	Donc $f$ est dérivable sur $\R$ avec \par
	$f'(x)$ = \colorbox{pink!200}{$1$} \textcircled{$\times$} \colorbox{pink!70}{$\cos(x)$} $\textcircled{+}$ \colorbox{gris2}{$(-sin(x)$} \textcircled{$\times$} \colorbox{gris1}{$x)$}\par
	$f'(x)$ = \colorbox{pink}{$1\times\cos(x)$} $\textcircled{+}$ \colorbox{gray}{$(-sin(x)\times x)$}\par
	On développe, \par
	$f'(x) = \cos - x \sin(x)$

\end{multicols}
\end{Meth}

\subsection{Inverse d'une fonction dérivable}

	\begin{bclogo}[couleur = yellow!30, arrondi = 0.1,logo=\bcbook]{fonction $\dfrac{1}{v}$}
	Si $v$ est une fonction dérivable sur un intervalle $I$ de $\R$ et qui ne s'annule pas, alors $\dfrac{1}{v}$ est dérivable sur $I$ et $\left(\dfrac{1}{v}\right)' =-\dfrac{v'}{v^2}$.
	\end{bclogo}
\begin{Ex}[Rédaction]
	Soit la fonction $f$ définie sur $\R$ (car pour tout $x\in \R$, $3x^2+2x+1\neq0 )$ par:\\
$ f(x) =\dfrac{1}{3x^2+2x+1}$\\
$f$ est de la forme $\dfrac{1}{u}$
avec :\\
$u(x) = 3x^2 + 2x + 1$\\
u est dérivable sur $\R$ et $u(x)\neq 0$ pour tout $x\in\R$\\
$u'(x) = 6x + 2$\\
$f$ est donc dérivable sur $\R$ et pour tout $x\in \R$ :\\
$f'(x) = -\dfrac{6x + 2}{(3x^2 + 2x + 1)^2}$\\
\end{Ex}
\newpage
%%%%%%%%%%%%%%%%%%%%%%%%%%%%%%%%ù
%	Méthode 4
%%%%%%%%%%%%%%%%%%%

\begin{Meth}
	Déterminer la fonction dérivée de la fonction $f$ définie sur $]5;+\infty[$ par $f(x)=\dfrac{1}{2x-10}$.
	
	Soit la fonction $f$ définie sur $[5 + \infty[$ par $f(x) = \dfrac{1}{2x-10}$\par
	$f$ est de la forme \textcolor{purple}{$\dfrac{1}{u}$} avec\par
	$u(x) = 2x -10$\par
	$u$ est dérivable sur $]5;+\infty[$ et non nul\par
	$u'(x) = 2$\par
	Donc $f$ est dérivable sur $]5;+\infty[$ et $\forall~x~\in ]5;+\infty[$ \par
	
	\framebox{$f'(x)$ = $\textcircled{-}$ $\dfrac{2} {(2x-10)^2}$}


\end{Meth} 




\subsection{Quotient de deux fonctions dérivables}

	\begin{bclogo}[couleur = yellow!30, arrondi = 0.1,logo=\bcbook]{Fonction  $\dfrac{u}{v}$}

	Si $u$ et $v$ sont deux fonctions dérivables sur un même intervalle $I$ de $\R$ et $v$ ne s'annule pas sur $I$, alors $\dfrac{u}{v}$ est dérivable sur $I$ et $\left(\dfrac{u}{v}\right)' =\dfrac{u'v-uv'}{v^2}$.
\end{bclogo}
\begin{Ex}[Rédaction]
		Soit la fonction $f$ définie sur $\R\setminus\{-\dfrac{2}{3}\}$  par:\\
	$ f(x) =\dfrac{2x-1}{3x+2}$\\
	$f$ est de la forme $\dfrac{u}{v}$
	avec :\\
	\begin{tabular}{l|ll}
		\ding{172} $u(x) = 2x-1$&    &\ding{172}			$v(x)=3x+2$\\
		\ding{173} $u$ est dérivable sur $\R$ &&\ding {173} $v$ est dérivable sur $\R$ et $v(x)\neq0$ pour $x\in \R \backslash \{-\dfrac{2}{3}\}$\\
		\ding{174} $u'(x) = 2$&    &\ding{174} 			$v'(x)=3$\\
	\end{tabular}\\
	donc $f$ est dérivable sur $\R\setminus\{-\dfrac{2}{3}\}$  et pour tout $x\in\R\setminus\{-\dfrac{2}{3}\}$:\\
	$f'(x) = \dfrac{2(3x + 2)-3(2x-1)}{(3x+2)^2}$\\
	$f'(x) = \dfrac{6x+4-6x+3}{(3x+2)^2}$\\
	$f'(x) = \dfrac{7}{(3x+2)^2}$\\
	\begin{Rem}
		On ne développe pas le dénominateur car par la suite on va étudier le signe de $f'(x)$ et comme $(3x+2)^2$ est positif il suffira d'étudier le signe du numérateur.
	\end{Rem}
\end{Ex}

%%%%%%%%%%%%%%%%%%%%%%%%%%%%%%%%ù
%	Méthode 5
%%%%%%%%%%%%%%%%%%%

\begin{Meth}
	Déterminer la fonction dérivée de la fonction $f$ définie sur $\R$ par $f(x)=\dfrac{\cos x}{x^2+1}$.
\end{Meth}


%%%%%%%%%%%%%%%%%%%%%%%%%%%%%%%%ù
%	Méthode 6
%%%%%%%%%%%%%%%%%%%

\begin{Meth}[Étude de variations d'une fonction]
	Déterminer les fonctions dérivées des fonctions $f$ définies sur $I$ suivantes puis établir leur tableau de variations :
	\begin{multicols}{2}
		\begin{enumerate}
			\item $f(x)=3x^2+2x-4$, $I=\R$ \\ 
			\item $f(x)=\dfrac3x$, $I=]0;+\infty[$ \\ 
			\item $f(x)=x^2(-2x+3)$, $I=\R$ \\ 
			\item $f(x)=\dfrac{3x+4}{1-2x}$, $I=]\frac12;+\infty[$ \\
			
		\end{enumerate}
	\end{multicols}
\end{Meth}


\newpage
%%%%%%%%%%%%%%%%%%%%%%%%%%%%%%%%%%%%%%%%%%%%%%%%%%%%%%%%%%%%%%%%%%%%%%%%%%%%%%%%%%%%%%%%%%%%%%%%%%%%%%%%%%%%%%%%%%%%%%
%4 Fonctions dérivées
%%%%%%%%%%%%%%%%%%%%%%%%%
\section{Fonction dérivée d'une fonction composée} 

\subsection{Dérivée des fonctions du type $x \mapsto f(ax+b)$}
	\begin{bclogo}[couleur = yellow!30, arrondi = 0.1,logo=\bcbook]{Propriété}

On définit sur un intervalle $J$ une fonction $g$ composée de la fonction affine $x \mapsto ax+b$ par une fonction $f$. On a le schéma de composition suivant $g$: $\begin{array}{l|rclcl}
& I & \longrightarrow & J  & \longrightarrow & \R\\
& x & \longmapsto & ax+b &  \longmapsto & f(ax+b) \end{array}$. 

	Soient $a$ et $b$ deux réels et $f$ une fonction dérivable sur $I$. Soit $J$ un intervalle tel que pour tout $x$ appartenant à $I$, $ax+b$ appartient à $J$. \\
	Alors la fonction $g$ : $x \mapsto f(ax+b) $ est dérivable sur $J$ et pour tout $x$ dans $J$ :  
	\begin{center}
	$g'(x)=a \times f'(ax+b)$
	\end{center}
\end{bclogo}


%%%%%%%%%%%%%%%%%%%%%%%%%%%%%%%%ù
%	Méthode 7
%%%%%%%%%%%%%%%%%%%

\begin{Meth}
Déterminer la fonction dérivée de la fonction $g$ définie sur $\R$ par $g(x)=(3x-2)^4$.
\end{Meth} 


%%%%%%%%%%%%%%%%%%%%%%%%%%%%%%%%%%%%%%%%%%%%%%%%%%%%%%%%%%%%%%%%%%%%%%%%%%%%%%%%%%%%%%%%%%%%%%%%%%%%%%%%%%%%%%%%%%%%%% 

\subsection{Dérivée des fonctions trigonométriques composées}

	\begin{bclogo}[couleur = yellow!30, arrondi = 0.1,logo=\bcbook]{Propriété}
	Soient $A$, $\omega$ et $\varphi$ des réels. \\
	Les fonctions $f : t \longmapsto A\cos(\omega t+\varphi)$ et $g:t \longmapsto  A\sin(\omega t+\varphi) $ sont dérivables sur $\R$ et pour tout $x$ dans $\R$ on a:\begin{itemize}
		\item $$f'(t)=-A \times \omega \sin(\omega t+\varphi)$$
		\item $$g'(t)=A \times \omega \cos(\omega t+\varphi)$$ 
	\end{itemize}
	 
	\end{bclogo}


%%%%%%%%%%%%%%%%%%%%%%%%%%%%%%%%ù
%	Méthode 8
%%%%%%%%%%%%%%%%%%%

\begin{Meth}
Déterminer la fonction dérivée de la fonction $f$ définie sur $\R$ par $f(t)=10\cos(25t+\dfrac{\pi}{4})$.
\end{Meth} 



		

	

`\end{document}
