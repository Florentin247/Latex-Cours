\documentclass[12pt,a4paper]{article}
\usepackage[utf8]{inputenc}
\usepackage[french]{babel}
\frenchbsetup{StandardLists=true}
\usepackage[T1]{fontenc}
\usepackage{amsmath}
\usepackage{amsthm}
\usepackage{amsfonts}
\usepackage{amssymb}
\usepackage{graphicx}
\usepackage{framed}
\usepackage{fancyhdr}
\usepackage[left=1.3cm,right=1.3cm,top=1.8cm,bottom=1.2cm]{geometry}
\usepackage{array} 
\usepackage{fancyhdr} 
\usepackage{fancybox}
\usepackage{pst-tree}
\usepackage[framed]{ntheorem}
\usepackage{tabularx}
\usepackage{pstricks-add}
\usepackage{eurosym}
%\usepackage{pst-tree}
\usepackage[np]{numprint}
\usepackage{pifont}
\usepackage{mathrsfs}
\usepackage{amssymb}
\usepackage{amsthm}
\usepackage{pgf,tikz}
\usepackage[tikz]{bclogo}
\usepackage{pgfkeys}
\usepackage{mathrsfs}
\usepackage{multicol}
\usetikzlibrary{arrows}
\usepackage{listingsutf8}
\lstset{%
	language=python,%
	basicstyle=\ttfamily,%
}

\rfoot{\small -\thepage-}
\cfoot{}

\def\R{{\mathbb R}}
\def\Q{{\mathbb Q}}
\def\Z{{\mathbb Z}}
\def\D{{\mathbb D}}
\def\N{{\mathbb N}}
\def\C{{\mathbb C}}

\pagestyle{fancy}

\renewcommand{\thesection}{\Roman{section}}
\renewcommand{\thesubsection}{\arabic{subsection}}
\renewcommand{\thesubsubsection}{\alph{subsubsection}}
\renewcommand{\labelitemi}{$\bullet$}
\newcommand{\VE}[1]{\overrightarrow{#1}}
%\renewcommand{\r}{($O$ ; $\vec{i}$ , $\vec{j}$)}
\renewcommand{\arraystretch}{0.7}
\tikzstyle{mybox} = [draw=black, very thick, rectangle, rounded corners, inner sep=20pt, inner ysep=20pt] 
\tikzstyle{fancytitle} =[draw=black, very thick, rectangle, rounded corners, fill=white, text=black] % fill obligé sinon ne recouvre pas boite du dessous
\usepackage{array,multirow,makecell}
\setcellgapes{1pt}
\newcommand{\Syst}[2]{\left\{\begin{array}{ccc} #1\\ #2 \end{array}\right.}

\makegapedcells
\newcolumntype{R}[1]{>{\raggedleft\arraybackslash }b{#1}}
\newcolumntype{L}[1]{>{\raggedright\arraybackslash }b{#1}}
\newcolumntype{C}[1]{>{\centering\arraybackslash }b{#1}}
\theoremstyle{break}
\theorembodyfont{\upshape}
\newtheorem{Prop}{Propri\'et\'e}
\newtheorem{Def}{D\'efinition}
\newtheorem{Rem}{Remarque}
\newtheorem{exo}{Exercice}
\newtheorem{Meth}{Methode}
\newtheorem{cpreuve}{Preuve}
\newtheorem{Th}{Théorème}
\newtheorem{Act}{Activité}
\theorembodyfont{\small \sffamily }
\newtheorem{Ex}{Exemple}
\newtheorem{Preu}{Preuve}
\everymath{\displaystyle}
\pagestyle{fancy}
\fancyhead[L]{1STI2D}
\fancyhead[C]{ }
\fancyhead[R]{2019/2020}
\rfoot{\small -\thepage-}
\cfoot{}

\begin{document}

\begin{center}
   \shadowbox{\begin{large}
      \textcolor{black}{DM de Mathématiques}
   \end{large}}
\end{center}
\textbf{Exercice 1}\\
La courbe $\mathscr{C}$ est la représentation graphique d’une fonction $f$ définie et dérivable sur $\R$, dans un repère orthogonal.
\begin{multicols}{2}

\begin{enumerate}
	\item Déterminer graphiquement :
	\begin{enumerate}
		\item $f(0)$ et $f'(0)$;
		\item $f(-1)$ et $f'(-1)$;
		\item $f(2)$ et $f'(2)$;
		\item L’équation de la tangente au point d’abscisse $-1$ ;
		\item L’équation de la tangente au point d’abscisse $0$
			\end{enumerate}
		\item La droite $T$ tangente à la courbe $\mathscr{C}$ au point d’abscisse $-2$ et d’ordonnée $-1$ passe par le point A de coordonnées  $(1; 26)$
		\begin{enumerate}
			\item Déterminer par le calcul une équation de $T$.
			\item En déduire $f'(-2)$
		\end{enumerate}
\end{enumerate}
\vfill\columnbreak
	\psset{xunit=1cm,yunit=0.3cm,algebraic=true}
\def\xmin {-3}
\def\xmax {4}
\def\ymin {-7}
\def\ymax {15}

	\begin{pspicture*}(\xmin,\ymin)(\xmax,\ymax)
	\psgrid[subgriddiv=2,gridlabels=3pt,gridwidth=0.5pt,griddots=10,subgriddots=10](\xmin,\ymin)(\xmax,\ymax)
	\psaxes[Dy=5]{->}(0,0)(\xmin,\ymin)(\xmax,\ymax)
		\psplot[linewidth=1pt]{-3}{3}{x^3-3*x+1}
		\psline[arrowscale=2,linestyle=dashed]{<->}(-1,4)(1,-2)
		\psline[arrowscale=2,linestyle=dashed]{<->}(-2.5,3)(0.5,3)
			\psline[arrowscale=2,linestyle=dashed]{<->}(1,-6)(3,12)
			\psdot[dotstyle=*](-1,3)
				\psdot[dotstyle=*](0,1)
					\psdot[dotstyle=*](2,3)
	\end{pspicture*}

\end{multicols}
\textbf{Exercice 2}\\
	Un propriétaire propose à la location deux appartements notés T1 et T2. Le loyer mensuel net pour chacun des appartements se compose de trois parties :
	\begin{itemize}
		\item le loyer mensuel hors charges (loyer HC) ;
		\item les charges ;
		\item la taxe locative sur le ramassage des ordures ménagères.
	\end{itemize}
	
	Le tableau ci-après contient les informations en euros relatives à la location de ces deux appartements pour le mois de janvier.\\
	\begin{center}
	\renewcommand{\arraystretch}{2}
	\begin{tabular}{|c|c|c|c|c|}
		\hline
		&loyer HC& charges& taxe locative &loyer mensuel net\\
		\hline
		T1& 360 &&&461\\ 
		\hline
		T2 &&&54,60&\\
		\hline
		total&&&& 1043\\
		\hline
	\end{tabular}\\
	\end{center}
	La taxe locative représente 10 \% du loyer HC de l’appartement T1 et 12 \% de l’appartement T2.
	\begin{enumerate}
		\item 
		\begin{enumerate}
			\item Calculer le montant du loyer HC de l’appartement T2.(On écrira les détails des calculs)
			\item Compléter le tableau en indiquant les opérations effectuées sur votre copie.
			(c) Pour l’appartement T1, calculer la proportion des charges par rapport au loyer mensuel
			net, exprimée en pourcentage arrondi à 0,1 \% près.
		\end{enumerate}
		\item Si un locataire de l’appartement T2 reçoit une aide de 260 \euro par mois, quelle est, en pourcentage arrondi à 0,01 \% près, la part de cette aide par rapport au loyer HC ?
		
	\end{enumerate}
\newpage
\textbf{Exercice 3}\\
		\textsf{\small{\textsc{\textbf{partie a}}}}\\
		On considère la suite $\left(u_{n}\right)$ définie par $u_{0} = 250$ et pour tout entier naturel $n$, $ u_{n+1} = 0,72u_{n} +420$.
	
	\begin{enumerate}
		\item Calculer $u_2$.
		
		\item Soit $\left(v_{n}\right)$  la suite définie pour tout entier naturel $n$ par $v_{n} = u_{n} - 1500$. 
		\begin{enumerate}
			\item Démontrer que la suite $\left(v_{n}\right)$ est une suite géométrique dont on précisera le premier terme et la raison. 
			\item Exprimer $v_{n}$ en fonction de $n$.
			\item En déduire que, pour tout nombre entier naturel $n$, $u_{n} = 1500 -1250\times  0,72^n$.
		\end{enumerate} 
	
	\end{enumerate}
		\textsf {\textbf{\textsc{partie b}}}\\
	Une municipalité a décidé de proposer un abonnement mensuel à un service de location de vélos.
	
	Au mois de janvier 2018, 250 personnes se sont abonnées à ce service.
	
	Une étude statistique a permis de modéliser l'évolution du nombre d'abonnements pour les prochains mois à l'aide de la suite $\left(u_{n}\right)$ définie dans la partie A.
	
	\begin{enumerate}
		\item On considère l'algorithme suivant :
		\begin{center}
			\begin{tabular}{|l|}
				\hline
					\begin{minipage}{.3\linewidth}
					
					$U \gets \np{250}$
					
					$N \leftarrow  0$
					
					Tant que  $U \leqslant \np{1435}$
					
					\begin{itemize}
						\item[] $U \leftarrow 0,72 \times U + 420$
						\item[] $N \leftarrow N+1$
					\end{itemize}
					Fin Tant que
				\end{minipage}\\
			\hline	
			\end{tabular}
		
				
		\end{center}
		\begin{enumerate}
			\item Donner une interprétation de la valeur $N=9$ obtenue à la fin de l'exécution de cet algorithme.
		\item Écrire cet algorithme en Python.
		\end{enumerate}
	
		\item Selon ce modèle, donner une estimation du nombre d'abonnés au bout de 12 mois.
		\item Est-il possible d'envisager nombre d'abonnés supérieur à \np{2000} ?
	\end{enumerate}

\end{document}
  
