\documentclass[10pt,a4paper]{article}
\usepackage[utf8]{inputenc}
\usepackage[french]{babel}
\frenchbsetup{StandardLists=true}
\usepackage[T1]{fontenc}
\usepackage{amsmath}
\usepackage{amsthm}
\usepackage{amsfonts}
\usepackage{amssymb}
\usepackage{graphicx}
\usepackage{framed}
\usepackage{fancyhdr}
\usepackage[left=1.3cm,right=1.3cm,top=1.8cm,bottom=1.2cm]{geometry}
\usepackage{array} 
\usepackage{tabularx}
\usepackage{fancybox}
\usepackage{pst-tree}
\usepackage[framed]{ntheorem}
\usepackage{tabularx}
\usepackage{pstricks-add}
\usepackage{graphicx}
\usepackage{eurosym}
%\usepackage{pst-tree}
\usepackage[np]{numprint}
\usepackage{pifont}
\usepackage{mathrsfs}
\usepackage{amssymb}
\usepackage{amsthm}
\usepackage{pgf,tikz}
\usepackage[tikz]{bclogo}
\usepackage{pgfkeys}
\usepackage{mathrsfs}
\usepackage{multicol}
\usetikzlibrary{arrows}


\rfoot{\small -\thepage-}
\cfoot{}

\def\R{{\mathbb R}}
\def\Q{{\mathbb Q}}
\def\Z{{\mathbb Z}}
\def\D{{\mathbb D}}
\def\N{{\mathbb N}}
\def\C{{\mathbb C}}

\pagestyle{fancy}

\renewcommand{\thesection}{\Roman{section}}
\renewcommand{\thesubsection}{\arabic{subsection}}
\renewcommand{\thesubsubsection}{\alph{subsubsection}}
\renewcommand{\labelitemi}{$\bullet$}
\newcommand{\VE}[1]{\overrightarrow{#1}}
%\renewcommand{\r}{($O$ ; $\vec{i}$ , $\vec{j}$)}
\renewcommand{\arraystretch}{1}
\tikzstyle{mybox} = [draw=black, very thick, rectangle, rounded corners, inner sep=20pt, inner ysep=20pt] 
\tikzstyle{fancytitle} =[draw=black, very thick, rectangle, rounded corners, fill=white, text=black] % fill obligé sinon ne recouvre pas boite du dessous
\usepackage{array,multirow,makecell}
\setcellgapes{1pt}
\makegapedcells
\newcolumntype{R}[1]{>{\raggedleft\arraybackslash }b{#1}}
\newcolumntype{L}[1]{>{\raggedright\arraybackslash }b{#1}}
\newcolumntype{C}[1]{>{\centering\arraybackslash }b{#1}}
\theoremstyle{break}
\theorembodyfont{\upshape}
\newtheorem{Prop}{Propri\'et\'e}
\newtheorem{Def}{D\'efinition}
\newtheorem{Rem}{Remarque}
\newtheorem{exo}{Exercice}
\newtheorem{Meth}{Methode}
\newtheorem{cpreuve}{Preuve}
\newtheorem{Th}{Théorème}
\newtheorem{Act}{Activité}
\theorembodyfont{\small \sffamily }
\newtheorem{Ex}{Exemple}
\newtheorem{Preu}{Preuve}
\everymath{\displaystyle}
\pagestyle{fancy}
\lhead[]{\small }
\chead[]{\textsc{\shadowbox{\begin{large}
				\textcolor{black}{QUESTIONS du JOUR 1}
\end{large}}}}
\rhead[]{\small {}}
\fancyhead[R]{2019/2020}
\lfoot{}
\rfoot{\small -\thepage-}

\begin{document}
A faire sans calculatrice , compléter sur le sujet et coller dans le cahier d'exercices . \\
{\renewcommand{\arraystretch}{3}
\begin{tabular}{|p{1cm}|p{10cm}|p{4.5cm}| }
	\hline
&Enoncé&Réponse\\
\hline
1&On a placé sur le cercle trigonométrique suivant des points.

	\begin{center}
		\psset{unit=1.8cm}
		\begin{pspicture}(-1.2,-1.2)(1.2,1.2)
		\newrgbcolor{bleu}{0.1 0.05 .5}
		\newrgbcolor{prune}{.6 0 .48}
		\newrgbcolor{rose}{.95 .8 .9}
		\def\pshlabel#1{\footnotesize #1}
		\def\psvlabel#1{\footnotesize #1}
		\psaxes[linewidth=.75pt,labels=none,ticks=none]{->}(0,0)(-1.2,-1.2)(1.2,1.2)
		\psaxes[linewidth=1.5pt,linecolor=red]{->}(0,0)(1,1)
		\uput[dl](0,0){\footnotesize{O}}\uput[dr](1,0){\footnotesize{\prune $I$}}\uput[ul](0,1){\footnotesize{\prune $J$}}
		\pscircle[linewidth=1.25pt, linecolor=bleu,linestyle=solid](0,0){1} 
		\psset{linecolor=prune,linewidth=.5pt,linestyle=dashed,labelsep=4pt}
		\rput{0}(0,0){\multido{\n=45+90,\i=1+1}{4}{\cnode*(1;\n){2pt}{A\i}}}
		\rput{0}(0,0){\multido{\n=30+30,\i=1+1}{12}{\cnode*(1;\n){2pt}{B\i}}}
		\ncline{A1}{A2}  \ncline{A2}{A3}  \ncline{A3}{A4}  \ncline{A4}{A1}
		\ncline{B1}{B5}  \ncline{B5}{B7}  \ncline{B7}{B11}  \ncline{B11}{B1}
		\ncline{B2}{B4}  \ncline{B4}{B8} \ncline{B8}{B10} \ncline{B10}{B2} 
		\nput{10}{A1}{\footnotesize{$A$}}\nput{170}{A2}{\footnotesize{$B$}}\nput{-170}{A3}{\footnotesize{$C$}}\nput{-10}{A4}{\footnotesize{$D$}}
		\nput{0}{B1}{\footnotesize{$E$}}\nput{180}{B5}{\footnotesize{$F$}}\nput{-180}{B7}{\footnotesize{$G$}}\nput{0}{B11}{\footnotesize{$H$}}
		\nput{25}{B2}{\footnotesize{$K$}}\nput{145}{B4}{\footnotesize{$L$}}\nput{-105}{B8}{\footnotesize{$M$}}\nput{-105}{B10}{\footnotesize{$N$}}
		\uput[ur](0,1){\footnotesize{$J$}}\uput[ul](-1,0){\footnotesize{$I'$}}\uput[dl](0,-1){\footnotesize{$J'$}}
		\end{pspicture}\\
		\begin{itemize}
			\item Quel est le point image associé à $x=\dfrac{13\pi}3$?
			\item Quel est le point image associé à  $y=-\dfrac{17\pi}4$ ?
		\end{itemize}
		
		
	\end{center}&$x =\dfrac{\pi}{3}$ $\Rightarrow$ $K$\par $y = \dfrac{-\pi}{4}$ $\Rightarrow$ $D $

\\
\hline
\rotatebox{-45}{Correction}&&\\
\hline
2&Déterminer la mesure principale de l'angle orienté dont une mesure en radian est  $\dfrac {47\pi} {3}$.& l'angle orianté a pour mesure pricipale $-\dfrac{\pi}{3}$\\

\hline
\rotatebox{-45}{Correction}&&\\
\hline
3&Déterminer la valeur exacte de :	$\cos (\dfrac {7\pi}{6})=$&$G$$(-\dfrac{\sqrt{3}}{2}$;$-\dfrac{1}{2})$$\Rightarrow$$-\dfrac{\sqrt{3}}{2}$ \\
\hline
\rotatebox{-45}{Correction}&&\\
\hline
4&Résoudre dans $]-\pi;\pi]$ l'équation $\cos (x)=-\dfrac{\sqrt{3}}{2}$&$ \dfrac{5\pi}{6}$ et $-\dfrac{5\pi}{6}$\\
\hline
\rotatebox{-45}{Correction}&&\\
\hline

5&Résoudre dans $[0 ; 2\pi]$, $\sin(x)=-\dfrac12$& $\dfrac{-5\pi}{6}$ et $\dfrac{-\pi}{6}$ \\
\hline
\rotatebox{-45}{Correction}&&\\
\hline
\end{tabular}}


(\cos;\sin)



\end{document}
