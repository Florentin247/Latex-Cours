\documentclass[10pt,a4paper]{article}
\usepackage[utf8]{inputenc}
\usepackage[french]{babel}
\frenchbsetup{StandardLists=true}
\usepackage[T1]{fontenc}
\usepackage{amsmath}
\usepackage{amsthm}
\usepackage{amsfonts}
\usepackage{amssymb}
\usepackage{graphicx}
\usepackage{framed}
\usepackage{fancyhdr}
\usepackage[left=1.3cm,right=1.3cm,top=1.8cm,bottom=1.2cm]{geometry}
\usepackage{array} 
\usepackage{tabularx}
\usepackage{fancybox}
\usepackage{pst-tree}
\usepackage[framed]{ntheorem}
\usepackage{tabularx}
\usepackage{pstricks-add}
\usepackage{graphicx}
\usepackage{eurosym}
%\usepackage{pst-tree}
\usepackage[np]{numprint}
\usepackage{pifont}
\usepackage{mathrsfs}
\usepackage{amssymb}
\usepackage{amsthm}
\usepackage{pgf,tikz}
\usepackage[tikz]{bclogo}
\usepackage{pgfkeys}
\usepackage{mathrsfs}
\usepackage{multicol}
\usetikzlibrary{arrows}


\rfoot{\small -\thepage-}
\cfoot{}

\def\R{{\mathbb R}}
\def\Q{{\mathbb Q}}
\def\Z{{\mathbb Z}}
\def\D{{\mathbb D}}
\def\N{{\mathbb N}}
\def\C{{\mathbb C}}

\pagestyle{fancy}

\renewcommand{\thesection}{\Roman{section}}
\renewcommand{\thesubsection}{\arabic{subsection}}
\renewcommand{\thesubsubsection}{\alph{subsubsection}}
\renewcommand{\labelitemi}{$\bullet$}
\newcommand{\VE}[1]{\overrightarrow{#1}}
%\renewcommand{\r}{($O$ ; $\vec{i}$ , $\vec{j}$)}
\renewcommand{\arraystretch}{1}
\tikzstyle{mybox} = [draw=black, very thick, rectangle, rounded corners, inner sep=20pt, inner ysep=20pt] 
\tikzstyle{fancytitle} =[draw=black, very thick, rectangle, rounded corners, fill=white, text=black] % fill obligé sinon ne recouvre pas boite du dessous
\usepackage{array,multirow,makecell}
\setcellgapes{1pt}
\makegapedcells
\newcolumntype{R}[1]{>{\raggedleft\arraybackslash }b{#1}}
\newcolumntype{L}[1]{>{\raggedright\arraybackslash }b{#1}}
\newcolumntype{C}[1]{>{\centering\arraybackslash }b{#1}}
\theoremstyle{break}
\theorembodyfont{\upshape}
\newtheorem{Prop}{Propri\'et\'e}
\newtheorem{Def}{D\'efinition}
\newtheorem{Rem}{Remarque}
\newtheorem{exo}{Exercice}
\newtheorem{Meth}{Methode}
\newtheorem{cpreuve}{Preuve}
\newtheorem{Th}{Théorème}
\newtheorem{Act}{Activité}
\theorembodyfont{\small \sffamily }
\newtheorem{Ex}{Exemple}
\newtheorem{Preu}{Preuve}
\everymath{\displaystyle}
\pagestyle{fancy}
\lhead[]{\small }
\chead[]{\textsc{\shadowbox{\begin{large}
				\textcolor{black}{QUESTIONS du JOUR 2}
\end{large}}}}
\rhead[]{\small {}}
\fancyhead[R]{2019/2020}
\lfoot{}
\rfoot{\small -\thepage-}

\begin{document}



A faire sans calculatrice , compléter sur le sujet et coller dans le cahier d'exercices . \\
{\renewcommand{\arraystretch}{3}
	\begin{tabular}{|p{2cm}|p{10cm}|p{4.5cm}| }
		\hline
		&Enoncé&Réponse\\
		\hline
		1&Résoudre dans $\left[ 0;2\pi\right[ $ l'équation:  $\cos(x) = \sin\dfrac{7\pi}{6}$&$\dfrac{2\pi}{3}$ et $\dfrac{4\pi}{3} $\\
		\hline
		\rotatebox{-45}{Correction}&&\\
		\hline
		2&Si la représentation graphique $\mathscr{C}$ d'une fonction $f$ passe par le point $A(2;5)$ et si $f'(2)=1$ alors l'équation de la tangente est?&$y=x+3$\\
		
		\hline
		\rotatebox{-45}{Correction}&&\\
		\hline
		3&La courbe ci-dessous est la représentation graphique d'une fonction $f$. Quel est le signe de $f'(-2)$?&le signe est positif\\
		
		&\begin{minipage}{9.8cm}
			\begin{center}
				\psset{unit=0.8cm,algebraic=true}
				\def\xmin {-2.7}
				\def\xmax {2.7}
				\def\ymin {-3}
				\def\ymax {3}
				\begin{pspicture*}(\xmin,\ymin)(\xmax,\ymax)
				\psgrid[subgriddiv=2,gridlabels=3pt,gridwidth=0.5pt,griddots=10,subgriddots=10](\xmin,\ymin)(\xmax,\ymax)
				\psaxes{->}(0,0)(\xmin,\ymin)(\xmax,\ymax)
				
				\psplot[linewidth=1pt,linestyle=dashed]{-2.8}{2.8}{x*x*x-3*x}
				\uput[r](1.5,2){$\mathcal{C}_f$}
				\psdot[dotstyle=+](1,1)
				
				\end{pspicture*}
			\end{center}
		\end{minipage}&\\
		\hline
		\rotatebox{-45}{Correction}&&\\
		\hline
		4&Quelle est la dérivée $g'$ de $g$ définie sur $\R$ par $g(x)=x^3-3x$?& $3x^2-3$ $\Rightarrow$$3(x^2-1)$ \\
		\hline
		\rotatebox{-45}{Correction}&&\\
		\hline
		
			5&Soit $g'$ la dérivée trouvée à la question 4. Résoudre $g'(x)=0$ et en déduire sur quel(s) intervalle(s) $g$ est croissante & $x = -1$ et $x = 1$ \par donc $]-\infty; -1[\cup]1; +\infty[$\\
		\hline
		\rotatebox{-45}{Correction}&C'est une fonction du second degré donc la fonction est croissante en dehors des racines.&\\
		\hline
\end{tabular}}

\end{document}
