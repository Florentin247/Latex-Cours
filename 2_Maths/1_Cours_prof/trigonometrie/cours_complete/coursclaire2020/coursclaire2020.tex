\documentclass[10pt,a4paper]{article}
\usepackage[utf8]{inputenc}
\usepackage[french]{babel}
\frenchbsetup{StandardLists=true}
\usepackage[T1]{fontenc}
\usepackage{amsmath}
\usepackage{amsthm}
\usepackage{amsfonts}
\usepackage{amssymb}
\usepackage{graphicx}
\usepackage{framed}
\usepackage{fancyhdr}
\usepackage[left=1.3cm,right=1.3cm,top=1.8cm,bottom=1.2cm]{geometry}
\usepackage{array} 
\usepackage{fancyhdr} 
\usepackage{fancybox}
\usepackage{pst-tree}
\usepackage[framed]{ntheorem}
\usepackage{tabularx}
\usepackage{pstricks-add}
\usepackage{pstricks,pst-plot,pst-eucl}
\usepackage{eurosym}
\usepackage{pst-tree}
\usepackage[np]{numprint}
\usepackage{pifont}
\usepackage{mathrsfs}
\usepackage{amssymb}
\usepackage{amsthm}
\usepackage{pgf,tikz}
\usepackage[tikz]{bclogo}
\usepackage{pgfkeys}
\usepackage{mathrsfs}
\usepackage{multicol}
\usetikzlibrary{arrows}
\usepackage{listingsutf8}
\lstset{%
	language=python,%
	basicstyle=\ttfamily,%
}

\rfoot{\small -\thepage-}
\cfoot{}

\def\R{{\mathbb R}}
\def\Q{{\mathbb Q}}
\def\Z{{\mathbb Z}}
\def\D{{\mathbb D}}
\def\N{{\mathbb N}}
\def\C{{\mathbb C}}

\pagestyle{fancy}
\newcommand{\nwc}{\newcommand}
\nwc{\dsp}{\displaystyle}
\nwc{\ct}{\centerline}
\nwc{\bge}{\begin{equation}}\nwc{\ene}{\end{equation}}
\nwc{\bgar}{\begin{array}}\nwc{\enar}{\end{array}}
\nwc{\bgit}{\begin{itemize}}\nwc{\enit}{\end{itemize}}
\nwc{\bgen}{\begin{enumerate}}\nwc{\enen}{\end{enumerate}}

\nwc{\la}{\left\{}\nwc{\ra}{\right\}}
\nwc{\lp}{\left(}\nwc{\rp}{\right)}
\nwc{\lb}{\left[}\nwc{\rb}{\right]}

\nwc{\bgsk}{\bigskip}
\nwc{\vsp}{\vspace{0.1cm}}
\nwc{\vspd}{\vspace{0.2cm}}
\nwc{\vspt}{\vspace{0.3cm}}
\nwc{\vspq}{\vspace{0.4cm}}
\renewcommand{\thesection}{\Roman{section}}
\renewcommand{\thesubsection}{\arabic{subsection}}
\renewcommand{\thesubsubsection}{\alph{subsubsection}}
\renewcommand{\labelitemi}{$\bullet$}
\newcommand{\VE}[1]{\overrightarrow{#1}}
%\renewcommand{\r}{($O$ ; $\vec{i}$ , $\vec{j}$)}
\renewcommand{\arraystretch}{0.7}
\tikzstyle{mybox} = [draw=black, very thick, rectangle, rounded corners, inner sep=20pt, inner ysep=20pt] 
\tikzstyle{fancytitle} =[draw=black, very thick, rectangle, rounded corners, fill=white, text=black] % fill obligé sinon ne recouvre pas boite du dessous
\usepackage{array,multirow,makecell}
\setcellgapes{1pt}
\newcommand{\Syst}[2]{\left\{\begin{array}{ccc} #1\\ #2 \end{array}\right.}

\makegapedcells
\newcolumntype{R}[1]{>{\raggedleft\arraybackslash }b{#1}}
\newcolumntype{L}[1]{>{\raggedright\arraybackslash }b{#1}}
\newcolumntype{C}[1]{>{\centering\arraybackslash }b{#1}}
\theoremstyle{break}
\theorembodyfont{\upshape}
\newtheorem{Prop}{Propri\'et\'e}
\newtheorem{Def}{D\'efinition}
\newtheorem{Rem}{Remarque}
\newtheorem{exo}{Exercice}
\newtheorem{Meth}{Methode}
\newtheorem{cpreuve}{Preuve}
\newtheorem{Th}{Théorème}
\newtheorem{Act}{Activité}
\theorembodyfont{\small \sffamily }
\newtheorem{Ex}{Exemple}
\newtheorem{Preu}{Preuve}
\everymath{\displaystyle}
\pagestyle{fancy}
\fancyhead[L]{1STI2D}
\fancyhead[C]{COURS TRIGONOMETRIE }
\fancyhead[R]{2019/2020}
\rfoot{\small -\thepage-}
\cfoot{}



\begin{document}
	
	\vspace*{1cm}
	
	\begin{center}
		\shadowbox{\begin{large}
				\textcolor{black}{TRIGONOMÉTRIE}
			\end{large}}
		\end{center}
	\tableofcontents
		\section{Un peu d'histoire}
		\textit{Il faut remonter jusqu’aux babyloniens, 2000 ans avant notre ère, pour trouver les premières traces de tables de données astronomiques. \\
			Car à la base, la trigonométrie est une géométrie appliquée à l’étude du monde, de l’univers et est indissociable de l’astronomie. On attribue à Hipparque de Nicée (-190 ; -120) les premières tables trigonométriques. Elles font correspondre l’angle au centre et la longueur de la corde interceptée dans le cercle. Le grec Claude Ptolémée (90 ? ; 160 ?) poursuit dans l’Almageste les travaux d’Hipparque avec une meilleure précision et introduit les premières formules de trigonométrie.\\
			 Plus tard, l’astronome et mathématicien Regiomontanus (1436 ; 1476), de son vrai nom Johann Müller développe la trigonométrie comme une branche indépendante des mathématiques. Il serait à l’origine de l’usage systématique du terme sinus.\\
			  Au XVIe siècle, le français François Viète (1540 ; 1607), conseiller d’Henri IV, fera évoluer la trigonométrie pour lui donner le caractère qu’on lui connaît aujourd’hui.\\
			   De nos jours, la trigonométrie trouve des applications très diverses, particulièrement dans les sciences physiques. La propogation des ondes, par exemple, est transcrite par des fonctions trigonométriques .}
		\section{Enroulement de la droite des réels sur le cercle trigonométrique}
		\subsection{Le cercle trigonométrique}
		
		\begin{bclogo}[couleur = yellow!30, arrondi = 0.1,logo=\bcbook]{Définition}
				Le plan étant rapporté à un repère orthonormé $(O ; \overrightarrow{i}, \overrightarrow{j})$, \textbf{le cercle trigonométrique} est le cercle $C$ :
				\begin{itemize}
					\item de centre $O$ et de \textbf{rayon $1$}
					\item sur lequel on choisit un sens de parcours appelé \textbf{sens direct ou sens positif (le sens contraire des aiguilles d'une montre)}.
				\end{itemize} 
		\end{bclogo}
			
		\begin{center}
			\psset{unit=2cm,arrowsize=7pt}
			\begin{pspicture}(-1.3,-1.3)(1.3,1.4)
			\rput(-0.15,-0.15){$O$}
			\psline(-1.2,0)(1.2,0)
			\psline[linewidth=1.4pt]{->}(0,0)(1,0)\rput(0.5,-0.15){$\vec{i}$}
			\psline(0,-1.2)(0,1.2)
			\psline[linewidth=1.4pt]{->}(0,0)(0,1)\rput(-0.15,0.5){$\vec{j}$}
			\pscircle(0,0){1}
			%
			\psarc{->}(0,0){1.3}{20}{65}\rput(1.1,1.){\LARGE$+$}
			\psarc{<-}(0,0){1.3}{-65}{-20}\rput(1.1,-1.){\LARGE$-$}
			\end{pspicture}
		\end{center}
		\subsection{Enroulement de la droite réelle}
		Étant donné le cercle trigonométrique $\mathcal{C}$ de centre O, dans le repère orthonormé direct (O,I, J), on considère la droite $(IA)$  tangente à $\mathcal{C}$ passant par I, A étant le point de coordonnées A(1;1).\\
		Sur cette droite, on considère le repère (I,A), ce qui permet
		d’établir une graduation. A chaque point de cette droite
		correspond un nombre réel et à chaque nombre réel
		correspond un point de la droite (0 correspond au point I,
		1 au point A).\\
		La droite (IA) représente donc l’ensemble $\R$ des nombres
		réels.\\
		Ensuite, on « enroule » cette droite réelle (IA) sur le cercle
		trigonométrique soit dans le sens direct soit dans le sens indirect. (voir graphiques ci-dessous).\\
			\begin{minipage}{8cm}
		\psset{unit=1.5cm}
		\begin{center}
			\begin{pspicture}(-1.2,-1.5)(1.2,2.5)
			\newrgbcolor{bleu}{0.1 0.05 .5}
			\newrgbcolor{prune}{.6 0 .48}
			\def\pshlabel#1{\footnotesize #1}
			\def\psvlabel#1{\footnotesize #1}
			\psaxes[linewidth=.75pt,labels=none,ticks=none]{->}(0,0)(-1.4,-1.4)(1.4,1.4)
			\psaxes[linewidth=1.5pt,linecolor=red]{->}(0,0)(1,1)
			\uput[dl](0,0){\footnotesize{O}}\uput[dr](1,0){\footnotesize{ $I$}}\uput[ul](0,1){\footnotesize{$J$}}\uput[d](.5,0){\footnotesize{\red $\VE{i}$}}\uput[l](0,.5){\footnotesize{\red $\VE{j}$}}
			\psline[linewidth=1pt, linecolor=bleu]{->}(1,-1.5)(1,2.8)
			\psline[linewidth=1.5pt, linecolor=prune](1,0)(1,2.18)
			\psbezier[linewidth=1pt,linestyle=dashed, linecolor=prune,arrowscale=1.5]{->}(1,2.18)(.4,2.18)(-.4,1.8)(-.57242,.819952)
			\psarc[linewidth=1pt, linecolor=bleu](0,0){1}{124.9} {360}
			\psarc[linewidth=1.5pt,linecolor=prune](0,0){1}{0}{124.92}
			\psdot[linecolor=prune](-.57242,.819952)
			\psline[linewidth=.75pt](.92,1)(1.08,1)
			\uput[dr](1,2.8){\bleu{$\mathcal{D}$}}
			\uput[r](1,1){\bleu{$A$}}
			\uput[l](1,1){\footnotesize{1}}
			\uput[r](1,2.18){$x$}
			\uput[ul](-.57,.82){$M$}
			\uput[dl](-.75,-.75){\bleu{$\mathcal{C}$}}
			\end{pspicture}
	\end{center}
	\end{minipage}
	\psset{unit=0.7cm}
		\begin{minipage}{7cm}
			\begin{pspicture}(-5,-6.8)(8,7)
			\psaxes[labels=none,dx=2,dy=2]{->}(0,0)(-5,-6)(8,7)
			%\psgrid[subgriddiv=1,griddots=10,gridlabels=0](-5,-6)(8,7)
			\pscircle[linecolor=red](0,0){2}
			\rput(-0.3,-0.3){\textcolor{red}{$O$}}
			\psline[linecolor=green,linewidth=0.07](2,0)(2,-6)
				\psline[linecolor=blue,linestyle=dotted,linewidth=0.07](2,0)(2,6)
			%spirale
			\parametricplot[plotpoints=200,linecolor=blue]{0}{6.8}
			{2.72 t 5 div exp t 57.3 mul cos mul 2 mul
				2.72 t 5 div exp t 57.3 mul sin mul 2 mul}
			\parametricplot[plotpoints=200,linecolor=blue,linestyle=dashed]{6.8}{7.2}
			{2.72 t 5 div exp t 57.3 mul cos mul 2 mul
				2.72 t 5 div exp t 57.3 mul sin mul 2 mul}
			%points spirale
			\pstGeonode[labelColor=blue,PosAngle={45,90,90,130,165,190,200,230,280},PointName={1,2,3,\pi,4,5,6,2\pi,2\pi+1},linecolor=blue](1.50,1.87){S1}(-0.23,2.77){S2}(-2.15,2.34){S3}(-2.38,2.18){Sp}(-3.46,0.87){S4}(-3.81,-1.08){S5}(-3.19,-2.96){S6}(-2.87,-3.42){S2p}(-1.33,-4.67){S}
			\pstGeonode[PointName=none,linecolor=blue](-1.81,-4.39){A}(0.03,-5.14){A}(2.01,-5.15){A}(3.89,-4.47){A}(5.43,-3.22){A}(6.53,-1.55){A}(7.09,0.36){A}(7.11,2.35){A}(6.61,4.29){A}
			%points cercle
			\pstGeonode[labelColor=red,PosAngle={225,280,0,320,45,100,165,145},PointName={1,2,3,\pi,4,5,6,2\pi},linecolor=red,PointNameSep=0.4](1.08,1.68){C1}(-0.83,1.82){C2}(-1.98,0.28){C3}(-2,0){Cp}(-1.31,-1.51){C4}(0.57,-1.92){C5}(1.92,-0.56){C6}(2,0){C2p}
			%pointilles
			\pscurve[linestyle=dashed](1.50,1.87)(1.25,1.8)(1.08,1.68)
			\pscurve[linestyle=dashed](-0.23,2.77)(-0.6,2.5)(-0.83,1.82)
			\pscurve[linestyle=dashed](-2.15,2.34)(-2.4,1.2)(-1.98,0.28)
			\pscurve[linestyle=dotted,linewidth=0.07,linecolor=purple](-2.38,2.18)(-2.5,1)(-2,0)
			\psbezier[linestyle=dashed](-3.46,0.87)(-3,-0.4)(-2,-1.15)(-1.31,-1.51)
			\psbezier[linestyle=dashed](-3.81,-1.08)(-3,-2)(-1,-2.4)(0.57,-1.92)
			\pscurve[linestyle=dashed](-3.19,-2.96)(0,-2.8)(1.5,-1.8)(1.92,-0.56)
			\pscurve[linestyle=dotted,linewidth=0.07,linecolor=purple](-2.87,-3.42)(0,-3)(1.5,-2)(2,0)
			\pscurve[linestyle=dotted,linewidth=0.07,linecolor=purple](-1.33,-4.67)(0,-4.3)(1.5,-3.2)(2.5,-1)(2,1)(1.08,1.68)
			\end{pspicture}
		\end{minipage}

		
	\begin{bclogo}[couleur = red!25, arrondi = 0.1,logo=\bcbook]{Propriété}
			\begin{itemize}
				\item Pour tout réel $x$, le point d'abscisse $x$ sur la droite $(IA)$ vient s'appliquer sur un point $M$ \textbf{unique} de $C$ appelé\textbf{ image du réel $x$ sur le cercle.}\\
				\item Tout point $N$ du cercle trigonométrique est l'image d'un réel $x'$. Il est alors aussi l'image de tous les réels $x'+2\pi$, $x'+4\pi$, ....., $x'-2\pi$, $x'-4\pi$ c'est à dire $x'+k\times 2\pi$ avec $k\in \mathbb{Z}$. ($2\pi$ étant la longueur d'un tour de cercle).
				
			\end{itemize}
		\end{bclogo}
		\begin{Ex}\begin{itemize}
				\item $0$ a pour image $I(1;0)$,de même $2\pi$ $4\pi$,$6\pi$ ... ont aussi pour image I.\\
				\item  $\dfrac{\pi}{2}$ a pour image $J(0;1)$, $\pi$ a pour image $I'(-1;0)$, $\dfrac{3\pi}{2}$ a pour image $J'(0;-1)$, $-\dfrac{\pi}{2}$ a pour image $I'(0;-1)$
				\item Le point $J(0;1)$ est aussi l'image de $\dfrac{\pi}{2}$ mais aussi de $\dfrac{5\pi}{2}$ et de $-\dfrac{3\pi}{2}$.
			\end{itemize}
			\newpage
		\end{Ex}
		\begin{Meth}[Repérage de points sur le cercle trigonométrique]
			On a placer sur le cercle trigonométrique suivant des points.\\
			\begin{center}
			\begin{center}
				\psset{unit=4cm}
				\begin{pspicture}(-1.2,-1.2)(1.2,1.2)
				\newrgbcolor{bleu}{0.1 0.05 .5}
				\newrgbcolor{prune}{.6 0 .48}
				\newrgbcolor{rose}{.95 .8 .9}
				\def\pshlabel#1{\footnotesize #1}
				\def\psvlabel#1{\footnotesize #1}
				\psaxes[linewidth=.75pt,labels=none,ticks=none]{->}(0,0)(-1.2,-1.2)(1.2,1.2)
				\psaxes[linewidth=1.5pt,linecolor=red]{->}(0,0)(1,1)
				\uput[dl](0,0){\footnotesize{O}}\uput[dr](1,0){\footnotesize{\prune $I$}}\uput[ul](0,1){\footnotesize{\prune $J$}}
				\pscircle[linewidth=1.25pt, linecolor=bleu,linestyle=solid](0,0){1} 
				\psset{linecolor=prune,linewidth=.5pt,linestyle=dashed,labelsep=4pt}
				\rput{0}(0,0){\multido{\n=45+90,\i=1+1}{4}{\cnode*(1;\n){2pt}{A\i}}}
				\rput{0}(0,0){\multido{\n=30+30,\i=1+1}{12}{\cnode*(1;\n){2pt}{B\i}}}
				\ncline{A1}{A2}  \ncline{A2}{A3}  \ncline{A3}{A4}  \ncline{A4}{A1}
				\ncline{B1}{B5}  \ncline{B5}{B7}  \ncline{B7}{B11}  \ncline{B11}{B1}
				\ncline{B2}{B4}  \ncline{B4}{B8} \ncline{B8}{B10} \ncline{B10}{B2} 
				\nput{10}{A1}{\footnotesize{$A~\textcolor{blue}{\dfrac{\pi}{4}}$}}\nput{170}{A2}{\footnotesize{$\textcolor{blue}{\dfrac{3\pi}{4}}~B$}}\nput{-170}{A3}{\footnotesize{$\textcolor{red}{\dfrac{-3\pi}{4}}~C$}}\nput{-10}{A4}{\footnotesize{$D~\textcolor{red}{\dfrac{-\pi}{4}}$}}
				\nput{0}{B1}{\footnotesize{$E~\textcolor{blue}{\dfrac{\pi}{6}}$}}\nput{180}{B5}{\footnotesize{$\textcolor{blue}{\dfrac{5\pi}{6}}~F$}}\nput{-180}{B7}{\footnotesize{$\textcolor{red}{\dfrac{-5\pi}{6}}~G$}}\nput{0}{B11}{\footnotesize{$H~\textcolor{red}{\dfrac{-\pi}{6}}$}}
				\nput{25}{B2}{\footnotesize{$K~\textcolor{blue}{\dfrac{\pi}{3}}$}}\nput{145}{B4}{\footnotesize{$\textcolor{blue}{\dfrac{2\pi}{3}}~L$}}\nput{-105}{B8}{\footnotesize{$\textcolor{red}{\dfrac{-2\pi}{3}}~M$}}\nput{-105}{B10}{\footnotesize{$N~\textcolor{red}{\dfrac{-\pi}{3}}$}}
					\uput[ur](0,1){\footnotesize{$J$}}\uput[ul](-1,0){\footnotesize{$I'$}}\uput[dl](0,-1){\footnotesize{$J'$}}
				\end{pspicture}
			\end{center}
			\end{center}
			
			\begin{enumerate}
				\item Associer à chacun des points le nombre réel de l'intervalle $]-\pi;\pi]$ dont \textcolor{blue}{il est le point image.}
				\item Associer à chacun des points le nombre réel de l'intervalle $[0;2\pi[$ dont \textcolor{red}{il est le point image.}
				\item  Placer sur le cercle les points $M', N'$ et $P'$ associés respectivement aux réels $\dfrac{19\pi}{3}$, $\dfrac{-21\pi}{4}$ et $ \dfrac{53\pi}{3}$
				
				\textit{\textbf{1 tour 2$\pi$ / $\dfrac{4\pi}{2}$ / $\dfrac{6\pi}{3}$ / $\dfrac{8\pi}{4}$ / $\dfrac{12\pi}{6}$}}

			\begin{itemize}
				\item on cherche la mesure principale de : $\dfrac{19\pi}{3}$\par
				on commence par diviser le numérateur par le dénominateur sans $\pi$.\par
				~~~~~~~~~~~~~~~$\dfrac{19}{3} \approx 6.3$ on encadre $6.3$ par 2 chiffres pair ici $6$ et $8$; $6\pi<\dfrac{19\pi}{3}<8\pi$	\par
				~~~~~~~~~~~~~~~~~~~~~~$\dfrac{18\pi}{3}<\dfrac{19\pi}{3}<\dfrac{24\pi}{3}$
				~~~~~~~~~~~~~~~$\dfrac{19\pi}{3}=\dfrac{18\pi}{3}+\dfrac{\pi}{3}$\par
				~~~~~~~~~~~~~~~Donc $\dfrac{\pi}{3}$ Est la mesure principale.
		\end{itemize}
			\begin{itemize}
				\item on cherche la mesure principale de : $\dfrac{-21\pi}{4}$\par
				on commence par diviser le numérateur par le dénominateur sans $\pi$.\par
				~~~~~~~~~~~~~~~$\dfrac{-21}{4} \approx -5.25$ on encadre $-5.25$ par 2 chiffres pair ici $-6$ et $-4$; $-6\pi<\dfrac{-21\pi}{4}<-4\pi$	\par
				~~~~~~~~~~~~~~~~~~~~~~$\dfrac{-24\pi}{4}<\dfrac{-21\pi}{4}<\dfrac{-16\pi}{4}$
				~~~~~~~~~~~~~~~$\dfrac{-21\pi}{4}=\dfrac{-24\pi}{4}+\dfrac{3\pi}{4}$\par
				~~~~~~~~~~~~~~~Donc $\dfrac{3\pi}{4}$ Est la mesure principale.
		\end{itemize}
			\begin{itemize}
				\item on cherche la mesure principale de : $\dfrac{53\pi}{3}$\par
				on commence par diviser le numérateur par le dénominateur sans $\pi$.\par
				~~~~~~~~~~~~~~~$\dfrac{53}{3} \approx 17.66666$ on encadre $17.6666$ par 2 chiffres pair ici $16$ et $18$; $16\pi<\dfrac{53\pi}{3}<18\pi$	\par
				~~~~~~~~~~~~~~~~~~~~~~$\dfrac{48\pi}{3}<\dfrac{53\pi}{3}<\dfrac{54\pi}{3}$
				~~~~~~~~~~~~~~~$\dfrac{53\pi}{3}=\dfrac{54\pi}{3}-\dfrac{\pi}{3}$\par
				~~~~~~~~~~~~~~~Donc $\dfrac{\pi}{3}$ Est la mesure principale.
		\end{itemize}
			\end{enumerate}
		\end{Meth}
	\newpage
		\section{Radian}	
		\subsection{Définition}
			\begin{bclogo}[couleur = yellow!30, arrondi = 0.1,logo=\bcbook]{Définition}
		\begin{minipage}{12cm}
			
				Pour tout point M sur le cercle trigonométrique, on définit un angle géométrique $\widehat{IOM}$. Cet angle géométrique intercepte l'arc $\overset{\frown}{IM}$ du cercle trigonométrique.\\
				La mesure de l'angle géométrique $\widehat{IOM}$ est de 1 radian lorsque la mesure de l'arc $\overset{\frown}{IM}$ est de 1 rayon.\\
			\end{minipage}
		\hfill
		\begin{minipage}{6cm}
							\psset{unit=1.5cm}
										\begin{pspicture}(-1.2,-1.2)(1.2,1.2)
						\newrgbcolor{bleu}{0.1 0.05 .5}
						\newrgbcolor{prune}{.6 0 .48}
						\newrgbcolor{rose}{.95 .8 .9}
						\def\pshlabel#1{\footnotesize #1}
						\def\psvlabel#1{\footnotesize #1}
						\pswedge[linecolor=rose,linewidth=.5pt,fillstyle=solid,fillcolor=rose]{.4}{0}{57.2958}
						\psaxes[linewidth=.75pt,labels=none,ticks=none]{->}(0,0)(-1.2,-1.2)(1.2,1.2)
						\psaxes[linewidth=1.5pt,linecolor=red]{->}(0,0)(1,1)
						\uput[dl](0,0){\footnotesize{O}}\uput[dr](1,0){\footnotesize{$I$}}\uput[ul](0,1){\footnotesize{ $J$}}\uput[d](.5,0){\footnotesize{\red $\VE{i}$}}\uput[l](0,.5){\footnotesize{\red $\VE{j}$}}
						\rput{57.2958}{\psline[linewidth=1.25pt,linecolor=prune](0,0)(1,0)\rput[bl]{*0}(1.05,0.05){\prune $M$}}
						\uput[r](.3,.2){\footnotesize{\prune{1 rad}}}
						\psarc[linecolor=prune]{-}(0,0){.4}{0}{57.2958}
						\psarc[linecolor=bleu,linewidth=1pt]{-}(0,0){1}{57.29}{360}
						\psarc[linecolor=prune,linewidth=1.25pt]{-}(0,0){1}{0}{57.2958}
						\end{pspicture}
									\end{minipage}
		\end{bclogo}
		\subsection{Conversion}
		Le périmètre d'un cercle de rayon 1 est 2$\pi$.
		Donc $2\pi$ radians équivaut à $360^{\circ}$.\\ 
		
		Soit 1 radian correspond à  $\frac{360^{\circ}}{2\pi}$ ou 1 radian $\approx57,30^{\circ}$.\\
		On retiendra : $\pi$ radians correspond à $180^{\circ}$\\
		\begin{Meth}[Convertir des mesures d'angle]
			Convertir les mesures d'angles donnés en radians ou en degrés.\\
		
				\renewcommand{\arraystretch}{1.8}
				\begin{tabular}{|c|*{11}{p{1.2cm}|}}
					\hline
					
					Mesures en degrés&15&30&45&60&90&125&135&180&270&300&360\\
					
					\hline
					
					Mesures  en radians&$\dfrac{\pi}{12}$&$\dfrac{\pi}{6}$&$\dfrac{\pi}{4}$&$\dfrac{\pi}{3}$&$\dfrac{\pi}{2}$&$\dfrac{25\pi}{36}$&$\dfrac{3\pi}{4}$&$\pi$&$\dfrac{3\pi}{2}$&$\dfrac{5\pi}{3}$&\\
					
					\hline
				\end{tabular}
		(Pour trouver facilement par la suite toutes les valeurs, on peut faire un produit en crois.)	
		\end{Meth}
	
		\section{Angle orienté }
		\subsection{Mesures d'angle orienté }
	
			\begin{bclogo}[couleur = yellow!30, arrondi = 0.1,logo=\bcbook]{Définition}
		
		\begin{minipage}{11cm} Soit $x$ un réel et $M$ un point du cercle trigonométrique repéré par $x$.
						
						On dit que $x$ est une mesure en radians de l'angle orienté $\left( \VE{OI}, \VE{OM}\right)$.
					
			\end{minipage}
			\hfill
			\begin{minipage}{4.8cm}
				\psset{unit=2cm}
				\begin{pspicture}(-1.2,-1.2)(1.2,1.2)
				\newrgbcolor{bleu}{0.1 0.05 .5}
				\newrgbcolor{prune}{.6 0 .48}
				\newrgbcolor{rose}{.95 .8 .9}
				\def\pshlabel#1{\footnotesize #1}
				\def\psvlabel#1{\footnotesize #1}
				\pswedge[linecolor=rose,linewidth=.5pt,fillstyle=solid,fillcolor=rose]{.4}{0}{50}
				\psaxes[linewidth=.75pt,labels=none,ticks=none]{->}(0,0)(-1.2,-1.2)(1.2,1.2)
				\psaxes[linewidth=1.5pt,linecolor=red]{->}(0,0)(1,1)
				\uput[dl](0,0){\footnotesize{O}}\uput[dr](1,0){\footnotesize{\prune $I$}}\uput[ul](0,1){\footnotesize{\prune $J$}}\uput[d](.5,0){\footnotesize{\red $\VE{i}$}}\uput[l](0,.5){\footnotesize{\red $\VE{j}$}}
				\pscircle[linewidth=1.25pt, linecolor=bleu](0,0){1} 
				\rput{50}{\psline[linewidth=1.25pt,linecolor=prune](0,0)(1,0)\rput[bl]{*0}(1,0){\prune $M$}}
				\uput[r](.3,.2){$x$}
				\psarc[linecolor=prune]{->}(0,0){.4}{0}{50}
				\psarc[linecolor=prune]{->}(0,0){1.3}{30}{60}
				\rput(1,1){\prune +}
				\end{pspicture}
			\end{minipage}
			
			\medskip
			\end{bclogo}
	
		\begin{Rem}
			L'angle orienté $(\overrightarrow{OI},\overrightarrow{OM})$ a une infinité de mesures.\\
			Si $x $ est une de ses mesures  alors $x+2\pi $ , $x+4\pi $ , $x-2\pi $ et plus généralement $x+2k\pi$ avec $k\in\mathbb{Z}$  sont aussi des mesures de $(\overrightarrow{OI},\overrightarrow{OM})$.\\
			Par convention on note alors $\left( \VE{OI}, \VE{OM}\right)= x +2k\pi$ où $k$ est un entier relatif et on dit que $\left( \VE{OI}, \VE{OM}\right)$ à pour mesure $x$ radians à $2\pi$ près.\\
			On peut aussi  écrire: mes $(\VE{OI},\VE{OM}) $ $\equiv \alpha$  $[2\pi]$ ou plus simplement:\\
				$(\VE{OI},\VE{OM})$  $\equiv$ $ \alpha$    $[2\pi]$  qui se lit: \textbf{La mesure de l'angle orienté $(\VE{OI},\VE{OM})$ est égal à $\alpha$ modulo $2\pi$}
		\end{Rem}
	\newpage
		\subsection{Mesure principale d'un angle orienté}
			\begin{bclogo}[couleur = yellow!30, arrondi = 0.1,logo=\bcbook]{Définition}
			Parmi toutes les mesures d'un angle orienté $(\VE{OI}, \VE{OM})$, il existe une unique mesure de l'angle appartenant à l'intervalle $\left] -\pi;\pi\right] $. Cette mesure s'appelle  appelée la \textbf{mesure principale} de $(\VE{OI}, \VE{OM})$.
		\end{bclogo}
		\begin{Meth}[Savoir trouver une mesure principale d'un angle orienté:]
			\begin{enumerate}
				\item Parmi les nombres suivants quels sont ceux qui peuvent être des mesures principales d'angles.
				\begin{multicols}{4}
					\begin{enumerate}
						\item $\dfrac{5\pi}{2}$ 
						\item $\fbox{$-\dfrac{7\pi}{12}$}$
						\item $\fbox{$\dfrac{-11\pi}{12}$}$
						\item $\dfrac{7\pi}{4}$
					\end{enumerate}
				\end{multicols}
				Quand le numérateur est inférieur au dénominateur, c'est une mesure principale.\par
				\item Déterminer la mesure principale associée à chacune des mesures suivantes:
				\begin{multicols}{4}
					\begin{enumerate}
						\item $\dfrac{47\pi}{3}$~$\rightarrow$~$-\dfrac{\pi}{3}$
						\item $-\dfrac{61\pi}{6}$~$\rightarrow$~$-\dfrac{\pi}{6}$
						\item $\dfrac{-23\pi}{5}$~$\rightarrow$~$\dfrac{2\pi}{3}$
						\item $\dfrac{-131\pi}{4}$~$\rightarrow$~$\dfrac{-3\pi}{4}$
					\end{enumerate}
				\end{multicols}
			\end{enumerate}
		\end{Meth}
	
		\section{Cosinus et sinus d'un nombre réel}
		\subsection{Définition}
		\begin{bclogo}[couleur = yellow!30, arrondi = 0.1,logo=\bcbook]{Définition}
		\begin{minipage}{0.5\textwidth}
			
				Dans le plan rapporté a un repère orthonormé direct $(O;\overrightarrow{OI},\overrightarrow{OJ})$ soit $M$ le point du cercle trigonométrique $\mathcal{C}$ image du réel $x$. On définit:\\
				\begin{itemize}
					\item le cosinus de $x$  qui est l'abscisse de $M$. Il est noté $\cos(x)$ ou $\cos x$.
					\item le sinus de $x$  qui est l'ordonnée de $M$. Il est noté $\sin(x)$ ou $\sin x$.
				\end{itemize}
				
		
		\end{minipage}
	\begin{minipage}{5.8cm}
		\begin{center}
		\psset{unit=2cm}
		\begin{pspicture}(-1.2,-1.2)(1.2,1.2)
		\newrgbcolor{bleu}{0.1 0.05 .5}
		\newrgbcolor{prune}{.6 0 .48}
		\newrgbcolor{rose}{.95 .8 .9}
		\def\pshlabel#1{\footnotesize #1}
		\def\psvlabel#1{\footnotesize #1}
		\psaxes[linewidth=.75pt,labels=none,ticks=none]{-}(0,0)(-1.2,-1.2)(1.2,1.2)
		\psaxes[linewidth=.75pt]{->}(0,0)(1,1)
		\uput[dl](0,0){\footnotesize{O}}\uput[dr](1,0){\footnotesize{$I$}}\uput[ul](0,1){\footnotesize{$J$}}
		\pscircle[linewidth=1.25pt, linecolor=bleu](0,0){1} 
		\rput{50}{\psline[linewidth=1.25pt,linecolor=prune](0,0)(1,0)}
		\psline[linewidth=1pt,linecolor=prune,linestyle=dashed](0,.766)(.642,.766)(.642,0)
		\uput[ur](.642,.766){\prune $M$}\uput[d](.642,0){$\cos x$}\uput[l](0,.766){$\sin x$}
		\uput[r](.3,.2){$x$}
		\psarc[linecolor=prune]{->}(0,0){.4}{0}{50}
		\end{pspicture}
		\end{center}
	\end{minipage}  	\end{bclogo}
		\subsection{Propriétés}
		$\forall x \in \R$ on a les propriétés suivantes:
		\begin{bclogo}[couleur = red!25, arrondi = 0.1,logo=\bcbook]{Propriétés}
		
				\begin{minipage}{5cm}
					\begin{itemize}
						\item $-1\leqslant \cos x\leqslant1$
						\item $-1\leqslant \sin x\leqslant1$
						\item $\cos^2 x +\sin^2x=1$
					\end{itemize} 
				\end{minipage}
			\end{bclogo}
		\subsection{Valeurs remarquables}
		\begin{minipage}{10cm}
			\begin{tabular}{|l|c|c|c|c|c|}
				\hline
				\small
				&&&&&\\
				$x$&$~~0~~$&$~~\dfrac{\pi}{6}~~$&$~~\dfrac{\pi}{4}~~$&$~~\dfrac{\pi}{3}~~$&$~~\dfrac{\pi}{2}~~$\\
				&&&&&\\
				\hline
				&&&&&\\
				$\cos x$&$1$&$\dfrac{\sqrt{3}}{2}$&$\dfrac{\sqrt{2}}{2}$&$\dfrac{1}{2}$&$0$\\
				&&&&&\\
				\hline
				&&&&&\\
				$\sin x$&$0$&$\dfrac{1}{2}$&$\dfrac{\sqrt{2}}{2}$&$\dfrac{\sqrt{3}}{2}$&$1$\\
				&&&&&\\
				\hline
			\end{tabular}
		\end{minipage}
		\begin{minipage}{8cm}
	\begin{center}
			\psset{unit=4cm}
			\begin{pspicture}(-.1,-.2)(1.1,1.1)
			\newrgbcolor{bleu}{0.1 0.05 .5}
			\newrgbcolor{prune}{.6 0 .48}
			\newrgbcolor{rose}{.95 .8 .9}
			\def\pshlabel#1{\footnotesize #1}
			\def\psvlabel#1{\footnotesize #1}
			\psaxes[linewidth=.75pt,labels=none,ticks=none]{-}(0,0)(-.1,-.1)(1.1,1.1)
			\psaxes[linewidth=1pt]{->}(0,0)(1,1)
			\uput[dl](0,0){\footnotesize{O}}
			\psarc[linecolor=bleu,linewidth=1.25pt](0,0){1}{0}{90}
			\psset{linewidth=1pt,linecolor=prune,linestyle=dashed}
			\psline(.866,0)(.866,.5)(0,0.5)\psline(.707,0)(.707,.707)(0,0.707)\psline(0.5,0)(.5,.866)(0,0.866)
			\uput[15](.866,.5){\prune $\dfrac{\pi}{6}$}\uput[d](.866,0){\footnotesize{$\dfrac{\sqrt{3}}{2}$}}\uput[l](0,.5){\footnotesize{$\dfrac{1}{2}$}}
			\uput[30](.707,.707){\prune $\dfrac{\pi}{4}$}\uput[d](.707,0){\footnotesize{$\dfrac{\sqrt{2}}{2}$}}\uput[l](0,.707){\footnotesize{$\dfrac{\sqrt{2}}{2}$}}
			\uput[45](.5,.866){\prune $\dfrac{\pi}{3}$}\uput[d](.5,0){\footnotesize{$\dfrac{1}{2}$}}\uput[l](0,.866){\footnotesize{$\dfrac{\sqrt{3}}{2}$}}
			\uput[ur](0,1){\prune $\dfrac{\pi}{2}$}\uput[l](0,1){\footnotesize{$1$}}\uput[d](1,0){\footnotesize{$1$}}
			\end{pspicture}
			\end{center}
		\end{minipage}
		
		\newpage
	\begin{Meth}[Savoir trouver la valeur exacte d'un sinus ou d'un cosinus]
		Dans chacun des cas donner la valeur exacte des nombres suivants:
		\begin{multicols}{4}
			\begin{enumerate}
				\item $\cos(2020\pi)$
				\item $\sin(-\dfrac{\pi}{2})$
				\item $\cos(\dfrac{2\pi}{3})$
				\item $\sin(\dfrac{7\pi}{6})$
			\end{enumerate}
		\end{multicols}
\end{Meth}		
		\subsection{Relations trigonométriques et angles associés}
			\begin{bclogo}[couleur = red!25, arrondi = 0.1,logo=\bcbook]{Propriétés}	
		\begin{minipage}{9cm}
			Pour tout nombre réel $x$,  
			\[\la\bgar{ll}
			\cos(-x) = \cos x \vspd\\
			\sin(-x) = -\sin x
			\enar\right.
			\]
			\[
			\la\bgar{ll}
			\dsp \cos\lp\frac{\pi}{2}-x\rp = \sin x \vspd\\
			\dsp \sin\lp\frac{\pi}{2}-x\rp = \cos x
			\enar\right.
			\hspace{0.6cm}
			\la\bgar{ll}
			\dsp \cos\lp\frac{\pi}{2}+x\rp = -\sin x \vspd\\
			\dsp \sin\lp\frac{\pi}{2}+x\rp = \cos x
			\enar\right.
			\]
			\[ \la\bgar{ll}
			\cos\lp\pi-x\rp = -\cos x \vspd\\
			\sin\lp\pi-x\rp = \sin x
			\enar\right.
			\hspace{0.6cm}
			\la\bgar{ll}
			\cos\lp\pi+x\rp = -\cos x \vspd\\
			\sin\lp\pi+x\rp = -\sin x
			\enar\right.
			\]
		\end{minipage}
	\hfill
		\begin{minipage}{7cm}
		\psset{unit=2.5cm}%{xunit=5cm,yunit=5cm}
		\begin{pspicture}(-1.3,-1.5)(1.3,1.5)
		
		\pscircle(0,0){1}
		\psline[linewidth=0.8pt](-1.2,0)(1.2,0)
		\psline[linewidth=0.8pt](0,-1.2)(0,1.2)
		\psarc[linewidth=0.6pt]{->}(0,0){0.4}{0}{19}
		\put(0.45,0.05){\large{$x$}}
		
		\psline[linewidth=0.8pt](-0.94,-0.3)(0.94,.3)
		\psline[linewidth=0.8pt](-0.94,0.3)(0.94,-.3)
		\psline[linewidth=0.8pt](-0.3,0.94)(.3,-0.94)
		\psline[linewidth=0.8pt](0.3,0.94)(-.3,-0.94)
		
		\psline[linewidth=0.5pt,linestyle=dashed](0.94,-0.3)(0.94,0.3)
		\psline[linewidth=0.5pt,linestyle=dashed](-0.94,-0.3)(-0.94,0.3)
		\psline[linewidth=0.5pt,linestyle=dashed](-0.94,0.32)(0.94,0.32)
		\psline[linewidth=0.5pt,linestyle=dashed](-0.94,-0.32)(0.94,-0.32)
		\psline[linewidth=0.5pt,linestyle=dashed](-0.32,0.94)(0.32,0.94)
		\psline[linewidth=0.5pt,linestyle=dashed](-0.32,-0.94)(0.32,-0.94)
		\psline[linewidth=0.5pt,linestyle=dashed](-0.32,0.94)(-0.32,-0.94)
		\psline[linewidth=0.5pt,linestyle=dashed](0.32,0.94)(0.32,-0.94)
		
		\put(1.1,-.03){\large{$0$}}
		\put(1.1,0.4){\large{$x$}}
		\put(1,-0.45){\large{$-x$}}
		\put(-1.5,0.35){\large{$\pi-x$}}
		\put(-1.5,-0.35){\large{$\pi+x$}}
		\put(-1.4,0.){\large{$\pi$}}
		\put(-0.7,1.2){\large{$\dsp\frac{\pi}{2}+x$}}
		\put(0.25,1.2){\large{$\dsp\frac{\pi}{2}-x$}}
		\put(-.05,1.35){\large{$\dsp\frac{\pi}{2}$}}
		
		
		\put(1,-0.15){$\mathbf{\cos x}$}
		\put(-0.1,1.05){$\mathbf{\sin x}$}
		\end{pspicture}	
	\end{minipage}

	
		
		\end{bclogo}
		\begin{Rem}
			\begin{itemize}
				\item Il faut savoir retrouver ces formules à partir d'un dessin
				\item Les relations $\cos(\dfrac{\pi}{2}-x)=\sin(x)$ et 	$\sin(\dfrac{\pi}{2}-x)=\cos(x)$ sont très importantes car elles permettent de transformer un sinus en cosinus et vice versa.
			\end{itemize}
		\end{Rem}
		\begin{Meth}[Savoir utiliser les angles associés]
		\begin{enumerate}
			\item Soit $x$ un nombre réel, exprimer les nombres suivants en fonction de $\cos(x)$ ou $sin(x)$
			\begin{enumerate}
				\item A=$\cos(\dfrac{\pi}{2}-x)+\cos(2\pi+x)+2\sin(\pi+x)$
				\item$ B=3\cos(\pi+x)+5sin(\dfrac{\pi}{2}-x)+2\sin(-x)$
			\end{enumerate}
			\item On admet que $\cos(\dfrac{\pi}{5})=\dfrac{1+\sqrt{5}}{4}$
			\begin{enumerate}
				\item En déduire la valeur exacte de $\sin(\dfrac{\pi}{5})$
				\item  Donner ensuite la valeur exacte de cosinus et sinus des nombres réels suivants:
				\begin{multicols}{5}
					\begin{enumerate}
						\item $\dfrac{4\pi}{5}$
						\item $-\dfrac{\pi}{5}$
						\item $\dfrac{6\pi}{5}$
						\item $\dfrac{3\pi}{5}$
						\item $\dfrac{a\pi}{5}$
					\end{enumerate}
				\end{multicols}
			\end{enumerate}
		\end{enumerate}
	\end{Meth}
\newpage
		\subsection{Équations trigonométriques}
		\begin{bclogo}[couleur = red!25, arrondi = 0.1,logo=\bcbook]{Propriétés}
			Le cercle trigonométrique et la configuration des angles associés nous permettent de résoudre des équations 
			du type $\cos x = \cos a$ et $\sin x = \sin a$ où $a$ est un réel connu.
		\begin{itemize}
			\item \textsf{\'Equation} $\cos x =\cos a$
			
			\begin{minipage}{11cm}
				Soit $a$ un réel donné. Les solutions dans $\R$ de l'équation $\cos x =\cos a$ sont :\[ 
					\begin{cases}
					x &= a + 2k \pi\\
					x &= -a + 2k \pi
					\end{cases}
					\text{ où }\;k\;\text{ est un entier relatif.}
					\]
				
				
			\end{minipage}
			\hfill
			\begin{minipage}{4.5cm}
				\begin{flushright}
					\psset{unit=1.8cm}
					\begin{pspicture}(-1.2,-1.1)(1.2,1.2)
					\newrgbcolor{bleu}{0.1 0.05 .5}
					\newrgbcolor{prune}{.6 0 .48}
					\newrgbcolor{rose}{.95 .8 .9}
					\newrgbcolor{grisb}{.85 .85 .9}
					\psaxes[linewidth=.75pt,labels=none,ticks=none]{-}(0,0)(-1.2,-1.1)(1.2,1.2)
					\psaxes[linewidth=1pt,labels=none,ticks=none]{->}(0,0)(1,1)
					\uput[dl](0,0){\footnotesize{O}}\uput[dr](1,0){\footnotesize{\bleu $I$}}\uput[ul](0,1){\footnotesize{\bleu $J$}}
					\pscircle[linewidth=1.25pt, linecolor=bleu](0,0){1} 
					\rput{50}{\psline[linewidth=1.25pt,linecolor=prune](0,0)(1,0)}
					\rput{-50}{\psline[linewidth=1.25pt,linecolor=prune](0,0)(1,0)}
					\uput[ur](.642,.766){\prune $M(a)$}\uput[dr](.642,-.766){\prune $N(-a)$}
					\psline[linewidth=.5pt,linecolor=prune,linestyle=dashed](.642,.766)(.642,-.766)
					\psline[linewidth=1.5pt,linecolor=red](0,0)(.642,0)
					\uput[d](.33,0){\red $\cos a$}
					\end{pspicture}
				\end{flushright}
			\end{minipage}
			\item \textsf{\'Equation} $\sin x =\sin a$
			
			\begin{minipage}{11cm}
				Soit $a$ un réel donné. Les solutions dans $\R$ de l'équation $\sin x =\sin a$ sont :\[ 
					\begin{cases}
					x &= a + 2k \pi\\
					x &=\pi -a + 2k \pi
					\end{cases}
					\text{ où }\;k\;\text{ est un entier relatif.}
					\]
				
			\end{minipage}
			\hfill
			\begin{minipage}{4.5cm}
				\begin{flushright}
					\psset{unit=1.8cm}
					\begin{pspicture}(-1.2,-1)(1.2,1.2)
					\newrgbcolor{bleu}{0.1 0.05 .5}
					\newrgbcolor{prune}{.6 0 .48}
					\newrgbcolor{rose}{.95 .8 .9}
					\newrgbcolor{grisb}{.85 .85 .9}
					\psaxes[linewidth=.75pt,labels=none,ticks=none]{-}(0,0)(-1.2,-1.1)(1.2,1.2)
					\psaxes[linewidth=1pt,labels=none,ticks=none]{->}(0,0)(1,1)
					\uput[dl](0,0){\footnotesize{O}}\uput[dr](1,0){\footnotesize{\bleu $I$}}\uput[ul](0,1){\footnotesize{\bleu $J$}}
					\pscircle[linewidth=1.25pt, linecolor=bleu](0,0){1} 
					\rput{40}{\psline[linewidth=1.25pt,linecolor=prune](0,0)(1,0)}
					\rput{140}{\psline[linewidth=1.25pt,linecolor=prune](0,0)(1,0)}
					\uput[ur](.766,.642){\prune $M(a)$}\uput[ul](-.7,.642){\prune $N (\pi -a)$}
					\psline[linewidth=.5pt,linecolor=prune,linestyle=dashed](-.766,.642)(.766,.642)
					\psline[linewidth=1.5pt,linecolor=red](0,0)(0,.642)
					\uput[l](0,.321){\red $\sin a$}
					\end{pspicture}
				\end{flushright}
			\end{minipage}
		\end{itemize}
		
			
			
		\end{bclogo}
		
	\begin{Meth}[Savoir résoudre des équations trigonométriques]
		Résoudre dans $]-\pi; \pi]$, puis dans $\R$ et enfin dans $[0;2\pi[$ les équations suivantes :
		\begin{multicols}{4}
		\begin{enumerate}
			\item $\cos x = \dfrac{\sqrt{2}}{2} $
			\item $\sin x = -\dfrac{1}{2}$
			\item $\cos (x) = \sin\left(\dfrac{3\pi}{4}\right) $
			\item $\cos (2x) = \dfrac{1}{2}$ 
			
			\item $ \sin \left(x-\dfrac{\pi}{6}\right) = \dfrac{\sqrt{3}}{2}$ 
				
		\end{enumerate}
		\end{multicols}
				\end{Meth}
		\begin{Meth} QCM (plusieurs réponses possibles)
			\begin{enumerate}
				\item  Soit $x\in[\pi ; 2\pi]$ tel que $\cos(x)=-\dfrac45$. Alors on a :
				$$\sin(x) \leq 0 \qquad \sin^2(x)=\dfrac15 \qquad \sin(x)=0.6 \qquad \sin(x)=-\dfrac35$$
				\item  Soit $x\in\left[\dfrac{\pi}2 ; \dfrac{3\pi}2\right]$ tel que $\sin(x)=\dfrac{21}{29}$. Alors on a :
				$$\cos(x) \leq 0 \qquad \cos(x)=\sqrt{1-\sin^2(x)} \qquad \cos(x)=-\dfrac{20}{29} \qquad \cos(x)=\dfrac{20}{29}$$
				\item  Soit $x\in\left[-\dfrac{\pi}2 ; \dfrac{\pi}2\right]$ tel que $\sin(x)=\dfrac{1}{4}$. Alors on a :
				$$\cos(x) \leq 0 \qquad \cos(x)=\dfrac34 \qquad \cos(x)=-\dfrac{\sqrt{15}}{4} \qquad \cos(x)=\dfrac{\sqrt{15}}{4}$$
			
			
			\end{enumerate}
		
		\end{Meth}
	\newpage
	\section{Fonctions cosinus et sinus}
	\subsection{Définitions}
		\begin{bclogo}[couleur = red!25, arrondi = 0.1,logo=\bcbook]{Définitions}
	\begin{itemize}
		\item La fonction cosinus, notée $\cos$,  est la fonction définie sur $\mathbb{R}$ par $\cos : x\mapsto \cos x$.
		\item La fonction sinus, notée $\sin$,  est la fonction  définie sur $\mathbb{R}$ par $\sin : x\mapsto \sin x$.
	\end{itemize}
	\end{bclogo}
	\subsection{Périodicité}
		\begin{bclogo}[couleur = red!25, arrondi = 0.1,logo=\bcbook]{Périodicité}
	Pour tout réel $x$, $\cos \left( x+2\pi \right)= \cos(x)$ et $\sin \left( x+2\pi \right)= \sin(x)$. On dit que les fonctions cosinus et sinus sont périodiques de période $2\pi$.
\end{bclogo}
	\begin{Rem}
			La fonction cosinus ( ou la fonction sinus ) est entièrement connue dès qu'on connaît ses valeurs sur un intervalle $\left[ a ; a+ 2\pi \right[$ d'amplitude $2\pi$.\\
		
		D'un point de vue graphique, les courbes représentatives des fonctions cosinus ou sinus sont inchangées par la translation de vecteur $2\pi \VE{\imath}$
	\end{Rem}

	
	\subsection{parité}
	
	\begin{bclogo}[couleur = red!25, arrondi = 0.1,logo=\bcbook]{Parité}
		\begin{itemize}
			\item Pour tout réel $x$, $\cos (-x)= \cos(x)$. On dit que la fonction cosinus est paire.
			
			La courbe représentative de la fonction cosinus admet l'axe des ordonnées pour axe de symétrie.
			
			\item Pour tout réel $x$, $\sin (-x)= -\sin(x)$. On dit que la fonction sinus est impaire.
			
			La courbe représentative de la fonction sinus admet l'axe l'origine du repère pour centre de symétrie.
		\end{itemize}
	\end{bclogo}
\begin{Rem}
		Il suffit d'étudier les fonctions cosinus et sinus sur l'intervalle $\left[ 0 ; \pi \right]$ pour les connaître sur $\left[ -\pi ; \pi \right]$ à l'aide de la parité et enfin sur $\R$ à l'aide de la périodicité.
\end{Rem}

	
	\subsection{variation}
	
	\psset{xunit=.95cm,yunit=1cm} 
	\begin{pspicture}(6.5,3)
	\rput(3.25,3){Sur $\left[ 0 ; \pi \right]$}
	\psset{linewidth=.5pt}
	\psframe(6.5,2.5)\psline(0,1.5)(6.5,1.5) \psline(1.5,0)(1.5,2.5)
	\rput(0.75,2){$x$} \rput(2,2){$0$} \rput(4,2){$\dfrac{\pi}{2}$} \rput(6,2){$\pi$}
	\rput(0.75,.75){$\cos(x)$} \rput(2,1.2){\rnode{A}{$1$}} \rput(6,.3){\rnode{B}{$-1$}}
	\psset{nodesep=5pt,arrows=->,linewidth=.75pt}
	\ncline{A}{B} \ncput*{0}
	\end{pspicture}
	\hfill
	\psset{xunit=.95cm,yunit=1cm} 
	\begin{pspicture}(10.5,3)
	\rput(5.25,3){Sur $\left[ -\pi ; \pi \right]$}
	\psset{linewidth=.5pt}
	\psframe(10.5,2.5)\psline(0,1.5)(10.5,1.5) \psline(1.5,0)(1.5,2.5)
	\rput(0.75,2){$x$} \rput(2,2){$-\pi$} \rput(4,2){$-\dfrac{\pi}{2}$} \rput(6,2){$0$}\rput(8,2){$\dfrac{\pi}{2}$} \rput(10,2){$\pi$}
	\rput(0.75,.75){$\cos(x)$} \rput(2,.3){\rnode{A}{$-1$}} \rput(6,1.2){\rnode{B}{$-1$}}\rput(10,.3){\rnode{C}{$-1$}}
	\psset{nodesep=5pt,arrows=->,linewidth=.75pt}
	\ncline{A}{B} \ncput*{0}\ncline{B}{C} \ncput*{0}
	\end{pspicture}
	
	\medskip
	
	\psset{xunit=.95cm,yunit=1cm} 
	\begin{pspicture}(6.5,2.5)
	\psset{linewidth=.5pt}
	\psframe(6.5,2.5)\psline(0,1.5)(6.5,1.5) \psline(1.5,0)(1.5,2.5)
	\rput(0.75,2){$x$} \rput(2,2){$0$} \rput(4,2){$\dfrac{\pi}{2}$} \rput(6,2){$\pi$}
	\rput(0.75,.75){$\sin(x)$} \rput(2,.3){\rnode{A}{$0$}} \rput(4,1.2){\rnode{B}{$1$}}\rput(6,.3){\rnode{C}{$0$}}
	\psset{nodesep=5pt,arrows=->,linewidth=.75pt}
	\ncline{A}{B} \ncline{B}{C} 
	\end{pspicture}
	\hfill
	\psset{xunit=.95cm,yunit=1cm} 
	\begin{pspicture}(10.5,2.5)
	\psset{linewidth=.5pt}
	\psframe(10.5,2.5)\psline(0,1.5)(10.5,1.5) \psline(1.5,0)(1.5,2.5)
	\rput(0.75,2){$x$} \rput(2,2){$-\pi$} \rput(4,2){$-\dfrac{\pi}{2}$} \rput(6,2){$0$}\rput(8,2){$\dfrac{\pi}{2}$} \rput(10,2){$\pi$}
	\rput(0.75,.75){$\sin(x)$} \rput(2,1.2){\rnode{A}{$0$}} \rput(4,.3){\rnode{B}{$-1$}}\rput(8,1.2){\rnode{C}{$1$}}\rput(10,.3){\rnode{D}{$0$}}
	\psset{nodesep=5pt,arrows=->,linewidth=.75pt}
	\ncline{A}{B} \ncline{B}{C} \ncput*{0}  \ncline{C}{D}
	\end{pspicture}
	
	\subsection{courbes}
	
	\subsubsection{fonction cosinus}
	
	\psset{unit=1cm}
	\begin{pspicture}(-8.5,-1.2)(8.5,1.4)
	\newrgbcolor{bleu}{0.1 0.05 .5}
	\newrgbcolor{prune}{.6 0 .48}
	\def\pshlabel#1{\footnotesize #1}
	\def\psvlabel#1{\footnotesize #1}
	\psaxes[labelsep=.8mm,linewidth=.75pt,ticksize=-2pt 2pt,trigLabels=true,trigLabelBase=2,dx=\psPiH,xunit=\psPi]{->}(0,0)(-2.7,-1.2)(2.7,1.5)
	\psplot[plotpoints=800,linecolor=bleu,linewidth=1pt]{-8.5}{8.5}{x RadtoDeg cos}
	\psplot[plotpoints=400,linecolor=bleu,linewidth=1.5pt]{-\psPi}{\psPi}{x RadtoDeg cos}
	\psline[linecolor=prune,linewidth=.5pt,linestyle=dashed](-\psPi,-1)(-\psPi,1.2)
	\psline[linecolor=prune,linewidth=.5pt,linestyle=dashed](\psPi,-1)(\psPi,1.2)
	\psline[linecolor=prune,linewidth=.5pt]{<->}(-\psPi,1.2)(\psPi,1.2)
	\uput[u](0,1.2){\prune{$2\pi$}}
	\end{pspicture}
	
	\subsubsection{fonction sinus}
	
	\psset{unit=1cm}
	\begin{pspicture}(-8.5,-1.2)(8.5,1.4)
	\newrgbcolor{bleu}{0.1 0.05 .5}
	\newrgbcolor{prune}{.6 0 .48}
	\def\pshlabel#1{\footnotesize #1}
	\def\psvlabel#1{\footnotesize #1}
	\psaxes[labelsep=.8mm,linewidth=.75pt,ticksize=-2pt 2pt,trigLabels=true,trigLabelBase=2,dx=\psPiH,xunit=\psPi]{->}(0,0)(-2.7,-1.2)(2.7,1.5)
	\psplot[plotpoints=800,linecolor=bleu,linewidth=1pt]{-8.5}{8.5}{x RadtoDeg sin}
	\psplot[plotpoints=400,linecolor=bleu,linewidth=1.5pt]{-\psPi}{\psPi}{x RadtoDeg sin}
	\psline[linecolor=prune,linewidth=.5pt,linestyle=dashed](-\psPi,0)(-\psPi,1.2)
	\psline[linecolor=prune,linewidth=.5pt,linestyle=dashed](\psPi,0)(\psPi,1.2)
	\psline[linecolor=prune,linewidth=.5pt]{<->}(-\psPi,1.2)(\psPi,1.2)
	\uput[u](0,1.2){\prune{$2\pi$}}
	\end{pspicture}
	\section {Fonction $t\mapsto A\cos(\omega t+\varphi)$ et  $t\mapsto A\cos(\omega t+\varphi)$}
	 En physique, de nombreux phénomènes sont liés à la propagation d’onde: le son, la lumière, ...Les grandeurs associées à ces ondes peuvent être mathématisées par des fonctions sinusoïdales du type $t\mapsto A\cos(\omega t+ \varphi)$ et $t\mapsto A\sin(\omega t+ \varphi)$\\
	 On a représenté ci dessous la fonction $t\mapsto3\cos(0.5t+\dfrac\pi2)$
	 	
	 	\psset{xunit=0.5cm,yunit=1cm}
	 	\begin{pspicture*}(-20,-4)(20,4)
	 	\psset{algebraic=true}
	 		\newrgbcolor{bleu}{0.1 0.05 .5}
	 	\psaxes[trigLabels=true,trigLabelBase=2,dx=\psPiH,xunit=\psPi]{->}(0,0)(-10,-4)(10,4)
	 	\psplot[plotpoints=800,linecolor=bleu,linewidth=1.5pt]{-20}{20}{3*cos(0.5*x+\psPi/2)}
	 	
	 	
	 	
	 	\end{pspicture*}
	 \subsection{Amplitude}
	 	\begin{bclogo}[couleur = yellow!30, arrondi = 0.1,logo=\bcbook]{Définition}
	L’amplitude  d’une fonction périodique est sa valeur
	 maximale.\end{bclogo}
 	\begin{bclogo}[couleur = red!25, arrondi = 0.1,logo=\bcbook]{Propriété}
 		L’amplitude des fonctions   $t\mapsto A\cos(\omega t+ \varphi)$ et $t\mapsto A\sin(\omega t+ \varphi)$  est $A$.
	\end{bclogo}
\subsection{Phase}
\begin{bclogo}[couleur = yellow!30, arrondi = 0.1,logo=\bcbook]{Définition}
 $\omega t +\varphi$ est appelé la \textbf{phase instantanée du signal}.\\
 Si $t= 0$, $\varphi$ est appelée \textbf{la phase à l’origine du signal}.\\
 $\omega$ est appelée \textbf{la pulsation du signal}
 \end{bclogo}
\begin{Rem}
	En physique, la phase s’exprime en radians et la pulsation en radians par seconde.
\end{Rem}.
\subsection{Période et fréquence}
\begin{bclogo}[couleur = yellow!30, arrondi = 0.1,logo=\bcbook]{Définition}
\textbf{	La période T} d’une fonction est l’intervalle pour lequel la courbe de la fonction se reproduit à l’identique.\\
La fréquence $f$ est le nombre de périodes par seconde . On a $f=\dfrac1T$
\end{bclogo}
\begin{Rem}
En physique, la période $T$ s’exprime en secondes et la fréquence $f$ en Hertz(Hz)
\end{Rem}.
\begin{bclogo}[couleur = red!25, arrondi = 0.1,logo=\bcbook]{Propriété}
La période T des fonctions $t\mapsto A\cos(\omega t+ \varphi)$ et $t\mapsto A\sin(\omega t+ \varphi)$ est $\dfrac{2\pi}{\omega}=\2\pi\times f$
\end{bclogo}

	\begin{Meth}[Savoir déterminer les valeurs $\omega$, $\varphi$ et $A$ à partir d'un graphique]
	Déterminer la fonction $f\mapsto A\cos(\omega t+\varphi)$ dont la courbe est donnée ci dessous\\

	\psset{xunit=1cm,yunit=1cm}
	\begin{pspicture*}(-10,-4)(10,4)
	\psset{algebraic=true}
		\newrgbcolor{bleu}{0.1 0.05 .5}
	\psaxes[trigLabels=true,trigLabelBase=4,dx=\psPiH,xunit=\psPi]{->}(0,0)(-10,-4)(10,4)
		\psgrid[subgriddiv=2,gridlabels=3pt,gridwidth=0.5pt,griddots=10,subgriddots=10](-10,-4)(10,4)
	\psplot[plotpoints=800,linecolor=bleu,linewidth=1.5pt]{-10}{10}{2*cos(4*x+\psPi/4)}
	
	
	
	\end{pspicture*}
	\end{Meth}	
	\end{document}
