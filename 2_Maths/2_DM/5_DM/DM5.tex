\documentclass[10pt,a4paper]{article}
\usepackage[utf8]{inputenc}
\usepackage[french]{babel}
%\frenchbsetup{StandardLists=true}
\usepackage[T1]{fontenc}
\usepackage{amsmath}
\usepackage{amsthm}
\usepackage{amsfonts}
\usepackage{amssymb}
\usepackage{graphicx}
\usepackage{framed}
\usepackage{fancyhdr}
\usepackage[left=1.3cm,right=1.3cm,top=1.8cm,bottom=1.2cm]{geometry}
\usepackage{array} 
\usepackage{fancyhdr} 
\usepackage{fancybox}
\usepackage{pst-tree}
\usepackage[framed]{ntheorem}
\usepackage{tabularx}
\usepackage{pstricks-add}
\usepackage{eurosym}
\usepackage{pst-tree}
\usepackage[np]{numprint}
\usepackage{pifont}
\usepackage{mathrsfs}
\usepackage{amssymb}
\usepackage{amsthm}
%\usepackage{pgf,tikz}
%\usepackage[tikz]{bclogo}
\usepackage{pgfkeys}
\usepackage{mathrsfs}
%\usepackage{multicol}
%\usetikzlibrary{arrows}


\usepackage{listingsutf8}
\usepackage[latin1]{inputenc}
\usepackage{tikz}
\usetikzlibrary{trees}

\usepackage{colortbl}



\lstset{%
	language=python,%
	basicstyle=\ttfamily,%
}

\rfoot{\small -\thepage-}
\cfoot{}

\def\R{{\mathbb R}}
\def\Q{{\mathbb Q}}
\def\Z{{\mathbb Z}}
\def\D{{\mathbb D}}
\def\N{{\mathbb N}}
\def\C{{\mathbb C}}

\pagestyle{fancy}

\renewcommand{\thesection}{\Roman{section}}
\renewcommand{\thesubsection}{\arabic{subsection}}
\renewcommand{\thesubsubsection}{\alph{subsubsection}}
\renewcommand{\labelitemi}{$\bullet$}
\newcommand{\VE}[1]{\overrightarrow{#1}}
%\renewcommand{\r}{($O$ ; $\vec{i}$ , $\vec{j}$)}
\renewcommand{\arraystretch}{0.7}
\tikzstyle{mybox} = [draw=black, very thick, rectangle, rounded corners, inner sep=20pt, inner ysep=20pt] 
\tikzstyle{fancytitle} =[draw=black, very thick, rectangle, rounded corners, fill=white, text=black] % fill obligé sinon ne recouvre pas boite du dessous
%\usepackage{array,multirow,makecell}
%\setcellgapes{1pt}
\newcommand{\Syst}[2]{\left\{\begin{array}{ccc} #1\\ #2 \end{array}\right.}

\makegapedcells
\newcolumntype{R}[1]{>{\raggedleft\arraybackslash }b{#1}}
\newcolumntype{L}[1]{>{\raggedright\arraybackslash }b{#1}}
\newcolumntype{C}[1]{>{\centering\arraybackslash }b{#1}}
\theoremstyle{break}
\theorembodyfont{\upshape}
\newtheorem{Prop}{Propri\'et\'e}
\newtheorem{Def}{D\'efinition}
\newtheorem{Rem}{Remarque}
\newtheorem{exo}{Exercice}
\newtheorem{Meth}{Methode}
\newtheorem{cpreuve}{Preuve}
\newtheorem{Th}{Théorème}
\newtheorem{Act}{Activité}
\theorembodyfont{\small \sffamily }
\newtheorem{Ex}{Exemple}
\newtheorem{Preu}{Preuve}
\everymath{\displaystyle}
\pagestyle{fancy}
\fancyhead[L]{Florentin Lespinasse ~~~~~~~~~1STI2D4}
\fancyhead[C]{$N^o5$}
\fancyhead[R]{2019/2020}
\rfoot{\small -\thepage-}
\cfoot{}


\renewcommand{\arraystretch}{1.5}
\setlength{\tabcolsep}{0.5cm}

\begin{document}

\begin{center}
   \shadowbox{\begin{large}
      \textcolor{black}{DM de Mathématiques}
   \end{large}}
\end{center}


\begin{exo}\\
	Deux machines A et B d’une usine fabriquent des puces électroniques. A en produit 40\%; 5\% des puces fabriquées par A sont défectueuses, 2\% des puces produites par B sont défectueuses. On prélève une puce au hasard. Quelle est la probabilité qu’elle ait été produite par A sachant qu’elle est défectueuse?

\begin{table}[ht]
        \centering
        \begin{tabular}{|c|c|c|c|}
            \hline
							& A 	& B 	& Total \\
           \hline
         				Deffectueux & \cellcolor{gray} 0.020 & 0.012 & 0.032  \\
           \hline
            $\overline{Defectueux}$ & 0.380 & 0.588 & 0.968  \\
           \hline
            Total & 0.4 & 0.6 & 1  \\
           \hline

        \end{tabular}
    \end{table}

La probabilité que la pièce choisie provienne de l'usine A et qu'elle soit déffectueuse est de: 
$P(A \cap D) = \frac{0.02}{1}$


\end{exo}
\begin{exo}\\
	On se donne un arbre de probabilité pondéré :


\begin{enumerate}
	\item Compléter l'arbre avec les valeurs manquantes.

% Set the overall layout of the tree
\tikzstyle{level 1}=[level distance=3.5cm, sibling distance=3.5cm]
\tikzstyle{level 2}=[level distance=3.5cm, sibling distance=2cm]
% Define styles for bags and leafs
\tikzstyle{bag} = [text width=4em, text centered]
\tikzstyle{end} = [circle, minimum width=3pt,fill, inner sep=0pt]

% The sloped option gives rotated edge labels. Personally
% I find sloped labels a bit difficult to read. Remove the sloped options
% to get horizontal labels. 
\begin{tikzpicture}[grow=right, sloped]
\node[bag] { }
    child {
        node[bag] {C}
            child {
                node[end, label=right:
                    {$\overline{I}$}] {}
                edge from parent
                node[above] {}
                node[below] {$0.3$}
            }
            child {
                node[end, label=right:
                    {I}] {}
                edge from parent
                node[above] {$0.7$}
                node[below] {}
            }
            edge from parent
            node[above] {}
            node[below] {$0.3$}
    }
    child {
        node[bag] {B}
        child {
                node[end, label=right:
                    {$\overline{I}$}] {}
                edge from parent
                node[above] {}
                node[below] {$0.2$}
            }
            child {
                node[end, label=right:
                    {I}] {}
                edge from parent
                node[above] {$0.8$}
                node[below] {}
            }
        edge from parent
            node[above] {$0.1$}
            node[below] {}
    }
	child {
		node[bag] {A}
		child {
                node[end, label=right:
                    {$\overline{I}$}] {}
                edge from parent
                node[above] {}
                node[below] {$0.9$} 
            }
            child {
                node[end, label=right:
                    {I}] {}
                edge from parent
                node[above] {$0.1$}
                node[below] {}
            }
        edge from parent
            node[above] {$0.6$}
            node[below] {}
}

\end{tikzpicture}

	\item Ici , $p(A),~p(B),~p(C)$ sont les 3 premières branches de l'arbre. 	
		$p(A) = \frac{0.6}{1}$ ~~~~~~~ $p(B) = \frac{0.1}{1}$ ~~~~~~~ $p(C) = \frac{0.3}{1}$
	\item On peut lire que \\
		$p_A(I) = 0.1$ ~~~~~~~ $p_A(\overline{I}) = 0.9$ ~~~~~~~ $p_B(I) = 0.8$ ~~~~~~~ $p_C(I) = 0.7$

	\item On en déduit donc que \\
		$p(A\cap I) = p(A)*p_A(I) = 0.6*0.1 = 0.06$	\\
		$p(B\cap I) = p(B)*p_B(I) = 0.1*0.8 = 0.08$ \\
		$p(C\cap I) = p(C)*p_C(I) = 0.3*0.7 = 0.21$ \\ 
		$p(I)$ est égal à la somme des probabilitées de l'obtenir. Ici, $p(A\cap I)$,$p(B\cap I)$ et $p(C\cap I)$\par  
	Donc $0.06+0.08+0.21 = 0.29+0.06 = 0.35$
	\item Pour calculer $p_I(A)$, on utilise la formule:\\
	$p_I(A) = \frac{p(A\cap I)}{p(I)}$ \\
	On connaît déjà $p(A\cap I)$ qui vaut $0.06$ et $p(I)$ qui vaut $0.35$ on applique la formule:\\
	$p_I(A) = \frac{0.06}{0.35} \approx 0.1714 $ \\

\end{enumerate}
	
\end{exo}
\begin{exo}
L'iode 131 est très utilisé à petites doses dans l'imagerie médicale, par exemple la scintigraphie.\\
On étudie l'évolution au cours du temps d'un échantillon de noyaux d'iode 131 comportant $v_0 = 10^6$ noyaux à l'instant $t = 0$. On note $(v_n)$ le nombre de noyaux au bout de $n$ jours. Statistiquement le nombre de noyaux d'iode 131 diminue chaque jour de 8,3\%.

	\begin{enumerate}
		
		\item On cherche $v_{1}$ et $v_{2}$\\
				On sait que il y a une baisse de 8.3\% noyaux par jour.
				\ding{216} $ 1 - \frac{8}{100} = 1-0.083 = 0.917$

				$v_1 = v_0*0.917$~~~~
				$v_1 = 10^6*0.917$~~~~
				$v_1 = 917000$ \par
				
				$v_2 = v_1*0.917$~~~~
				$v_2 = 917000*0.917$~~~~
				$v_2 = 840889$\par
		\item Exprimer $v_{n+1}$ en fonction de $v_{n}$. En déduire la nature de la suite $\left(u_{n}\right)$. \par
			$v_n$ est une suite géométrique de raison $r = 0.917 $ et de premier terme $v_0 = 10"$ car on multiplie toujours par le résultat précedent. \par
			Donc \forall~$_n$~ \in~\mathbb{N}: ~~
			$v_n_+_1 = v_n*0.917$\par
			
		

		\item  Déterminer le nombre de noyaux d'iode 131 présents au bout de 5 jours. 
				Au bout de 5 jours, donc $v_4$ il y aura 707094.310321 noyaux d'iode 131.
	
		\item On considère la fonction Python ci-dessous:\\
			\begin{lstlisting}
		def iode():
		  n=0
		  u=10**6
		  while u>10**6/2:
		    n=n+1
		    u=0.917*u
		  return(u)
		\end{lstlisting}
			\begin{enumerate}
			\item À quoi correspond la valeur $n$ retournée par cette fonction ? \par
				La valeur $n$ correspond au nombre de jours écoulé. sachant que le $1^{er}$ jour est noté $v_0$ donc pour avoir le nombre de jour correct, on ajoute 1 à la valeure $n$.
			 
			\item Si on exécute cette fonction, quelle valeur obtient-on ?\par 
				On obtient la valeur $v_8 = 499982.3636883308$ \\

			\item Déterminer à partir de combien de jours la population de noyaux aura diminué au moins de moitié.
			Cette durée s'appelle la demi-vie de l'iode 131.

			\`A partir de $n = 8$ on dépasse la demi-vie de l'iode 131. Donc cela ignifie que au bout de 9 jours la population de noyaux aura diminué au moins de moitiée. \\
			
			\item Pour le Césium 137, le nombre de noyaux diminue chaque année de 2,3\,\%.
			
			Quelles modifications faut-il apporter à l'algorithme précédent pour trouver la demi-vie du césium 137 sachant que la population au départ est de $10^8$~noyaux ?

			Il faut changer le poucentage de baisse des noyaux, \ding{216} $0.23 = 1-0.023 = 0.977$ et la valeur de départ de $u$.

			 \begin{lstlisting}
        def iode137():
          n=0
          u=10**8
          while u>10**8/2:
            n=n+1
            u=0.977*u
          print(u)
        \end{lstlisting}
	Au bout de 30 jours, le césium 137 aura ateint sa demi-vie.


		\end{enumerate} 
	\end{enumerate}
	
	

	
	
\end{exo}

	


\end{document}
  
