\documentclass[10pt,a4paper]{article}
\usepackage[utf8]{inputenc}
\usepackage[french]{babel}
\frenchbsetup{StandardLists=true}
\usepackage[T1]{fontenc}
\usepackage{amsmath}
\usepackage{amsthm}
\usepackage{amsfonts}
\usepackage{amssymb}
\usepackage{graphicx}
\usepackage{framed}
\usepackage{fancyhdr}
\usepackage[left=1.3cm,right=1.3cm,top=1.8cm,bottom=1.2cm]{geometry}
\usepackage{array} 
\usepackage{fancyhdr} 
\usepackage{fancybox}
\usepackage{pst-tree}
\usepackage[framed]{ntheorem}
\usepackage{tabularx}
\usepackage{pstricks-add}
\usepackage{eurosym}
%\usepackage{pst-tree}
\usepackage[np]{numprint}
\usepackage{pifont}
\usepackage{mathrsfs}
\usepackage{amssymb}
\usepackage{amsthm}
\usepackage{pgf,tikz}
\usepackage[tikz]{bclogo}
\usepackage{pgfkeys}
\usepackage{mathrsfs}
\usepackage{multicol}
\usetikzlibrary{arrows}
\usepackage{listingsutf8}
\lstset{%
	language=python,%
	basicstyle=\ttfamily,%
}

\rfoot{\small -\thepage-}
\cfoot{}

\def\R{{\mathbb R}}
\def\Q{{\mathbb Q}}
\def\Z{{\mathbb Z}}
\def\D{{\mathbb D}}
\def\N{{\mathbb N}}
\def\C{{\mathbb C}}

\pagestyle{fancy}

\renewcommand{\thesection}{\Roman{section}}
\renewcommand{\thesubsection}{\arabic{subsection}}
\renewcommand{\thesubsubsection}{\alph{subsubsection}}
\renewcommand{\labelitemi}{$\bullet$}
\newcommand{\VE}[1]{\overrightarrow{#1}}
%\renewcommand{\r}{($O$ ; $\vec{i}$ , $\vec{j}$)}
\renewcommand{\arraystretch}{0.7}
\tikzstyle{mybox} = [draw=black, very thick, rectangle, rounded corners, inner sep=20pt, inner ysep=20pt] 
\tikzstyle{fancytitle} =[draw=black, very thick, rectangle, rounded corners, fill=white, text=black] % fill obligé sinon ne recouvre pas boite du dessous
\usepackage{array,multirow,makecell}
\setcellgapes{1pt}
\newcommand{\Syst}[2]{\left\{\begin{array}{ccc} #1\\ #2 \end{array}\right.}

\makegapedcells
\newcolumntype{R}[1]{>{\raggedleft\arraybackslash }b{#1}}
\newcolumntype{L}[1]{>{\raggedright\arraybackslash }b{#1}}
\newcolumntype{C}[1]{>{\centering\arraybackslash }b{#1}}
\theoremstyle{break}
\theorembodyfont{\upshape}
\newtheorem{Prop}{Propri\'et\'e}
\newtheorem{Def}{D\'efinition}
\newtheorem{Rem}{Remarque}
\newtheorem{exo}{Exercice}
\newtheorem{Meth}{Methode}
\newtheorem{cpreuve}{Preuve}
\newtheorem{Th}{Théorème}
\newtheorem{Act}{Activité}
\theorembodyfont{\small \sffamily }
\newtheorem{Ex}{Exemple}
\newtheorem{Preu}{Preuve}
\everymath{\displaystyle}
\pagestyle{fancy}
\fancyhead[L]{1STI2D}
\fancyhead[C]{ }
\fancyhead[R]{2019/2020}
\rfoot{\small -\thepage-}
\cfoot{}

\begin{document}

\begin{center}
   \shadowbox{\begin{large}
      \textcolor{black}{DM de Mathématiques}
   \end{large}}
\end{center}


\begin{exo}
	Deux machines A et B d’une usine fabriquent des puces électroniques. A en produit 40\%; 5\% des puces fabriquées par A sont défectueuses, 2\% des puces produites par B sont défectueuses. On prélève une puce au hasard.Quelle est la probabilité qu’elle ait été produite par A sachant qu’elle est défectueuse?
\end{exo}
\begin{exo}
	On se donne un arbre de probabilité pondéré :
	\psset{nodesep=2mm,levelsep=40mm,treesep=15mm}
	\begin{center}
		\pstree[treemode=R]{\TR{}}
		{
			\pstree{\TR{$A$}\taput{$0,6$}}
			{
				\TR{$I$} \taput{$\cdots$}
				\TR{$\overline{I}$} \tbput{$0,9$}
			}
			\pstree{\TR{$B$}\taput{$0,1$}}
			{
				\TR{$I$} \taput{$0,8$}
				\TR{$\overline{I}$} \tbput{$\cdots$}
			}
			\pstree{\TR{$C$} \tbput{$\cdots$}}
			{
				\TR{$I$} \taput{$0,7$}
				\TR{$\overline{I}$} \tbput{$0,3$}
			}
		}
	\end{center}
\begin{enumerate}
	\item Compléter l'arbre avec les valeurs manquantes.
	\item Donner $p(A)$, $p(B)$ et $p(C)$.
	\item Lire $p_A(I)$, $p_A(I)$, $p_B(I)$ et $p_C(I)$.
	\item En déduire $p(A\cap I)$, $p(B\cap I)$ et $p(C\cap I)$ puis calculer $p(I)$.
	\item Calculer $p_I(A)$.
\end{enumerate}
	
\end{exo}
\begin{exo}
L'iode 131 est très utilisé à petites doses dans l'imagerie médicale, par exemple la scintigraphie.\\
On étudie l'évolution au cours du temps d'un échantillon de noyaux d'iode 131 comportant $v_0 = 10^6$ noyaux à l'instant $t = 0$. On note $(v_n)$ le nombre de noyaux au bout de $n$ jours. Statistiquement le nombre de noyaux d'iode 131 diminue chaque jour de 8,3\%.

	\begin{enumerate}
		\item Calculer $v_{1}$ puis $v_{2}$. 
		\item Exprimer $v_{n+1}$ en fonction de $v_{n}$. En déduire la nature de la suite $\left(u_{n}\right)$. 
		\item  Déterminer le nombre de noyaux d'iode 131 présents au bout de 5 jours. 
	
		\item On considère la fonction Python ci-dessous:\\
			\begin{lstlisting}
		def iode():
		  n=0
		  u=10**6
		  while u>10**6/2:
		    n=n+1
		    u=0.917*u
		  return(u)
		\end{lstlisting}
			\begin{enumerate}
			\item À quoi correspond la valeur $n$ retournée par cette fonction ? 
			\item Si on exécute cette fonction, quelle valeur obtient-on ? 
			\item Déterminer à partir de combien de jours la population de noyaux aura diminué au moins de moitié.
			
			Cette durée s'appelle la demi-vie de l'iode 131. 
			\item Pour le Césium 137, le nombre de noyaux diminue chaque année de 2,3\,\%.
			
			Quelles modifications faut-il apporter à l'algorithme précédent pour trouver la demi-vie du césium 137 sachant que la population au départ est de $10^8$~noyaux ?
		\end{enumerate} 
	\end{enumerate}
	
	

	
	
\end{exo}

	


\end{document}
  
