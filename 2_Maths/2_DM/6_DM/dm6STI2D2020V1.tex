\documentclass[12pt,a4paper]{article}
\usepackage[utf8]{inputenc}
\usepackage[french]{babel}
\frenchbsetup{StandardLists=true}
\usepackage[T1]{fontenc}



\usepackage{amsmath}
\usepackage{amsthm}
\usepackage{amsfonts}
\usepackage{amssymb}
\usepackage{graphicx}
\usepackage{framed}
\usepackage{fancyhdr}
\usepackage[left=1.3cm,right=1.3cm,top=1.8cm,bottom=1.2cm]{geometry}
\usepackage{array} 
\usepackage{fancyhdr} 
\usepackage{fancybox}
\usepackage{pst-tree}
\usepackage[framed]{ntheorem}
\usepackage{tabularx}
\usepackage{pstricks-add}
\usepackage{eurosym}
%\usepackage{pst-tree}
\usepackage[np]{numprint}
\usepackage{pifont}
\usepackage{mathrsfs}
\usepackage{amssymb}
\usepackage{amsthm}
\usepackage{pgf,tikz}
\usepackage[tikz]{bclogo}
\usepackage{pgfkeys}
\usepackage{mathrsfs}
\usepackage{multicol}
\usetikzlibrary{arrows}
\usepackage{listingsutf8}
\lstset{%
	language=python,%
	basicstyle=\ttfamily,%
}

\rfoot{\small -\thepage-}
\cfoot{}

\def\R{{\mathbb R}}
\def\Q{{\mathbb Q}}
\def\Z{{\mathbb Z}}
\def\D{{\mathbb D}}
\def\N{{\mathbb N}}
\def\C{{\mathbb C}}

\pagestyle{fancy}

\renewcommand{\thesection}{\Roman{section}}
\renewcommand{\thesubsection}{\arabic{subsection}}
\renewcommand{\thesubsubsection}{\alph{subsubsection}}
\renewcommand{\labelitemi}{$\bullet$}
\newcommand{\VE}[1]{\overrightarrow{#1}}
%\renewcommand{\r}{($O$ ; $\vec{i}$ , $\vec{j}$)}
\renewcommand{\arraystretch}{0.7}
\tikzstyle{mybox} = [draw=black, very thick, rectangle, rounded corners, inner sep=20pt, inner ysep=20pt] 
\tikzstyle{fancytitle} =[draw=black, very thick, rectangle, rounded corners, fill=white, text=black] % fill obligé sinon ne recouvre pas boite du dessous
\usepackage{array,multirow,makecell}
\setcellgapes{1pt}
\newcommand{\Syst}[2]{\left\{\begin{array}{ccc} #1\\ #2 \end{array}\right.}

\makegapedcells
\newcolumntype{R}[1]{>{\raggedleft\arraybackslash }b{#1}}
\newcolumntype{L}[1]{>{\raggedright\arraybackslash }b{#1}}
\newcolumntype{C}[1]{>{\centering\arraybackslash }b{#1}}
\theoremstyle{break}
\theorembodyfont{\upshape}
\newtheorem{Prop}{Propri\'et\'e}
\newtheorem{Def}{D\'efinition}
\newtheorem{Rem}{Remarque}
\newtheorem{exo}{Exercice}
\newtheorem{Meth}{Methode}
\newtheorem{cpreuve}{Preuve}
\newtheorem{Th}{Théorème}
\newtheorem{Act}{Activité}
\theorembodyfont{\small \sffamily }
\newtheorem{Ex}{Exemple}
\newtheorem{Preu}{Preuve}
\everymath{\displaystyle}
\pagestyle{fancy}
\fancyhead[L]{Lespinasse Florentin}
\fancyhead[C]{1STI2D4}
\fancyhead[R]{le 10/02/2020}
\rfoot{\small -\thepage-}
\cfoot{}
 \usepackage{comment} 

\usepackage{colortbl}
\usepackage{xcolor, soul}



\begin{document}

\begin{center}
   \shadowbox{\begin{large}
      \textcolor{black}{DM de Mathématiques}
   \end{large}}
\end{center}
\textbf{Exercice 1}\\
La courbe $\mathscr{C}$ est la représentation graphique d’une fonction $f$ définie et dérivable sur $\R$, dans un repère orthogonal.
%\begin{multicols}{2}
\begin{enumerate}
	\item On peut lire graphiquement :
	\begin{enumerate}
		\item $f(0) = 1$ et $f'(0) = -3$;
		\item $f(-1) = 3$ et $f'(-1) = 0$;
		\item $f(2) = 3$ et $f'(2) = 9$;
		\item L’équation de la tangente au point d’abscisse $-1$\par
		L'équation de la tengente,\par
		$y = f'(-1)(x-(-1))+f(-1)$\par
		$y = 0(x+1)+3$\par
		$y = 0+3$\par
		Donc la tengente au point d'abscisse $-1$ a pour équation $y = 3$

		\item L’équation de la tangente au point d’abscisse $0$\par
		$y = f'(0)(x-0)+f(0)$\par
		$y = -3(x)+1$\par
		$y = -3x+1$\par
		Donc la tengente au point d'abscisse $0$ a pour équation $y = -3x+1$
			\end{enumerate}
		\item La droite $T$ tangente à la courbe $\mathscr{C}$ au point d’abscisse $-2$ et d’ordonnée $-1$ passe par le point A de coordonnées  $(1; 26)$
		\begin{enumerate}
			\item Déterminer par le calcul une équation de $T$.
		\begin{enumerate}
			\item On calcul $m$ le coéficient directeur:\par
			$y = \frac{y_b-y_a}{x_b-x_a}\Leftrightarrow$
			$y = \frac{-1-26}{-2-1}$ $\Leftrightarrow$
			$y = \frac{-27}{-3}\Leftrightarrow$
			$y = \frac{27}{3}\Leftrightarrow$
			$y = 9$\par
			\item On calcul $p$ l'ordonnée à l'origine:\par
			$y = 9x +p$ $\Leftrightarrow$
			$-1 = 9\times (-2) +p$ $\Leftrightarrow$
			$-1 = -18+p$ $\Leftrightarrow$
			$p = -1+18$ $\Leftrightarrow$
			$p = 17$\par
			Donc l'équation de la tengente $T$ à la courbe $\mathscr{C}$ est: \par
			$y = 9x+17$


		\end{enumerate}
			\item En déduire $f'(-2)$\par
			$f(x) = 9x+17$$\Leftrightarrow$
			$f'(x) = 9$$\Leftrightarrow$
			$f'(-2) = 9$ 
			
		\end{enumerate}
\end{enumerate}


\begin{comment}
%\vfill\columnbreak
	\psset{xunit=1cm,yunit=0.3cm,algebraic=true}
\def\xmin {-3}
\def\xmax {13}
\def\ymin {-7}
\def\ymax {30}

	\begin{pspicture*}(\xmin,\ymin)(\xmax,\ymax)
	\psgrid[subgriddiv=2,gridlabels=3pt,gridwidth=0.5pt,griddots=10,subgriddots=10](\xmin,\ymin)(\xmax,\ymax)
	\psaxes[Dy=5]{->}(0,0)(\xmin,\ymin)(\xmax,\ymax)
		\psplot[linewidth=1pt]{-3}{3}{x^3-3*x+1}
		\psplot[linewidth=1pt]{-3}{3}{9*x+17}
		\psplot[linewidth=1pt]{-3}{3}{9}
		\psline[arrowscale=2,linestyle=dashed]{<->}(-1,4)(1,-2)
		\psline[arrowscale=2,linestyle=dashed]{<->}(-2.5,3)(0.5,3)
			\psline[arrowscale=2,linestyle=dashed]{<->}(1,-6)(3,12)
			\psdot[dotstyle=*](-1,3)
				\psdot[dotstyle=*](0,1)
					\psdot[dotstyle=*](2,3)
						\psdot[dotstyle=*](1,26)
						\psdot[dotstyle=*](-2,9)
	\end{pspicture*}

%\end{multicols}
\end{comment}



\newpage
\textbf{Exercice 2}\\
	Un propriétaire propose à la location deux appartements notés T1 et T2. Le loyer mensuel net pour chacun des appartements se compose de trois parties :
	\begin{itemize}
		\item le loyer mensuel hors charges (loyer HC) ;
		\item les charges ;
		\item la taxe locative sur le ramassage des ordures ménagères.
	\end{itemize}
	
	Le tableau ci-après contient les informations en euros relatives à la location de ces deux appartements pour le mois de janvier.\\
	\begin{center}
	\renewcommand{\arraystretch}{2}
	\begin{tabular}{|c|c|c|c|c|}
		\hline
		&loyer HC& charges& taxe locative &loyer mensuel net\\
		\hline
		T1& 360 &\cellcolor{blue!70}65&\cellcolor{red!45}36&461\\ 
		\hline
		T2 &\cellcolor{green!85}455&\cellcolor{yellow}72,4&54,60&\cellcolor{green!25}582\\
		\hline
		total&\cellcolor{magenta!90}815&\cellcolor{cyan}137,4&\cellcolor{red}90,6& 1043\\
		\hline
	\end{tabular}\\
	\end{center}
	La taxe locative représente 10 \% du loyer HC de l’appartement T1 et 12 \% de l’appartement T2.
	\begin{enumerate}
		\item 
		\begin{enumerate}
			\item Calculer le montant du loyer HC de l’appartement T2.(On écrira les détails des calculs)\par
			On cherche la taxe locative de l'appartement T2, (54,60) ce résultat est égale à 12\% du loyer HC.\par
			Donc on résout un produit en croix, $x = \frac{54,60\times 100}{12}$ $\Leftrightarrow$ $x = 455$ (case \colorbox{green!85}{verte})\par

			\item Compléter le tableau en indiquant les opérations effectuées sur votre copie.\par
			Les charges du T1 sont égales a la soustraction du loyer mensuel net par l'addition du loyer HC et de la taxe locative. $461-360+36 = 65$ (case \colorbox{blue!50}{bleu})\par
			La taxe locative est égale à 10\% du loyer donc $x = \frac {36\times 10}{100} \Leftrightarrow x = 36$(case \colorbox{red!45}{saumon})\par
			Pour trouver le loyer mensuel net du T2, on soustrait le loyer total et le loyer mensuel net du T1. $1043-461 = 582$ (case \colorbox{green!25}{pistache})\par
			Les charges du T2 sont égales à la soustraction du loyer mensuel net et l'addition du loyer HC et de la taxe locative. $582-455+54,60 = 72,40$(case \colorbox{yellow}{jaune})\par
			Le loyer HC total est égale à la somme du loyer HC du T1 et du T2. $360+455 = 815$(case \colorbox{magenta!90}{magenta})\par
			Les charges totales sont égales à la somme des charges du T1 et du T2. $65+72,4 = 137,4$(case \colorbox{cyan}{cyan})\par
			La taxe locative totale est égale à la somme des taxes du T1 et du T2. $36+54,6 = 90,6$ (case \colorbox{red}{rouge})\par

			\item Pour l’appartement T1, calculer la proportion des charges par rapport au loyer mensuel
			net, exprimée en pourcentage arrondi à 0,1 \% près.\par
			La propotion de charges de l'appartement T1 est de $\frac{65}{461}\times 100 $ $\Leftrightarrow$ $ 0,141 \times 100 $ $\Leftrightarrow$ $ 14,1$\par
			Donc les charges représentent $14,1\%$ des charges mensuel.
			
		\end{enumerate}
		\item Si un locataire de l’appartement T2 reçoit une aide de 260 \euro par mois, quelle est, en pourcentage arrondi à 0,01 \% près, la part de cette aide par rapport au loyer HC ?\par
		La part de cette aide est de $\frac{260}{455} \times 100 $ $\Leftrightarrow$ $ 0,5714 \times 100 $ $\Leftrightarrow$ $ 57,14\%$\par
		L'aide que le locataire reçoit est de $57,14\%$ du prix qu'il paye pour son loyer HC.
		
	\end{enumerate}
\newpage
\textbf{Exercice 3}\\
		\textsf{\small{\textsc{\textbf{Partie a}}}}\\
		On considère la suite $\left(u_{n}\right)$ définie par $u_{0} = 250$ et pour tout entier naturel $n$, $ u_{n+1} = 0,72u_{n} +420$.
	
	\begin{enumerate}
		\item Calculer $u_2$.\par
		On cherche $u_1$ pour pouvoir trouver $u_2$.\par
		$u_1 = 0.72\times 250+420$ $\Leftrightarrow$ $u_1 = 180+420$ $\Leftrightarrow$ $u_1 = 600$\par
		$u_2 = 0.72\times 600+420$ $\Leftrightarrow$ $u_2 = 432+420$ $\Leftrightarrow$ $u_2 = 852$
		
		\item Soit $\left(v_{n}\right)$  la suite définie pour tout entier naturel $n$ par $v_{n} = u_{n} - 1500$. 
		\begin{enumerate}
			\item Démontrer que la suite $\left(v_{n}\right)$ est une suite géométrique dont on précisera le premier terme et la raison. 
			La suite $\left(v_{n}\right)$ est une suite géométrique car:\par
			On fait une multiplication toujours d'un même terme.\par
			$v_n = u_n -1500 $ $\Leftrightarrow$ $u_n = (0,72u_{n} +420) - 1500$ \par
			$v_n = 0,72u_n-1080$\par
			$v_n = u_n - 1500 $  $\Leftrightarrow$ $u_n = v_n + 1500$\par
			$v_{n+1} = 0,72(v_n+1500)-1080$  $\Leftrightarrow$ $v_{n+1} = 0,72v_n +0,72\times 1500 - 1080$ $\Leftrightarrow$ $v_{n+1} = 0,72v_n$ \par
			$v_0 = 250 - 1500$ $\Leftrightarrow$ $v_0 = -1250$\par
			Le premier terme de la suite géométrique $\left(v_{n}\right)$ est $v_0 = -1250$ et de raison $q = 0,72$\par
			\item Exprimer $v_{n}$ en fonction de $n$.\par
			$v_n = 0,72^n \times -1250$\par
			
			\item En déduire que, pour tout nombre entier naturel $n$, $u_{n} = 1500 -1250\times  0,72^n$.\par
			$v_n = u_n - 1500 $  $\Leftrightarrow$ $u_n = v_n + 1500 $ $\Leftrightarrow$ $ u_n = 1500 (-1250 \times 0,72^n) $ $\Leftrightarrow$ $ u_n = 1500 -1250 \times 0,72^n $  \par
			Donc $\forall n$ \in~$\mathbb{N}$, $ u_n = 1500 -1250 \times 0,72^n $
		\end{enumerate} 
	
	\end{enumerate}
\newpage
		\textsf {\textbf{\textsc{Partie b}}}\\
	Une municipalité a décidé de proposer un abonnement mensuel à un service de location de vélos.
	
	Au mois de janvier 2018, 250 personnes se sont abonnées à ce service.
	
	Une étude statistique a permis de modéliser l'évolution du nombre d'abonnements pour les prochains mois à l'aide de la suite $\left(u_{n}\right)$ définie dans la partie A.
	
	\begin{enumerate}
		\item On considère l'algorithme suivant :
		\begin{center}
			\begin{tabular}{|l|}
				\hline
					\begin{minipage}{.3\linewidth}
					
					$U \gets \np{250}$
					
					$N \leftarrow  0$
					
					Tant que  $U \leqslant \np{1435}$
					
					\begin{itemize}
						\item[] $U \leftarrow 0,72 \times U + 420$
						\item[] $N \leftarrow N+1$
					\end{itemize}
					Fin Tant que
				\end{minipage}\\
			\hline	
			\end{tabular}
		
				
		\end{center}
		\begin{enumerate}
			\item Donner une interprétation de la valeur $N=9$ obtenue à la fin de l'exécution de cet algorithme.\par
			$N$ indique le nombre de mois, ici $N = 9$ donc le nombre de mois pour que plus de 1435 abonnements soient achetés est de 9 mois.
		\item Écrire cet algorithme en Python.
 \begin{lstlisting}
          n=0
          u=250
          while u<=1435:
            n=n+1
            u=0.72*u+420
          print(u)
        \end{lstlisting}

		\end{enumerate}
	
		\item Selon ce modèle, donner une estimation du nombre d'abonnés au bout de 12 mois.\par
		Au bout de 12 mois, donc $N = 12$, d'après la modélisation, il y aura environ 1476 abonnés.
		\item Est-il possible d'envisager nombre d'abonnés supérieur à \np{2000} ?\par
		Non il est impossible d'envisager un nombre d'abonnés supérieur a \np{2000}, 
		car plus $N$ vas tendre vers de grands nombres,
		plus la quantitée d'abonnés vas tendre ves \np{1500} sans jamais les atteindres.
	\end{enumerate}

\end{document}
  
