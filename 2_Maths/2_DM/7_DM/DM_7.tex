%%%%%%
%BASE
%%%
\documentclass[12pt,a4paper]{article}
\usepackage[utf8]{inputenc}
\usepackage[T1]{fontenc}
\usepackage[francais]{babel}

%%%%%
%MATHS
%%%
\usepackage{amsmath}
\usepackage{array}

%%%%%
%COULEURS
%%%
\usepackage{xcolor}
\usepackage{color}

%%%%%
%PUCES
%%%
\usepackage{pifont}

\everymath{\displaystyle}
\usepackage{hyperref}
\setlength{\parindent}{10px}


%%%%%
%Haut de Page
%%%
\usepackage{fancybox}
\usepackage{fancyhdr}
\usepackage[left=1.3cm,right=1.2cm,top=2cm,bottom=1.5cm]{geometry}
\pagestyle{fancy}
\fancyhead[L]{Lespinasse Florentin}
\fancyhead[C]{}
\fancyhead[R]{1STI2D4~~~~~~~~13 Mars 2020}

\begin{document}
\begin{center}
        \shadowbox{\begin{large}
                \textcolor{black}{DM 7}
        \end{large}}
    \end{center}
    \vspace{0.5 cm}

\begin{itemize}
	\item \textbf{PARTIE A}
		\begin{enumerate}
			\item $f'(x)=3ax^2+2bx+c$
			\item $f(0):a(o)^3+b(0)^2+c(0)+d=0$\par
				  $f(0):d=0$ Donc la courbe passe par (0;0)\par
				  $f'(x):3ax^3+2bx^2+c=-0.6$
				  $f'(0):3a(0)^2+2b(0)+c=-0.6$\par
				  Donc $f'(0):c=-0.6$\par
				  Donc $c=-0.6$ et $d=0$. d est l'ordonnée à l'origine et passe par le point O(0;0).
			\item La courbe doit passer par le point A(6;3.6)\par
				  $f(6):a\times(6)^3+b\times(6)^2+(-0.6)\times(6)=3.6$\par  
				  $f(6):216a+36b-3.6=3.6$\par
				  $f(6):216a+36b=7.2$\par
				  Et la tengente de A,\par
				  $f'(A)=0 \Leftrightarrow 3\times a \times 6^2+2\times b \times -0.6 = 0$\par
				  $f'(A): 108a+12b-0.6=0$\par
				  $f'(A): 108a+12b=0.6$\par
				System:
	$				\left \{
					\begin{array}{rcl}
					216a+36b-3.6&=&3.6 \\
					108a+12b&=&0.6
					\end{array}
					\right.
\Leftrightarrow
					\left \{
                    \begin{array}{rcl}
                    216a+36b&=&7.2 \\
                    108a+12b&=&0.6
                    \end{array}
                    \right.\\
\Leftrightarrow	On~divise~par~3
					\left \{
                    \begin{array}{rcl}
                    72a+12b&=&2.4 \\
                    108a+12b&=&0.6
                    \end{array}
                    \right.

			\item   \left \{
                    \begin{array}{rcl}
                    72a+12b&=&2.4 \\
                    108a+12b&=&0.6
                    \end{array}
                    \right.
\Leftrightarrow
                    \left \{
                    \begin{array}{rcl}
                    72a+12b&=&2.4 \\
                    108a-72a&=&-1.8
                    \end{array}
                    \right.

\Leftrightarrow
                    \left \{
                    \begin{array}{rcl}
                    72a+12b&=&2.4 \\
                    108a+12b&=0.6&
                    \end{array}
                    \right.
\Leftrightarrow
                    \left \{
                    \begin{array}{rcl}
                    72a+12b&=&2.4 \\
                    36a&=&-1.8
                    \end{array}
                    \right.

\Leftrightarrow
                    \left \{
                    \begin{array}{rcl}
                    72a+12b&=&2.4 \\
                    a&=&-0.05
                    \end{array}
                    \right.
\Leftrightarrow
                    \left \{
                    \begin{array}{rcl}
                    72\times(-0.05)+12b&=&2.4 \\
                    a&=&-0.05
                    \end{array}
                    \right.
\Leftrightarrow
                    \left \{
                    \begin{array}{rcl}
                    -3.6+12b&=&2.4 \\
                    a&=&-0.05
                    \end{array}
                    \right.

\Leftrightarrow
                    \left \{
                    \begin{array}{rcl}
                    12b&=&2.4+3.6 \\
                    a&=&-0.05
                    \end{array}
                    \right.
\Leftrightarrow
                    \left \{
                    \begin{array}{rcl}
                    12b&=&6 \\
                    a&=&-0.05
                    \end{array}
                    \right.
\Leftrightarrow
                    \left \{
                    \begin{array}{rcl}
                    b&=&\frac{6}{12} \\
                    a&=&-0.05
                    \end{array}
                    \right.
\Leftrightarrow
                    \left \{
                    \begin{array}{rcl}
                    b&=&0.5 \\
                    a&=&-0.05
                    \end{array}
                    \right.

Donc $f(x)=-0.05x^3+0.5x^2-0.6+0$
 
		\end{enumerate}
	\item \textbf{PARTIE B}
		\begin{enumerate}
			\item $f(4)=-0.05\times4^3+0.5\times4^2-0.6\times4$\par
				  $f(4)=-3.2+8-2.4$\par
				  $f(4)=2.4$\par
				Donc $f(4)=2.4$\par
			\item $f'(4)=3\times(-0.05)\times4^2+0.5\times4-0.6$\par
				  $f'(4)=-0.15\times16+1\times4-0.6$\par
				  $f'(4)=-2.4+4-0.6$\par
				  $f'(4)=1$\par
				Donc $f'(4)=1$\par
		\end{enumerate}

\newpage
	\item \textbf{PARTIE C}
		\begin{enumerate}
			\item \dfrac{Yb-Ye}{Xb-Xe} = \dfrac{4.4-2.4}{6-4}=\dfrac{2}{2}=1\\

				Donc la tengente du point E est égale au coeficient directeur de (BE)\\
				$(BE)=1x+p$\par
					$6+p=4.4$\par
					$p=-1.6$\par
				$(BE)=x-1.6$\par
			Si le point E a por même équation que la droite (BE) alors B est sur la tengeante de E.\\
			Tengente de E: \par
					$E=x+p$\par
					$4+p=2.4$\par
					$p=-1.6$\par
					Donc E a la même équation de tengente que (BE) donc B est sur la tengente.
			\item Voir courbe. 
		\end{enumerate}
\end{itemize}




\end{document}

