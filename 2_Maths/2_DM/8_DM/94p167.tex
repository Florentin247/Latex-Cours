%%%%%%
%BASE
%%%
\documentclass[12pt,a4paper]{article}
\usepackage[utf8]{inputenc}
\usepackage[T1]{fontenc}
\usepackage[francais]{babel}

%%%%%
%MATHS
%%%
\usepackage{amsmath}
\usepackage{array}

%%%%%
%COULEURS
%%%
\usepackage{xcolor}
\usepackage{color}
\usepackage{colortbl}

%%%%%
%PUCES
%%%
\usepackage{pifont}

\everymath{\displaystyle}
\usepackage{hyperref}
\setlength{\parindent}{10px}

%%%%%
%Tableau
%%%
\renewcommand{\arraystretch}{1.5}
\setlength{\tabcolsep}{0.75cm}

%%%%%
%Haut de Page
%%%
\usepackage{fancybox}
\usepackage{fancyhdr}
\usepackage[left=1.3cm,right=1.2cm,top=2cm,bottom=1.5cm]{geometry}
\pagestyle{fancy}
\fancyhead[L]{Lespinasse Florentin}
\fancyhead[C]{}
\fancyhead[R]{1STI2D4~~~~~~~~\today}

\begin{document}
\begin{center}
        \shadowbox{\begin{large}
                \textcolor{black}{Exercice de maths}
        \end{large}}
    \end{center}
    \vspace{0.5 cm}
\begin{enumerate}
	\item Le tableau de production d'eau minérale. 
	\begin{table}[ht]
   		\centering
   		\begin{tabular}{|c|c|c|c|}
  			\hline
			 					& Source 1 	& Source 2 	& Total 	\\
		   \hline 
			Eau calcaire 		& 11.2 		& 3 		& 14.2		\\
			\hline
			Eau non calcaire 	& 58.8		& 27		& 85.8		\\
		   \hline
			Total 				& 70 		& 30  		& 100  		\\
		   \hline

   		\end{tabular}
	\end{table}
\item \begin{enumerate}
	\item La probabilité de\par
			$P(A)$ est de $\dfrac{7}{10}$,\par
			$P(C)$ est de $\dfrac{14.2}{100}$,\par
			$P(A\cap C)$ est de $\dfrac{11.2}{100}$,\par
			$P(B \cap C)$ est de $\dfrac{3}{100}$,\par
	\item	La probabilité que $P_C(A)$ est de $\dfrac{11.2}{14.2}$
\end{enumerate}
    \vspace{0.5 cm}
\item
	\begin{enumerate}
	\item Loi de probabilité 
		\begin{table}[ht]
        \centering
        \begin{tabular}{|c|c|c|c|c|c|c|c|}
            \hline
            $X_i$                    & 0  		& 1  	& 2		&	3	&	4	& Total     \\
           \hline
           $P(X=x_i)$ 			     &  0,54    & 0,36     &  0,09 & 	0,01	& 0,0004		&  1  \\
           \hline

        \end{tabular}
    \end{table}

Avec les valeures non approché, le résultat vaut 1.

$P(X=0)$ $ = 1 \times 0.858^4$ $= 0,541937434896 $ \par
$P(X=1)$ $ = 4 \times 0.858^3 \times 0.142^1$ $=  0,3587651080 $\par
$P(X=2)$ $ = 6 \times 0.858^2 \times 0.142^2$ $=  0,089064065376 $ \par
$P(X=3)$ $ = 4 \times 0.858^1 \times 0.142^3$ $=  0,009826804416 $ \par
$P(X=4)$ $ = 1 \times 0.142^4$ $= 0,000406586896 $ \par
	\item La probabilité pour qu'il y ait au moins 1 bouteille d'eau calcaire, est de $0.36+0.09+0.01+0.0004$ $ = 0.464 $ 

\end{enumerate}
\end{enumerate}
\end{document}

