%%%%%%
%BASE
%%%
\documentclass[12pt,a4paper]{article}
\usepackage[utf8]{inputenc}
\usepackage[T1]{fontenc}
\usepackage[francais]{babel}

%%%%%
%MATHS
%%%
\usepackage{amsmath}
\usepackage{array}

%%%%%
%COULEURS
%%%
\usepackage{xcolor}
\usepackage{color}
\usepackage{colortbl}

%%%%%
%PUCES
%%%
\usepackage{pifont}

\everymath{\displaystyle}
\usepackage{hyperref}
\setlength{\parindent}{10px}

%%%%%
%Tableau
%%%
\renewcommand{\arraystretch}{1.5}
\setlength{\tabcolsep}{1cm}

%%%%%
%Arbre de Probabilité
%%%
\usepackage{tikz}
\usetikzlibrary{trees}% Set the overall layout of the tree
\tikzstyle{level 1}=[level distance=3.5cm, sibling distance=5.5cm]
\tikzstyle{level 2}=[level distance=3.5cm, sibling distance=2cm]
\tikzstyle{level 3}=[level distance=3.5cm, sibling distance=1.5cm]
\tikzstyle{level 3}=[level distance=3.5cm, sibling distance=0.5cm]

% Define styles for bags and leafs
\tikzstyle{bag} = [text width=4em, text centered]
\tikzstyle{end} = [circle, minimum width=3pt,fill, inner sep=0pt]


%%%%%
%Haut de Page
%%%
\usepackage{fancybox}
\usepackage{fancyhdr}
\usepackage[left=1.3cm,right=1.2cm,top=2cm,bottom=1.5cm]{geometry}
\pagestyle{fancy}
\fancyhead[L]{Lespinasse Florentin}
\fancyhead[C]{}
\fancyhead[R]{1STI2D4~~~~~~~~\today}

\begin{document}
\begin{center}
        \shadowbox{\begin{large}
                \textcolor{black}{Exercice de maths}
        \end{large}}
    \end{center}
    \vspace{0.5 cm}
\begin{enumerate}
	\item Le tableau de production d'eau minérale. 
	\begin{table}[ht]
   		\centering
   		\begin{tabular}{|c|c|c|c|}
  			\hline
			 					& Source 1 	& Source 2 	& Total 	\\
		   \hline 
			Eau calcaire 		& 11.2 		& 3 		& 14.2		\\
			\hline
			Eau non calcaire 	& 58.8		& 27		& 85.8		\\
		   \hline
			Total 				& 70 		& 30  		& 100  		\\
		   \hline

   		\end{tabular}
	\end{table}
\item \begin{enumerate}
	\item La probabilité de\par
			$P(A)$ est de $\dfrac{7}{10}$,\par
			$P(C)$ est de $\dfrac{14.2}{100}$,\par
			$P(A\cap C)$ est de $\dfrac{11.2}{100}$,\par
			$P(B \cap C)$ est de $\dfrac{3}{100}$,\par
	\item	La probabilité que $P_C(A)$ est de $\dfrac{11.2}{14.2}$
\end{enumerate}
    \vspace{0.5 cm}
\item \begin{enumerate}
	\item Arbre de proba

%:-+-+-+- Engendré par : http://math.et.info.free.fr/TikZ/Arbre/

\begin{center}
% Racine à Gauche, développement vers la droite
\begin{tikzpicture}[xscale=1,yscale=1]
% Styles (MODIFIABLES)
\tikzstyle{fleche}=[->,>=latex,thick]
\tikzstyle{noeud}=[fill=yellow,circle,draw]
\tikzstyle{feuille}=[fill=yellow,circle,draw]
\tikzstyle{etiquette}=[midway,fill=white,draw]
% Dimensions (MODIFIABLES)
\def\DistanceInterNiveaux{3}
\def\DistanceInterFeuilles{2}
% Dimensions calculées (NON MODIFIABLES)
\def\NiveauA{(0)*\DistanceInterNiveaux}
\def\NiveauB{(1)*\DistanceInterNiveaux}
\def\NiveauC{(2)*\DistanceInterNiveaux}
\def\NiveauD{(3)*\DistanceInterNiveaux}
\def\InterFeuilles{(-1)*\DistanceInterFeuilles}
% Noeuds (MODIFIABLES : Styles et Coefficients d'InterFeuilles)
\node[noeud] (R) at ({\NiveauA},{(3.5)*\InterFeuilles}) {$Racine$};
\node[noeud] (Ra) at ({\NiveauB},{(1.5)*\InterFeuilles}) {$C$};
\node[noeud] (Raa) at ({\NiveauC},{(0.5)*\InterFeuilles}) {$C$};
\node[feuille] (Raaa) at ({\NiveauD},{(0)*\InterFeuilles}) {$C$};
\node[feuille] (Raab) at ({\NiveauD},{(1)*\InterFeuilles}) {$\overline{C}$};
\node[noeud] (Rab) at ({\NiveauC},{(2.5)*\InterFeuilles}) {$\overline{C}$};
\node[feuille] (Raba) at ({\NiveauD},{(2)*\InterFeuilles}) {$C$};
\node[feuille] (Rabb) at ({\NiveauD},{(3)*\InterFeuilles}) {$\overline{C}$};
\node[noeud] (Rb) at ({\NiveauB},{(5.5)*\InterFeuilles}) {$\overline{C}$};
\node[noeud] (Rba) at ({\NiveauC},{(4.5)*\InterFeuilles}) {$C$};
\node[feuille] (Rbaa) at ({\NiveauD},{(4)*\InterFeuilles}) {$C$};
\node[feuille] (Rbab) at ({\NiveauD},{(5)*\InterFeuilles}) {$\overline{C}$};
\node[noeud] (Rbb) at ({\NiveauC},{(6.5)*\InterFeuilles}) {$\overline{C}$};
\node[feuille] (Rbba) at ({\NiveauD},{(6)*\InterFeuilles}) {$C$};
\node[feuille] (Rbbb) at ({\NiveauD},{(7)*\InterFeuilles}) {$\overline{C}$};
% Arcs (MODIFIABLES : Styles)
\draw[fleche] (R)--(Ra) node[etiquette] {$...$};
\draw[fleche] (Ra)--(Raa) node[etiquette] {$...$};
\draw[fleche] (Raa)--(Raaa) node[etiquette] {$...$};
\draw[fleche] (Raa)--(Raab) node[etiquette] {$...$};
\draw[fleche] (Ra)--(Rab) node[etiquette] {$...$};
\draw[fleche] (Rab)--(Raba) node[etiquette] {$...$};
\draw[fleche] (Rab)--(Rabb) node[etiquette] {$...$};
\draw[fleche] (R)--(Rb) node[etiquette] {$...$};
\draw[fleche] (Rb)--(Rba) node[etiquette] {$...$};
\draw[fleche] (Rba)--(Rbaa) node[etiquette] {$...$};
\draw[fleche] (Rba)--(Rbab) node[etiquette] {$...$};
\draw[fleche] (Rb)--(Rbb) node[etiquette] {$...$};
\draw[fleche] (Rbb)--(Rbba) node[etiquette] {$...$};
\draw[fleche] (Rbb)--(Rbbb) node[etiquette] {$...$};
\end{tikzpicture}
\end{center}
%:-+-+-+-+- Fin








	\item

	\item

\end{enumerate}
\end{enumerate}
\end{document}

