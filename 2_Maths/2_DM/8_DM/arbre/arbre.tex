\documentclass[10pt,a4paper]{article}
\usepackage[utf8]{inputenc}
\usepackage[french]{babel}
\frenchbsetup{StandardLists=true} % à inclure si on utilise \usepackage[french]{babel}
\usepackage[T1]{fontenc}
\usepackage{amsmath}
\usepackage{amsthm}
\usepackage{amsfonts}
\usepackage{amssymb}
\usepackage{graphicx}
\usepackage{framed}
\usepackage{fancyhdr}
\usepackage[left=1.2cm,right=1.2cm,top=1.8cm,bottom=1.3cm]{geometry}
\usepackage{array} 
\usepackage{fancyhdr} 
\usepackage{fancybox}
\usepackage{pst-tree}
\usepackage[framed]{ntheorem}
\usepackage{tabularx}
\usepackage{pstricks,pstricks-add}
\usepackage{eurosym}
\usepackage{multicol}
\usepackage{tikz,tkz-tab}
\usepackage[tikz]{bclogo}
\usepackage{mathrsfs}
\usepackage{numprint}
\usepackage{pifont}
\rfoot{\small -\thepage-}
\cfoot{}
%papier millimétré
\newcommand{\mili}[4]{\psgrid[subgriddiv=10, gridlabels=0, gridwidth=0.4pt, subgridwidth=0.4pt,gridcolor=brown!80,subgridcolor=brown!40](#1,#2)(#3,#4)}

% Raccourcis diverses:
\newcommand{\nwc}{\newcommand}
\nwc{\dsp}{\displaystyle}
\nwc{\ct}{\centerline}
\nwc{\bgar}{\begin{array}}\nwc{\enar}{\end{array}}
\nwc{\bgit}{\begin{itemize}}\nwc{\enit}{\end{itemize}}
\nwc{\bgen}{\begin{enumerate}}\nwc{\enen}{\end{enumerate}}

\nwc{\la}{\left\{}\nwc{\ra}{\right\}}
\nwc{\lp}{\left(}\nwc{\rp}{\right)}
\nwc{\lb}{\left[}\nwc{\rb}{\right]}

\def\R{{\mathbb R}}
%\def\Q{{\mathbb Q}}
\def\Z{{\mathbb Z}}
%\def\D{{\mathbb D}}
\def\N{{\mathbb N}}
%\def\C{{\mathbb C}}


\nwc{\bgsk}{\bigskip}
\nwc{\vsp}{\vspace{0.1cm}}
\nwc{\vspd}{\vspace{0.2cm}}
\nwc{\vspt}{\vspace{0.3cm}}
\nwc{\vspq}{\vspace{0.4cm}}


\def\epsi{\varepsilon}
\def\vphi{\varphi}
\def\lbd{\lambda}

\def\Cf{\mathcal{C}_f}


\pagestyle{fancy}
\newcommand{\Oij}{$\left( {{\mathrm{O}};\vec i,\vec j} \right)$}
\renewcommand{\thesection}{\Roman{section}}
\renewcommand{\thesubsection}{\arabic{subsection}}
\renewcommand{\thesubsubsection}{\alph{subsubsection}}
\newcommand{\VE}[1]{\overrightarrow{#1}}
\newcolumntype{M}[1]{>{\centering\arraybackslash}m{#1}}
\renewcommand{\arraystretch}{}

\theoremstyle{break}
\theorembodyfont{\upshape}
\newtheorem{Prop}{Propri\'et\'e}
\newtheorem{Def}{D\'efinition}
\newtheorem{Rem}{Remarque}
\newtheorem{corr}{Correction}


\newtheorem{Th}{Théorème}
\theorembodyfont{\small \sffamily }
\newtheorem{Ex}{Exemple}
\newtheorem{Preu}{Preuve}
\theorembodyfont{\small	 \sffamily }
\newtheorem{Meth}{\underline{Methode}}

\pagestyle{fancy}
\fancyhead[L]{1STI2D}
\fancyhead[C]{COURS: PROBABILITES}
\fancyhead[R]{2019/2020}
\rfoot{\small -\thepage-}
\cfoot{}

\begin{document}

	
	\begin{center}
		\shadowbox{\begin{large}
				\textcolor{black}{VARIABLE ALÉATOIRE, LOI de Bernoulli}
			\end{large}}
		\end{center}
	\tableofcontents


	Exemple d’arbre représentant un schéma de Bernoulli pour n = 2\\
	%:-+-+-+- Engendré par : http://math.et.info.free.fr/TikZ/Arbre/
	\begin{center}
		% Racine à Gauche, développement vers la droite
		\begin{tikzpicture}[xscale=1,yscale=1]
		% Styles (MODIFIABLES)
		\tikzstyle{fleche}=[->,>=latex,thick]
		\tikzstyle{noeud}=[fill=yellow,circle,draw]
		\tikzstyle{feuille}=[fill=yellow,circle,draw]
		\tikzstyle{etiquette}=[midway,fill=white,draw]
		% Dimensions (MODIFIABLES)
		\def\DistanceInterNiveaux{3}
		\def\DistanceInterFeuilles{2}
		% Dimensions calculées (NON MODIFIABLES)
		\def\NiveauA{(0)*\DistanceInterNiveaux}
		\def\NiveauB{(1)*\DistanceInterNiveaux}
		\def\NiveauC{(2)*\DistanceInterNiveaux}
		\def\InterFeuilles{(-1)*\DistanceInterFeuilles}
		% Noeuds (MODIFIABLES : Styles et Coefficients d'InterFeuilles)
		\node[noeud] (R) at ({\NiveauA},{(1.5)*\InterFeuilles}) {$\Omega$};
		\node[noeud] (Ra) at ({\NiveauB},{(0.5)*\InterFeuilles}) {$S$};
		\node[feuille] (Raa) at ({\NiveauC},{(0)*\InterFeuilles}) {$S$};
		\node[feuille] (Rab) at ({\NiveauC},{(1)*\InterFeuilles}) {$\overline{S}$};
		\node[noeud] (Rb) at ({\NiveauB},{(2.5)*\InterFeuilles}) {$\overline{S}$};
		\node[feuille] (Rba) at ({\NiveauC},{(2)*\InterFeuilles}) {$S$};
		\node[feuille] (Rbb) at ({\NiveauC},{(3)*\InterFeuilles}) {$\overline{S}$};
		% Arcs (MODIFIABLES : Styles)
		\draw[fleche] (R)--(Ra) node[etiquette] {$\cdots$};
		\draw[fleche] (Ra)--(Raa) node[etiquette] {$\cdots$};
		\draw[fleche] (Ra)--(Rab) node[etiquette] {$\cdots$};
		\draw[fleche] (R)--(Rb) node[etiquette] {$\cdots$};
		\draw[fleche] (Rb)--(Rba) node[etiquette] {$\cdots$};
		\draw[fleche] (Rb)--(Rbb) node[etiquette] {$\cdots$};
		\end{tikzpicture}
	\end{center}
	%:-+-+-+-+- Fin
		On considère un schéma de Bernoulli avec n =3. La représentation de la situation avec un arbre est:\\	
	
	%:-+-+-+- Engendré par : http://math.et.info.free.fr/TikZ/Arbre/
	\begin{center}
		% Racine à Gauche, développement vers la droite
		\begin{tikzpicture}[xscale=1,yscale=1]
		% Styles (MODIFIABLES)
		\tikzstyle{fleche}=[->,>=latex,thick]
		\tikzstyle{noeud}=[fill=yellow,circle,draw]
		\tikzstyle{feuille}=[fill=yellow,circle,draw]
		\tikzstyle{etiquette}=[midway,fill=white,draw]
		% Dimensions (MODIFIABLES)
		\def\DistanceInterNiveaux{3}
		\def\DistanceInterFeuilles{2}
		% Dimensions calculées (NON MODIFIABLES)
		\def\NiveauA{(0)*\DistanceInterNiveaux}
		\def\NiveauB{(1)*\DistanceInterNiveaux}
		\def\NiveauC{(2)*\DistanceInterNiveaux}
		\def\NiveauD{(3)*\DistanceInterNiveaux}
		\def\InterFeuilles{(-1)*\DistanceInterFeuilles}
		% Noeuds (MODIFIABLES : Styles et Coefficients d'InterFeuilles)
		\node[noeud] (R) at ({\NiveauA},{(3.5)*\InterFeuilles}) {$\Omega$};
		\node[noeud] (Ra) at ({\NiveauB},{(1.5)*\InterFeuilles}) {$S$};
		\node[noeud] (Raa) at ({\NiveauC},{(0.5)*\InterFeuilles}) {$S$};
		\node[feuille] (Raaa) at ({\NiveauD},{(0)*\InterFeuilles}) {$S$};
		\node[feuille] (Raab) at ({\NiveauD},{(1)*\InterFeuilles}) {$\overline{S}$};
		\node[noeud] (Rab) at ({\NiveauC},{(2.5)*\InterFeuilles}) {$\overline{S}$};
		\node[feuille] (Raba) at ({\NiveauD},{(2)*\InterFeuilles}) {$S$};
		\node[feuille] (Rabb) at ({\NiveauD},{(3)*\InterFeuilles}) {$\overline{S}$};
		\node[noeud] (Rb) at ({\NiveauB},{(5.5)*\InterFeuilles}) {$\overline{S}$};
		\node[noeud] (Rba) at ({\NiveauC},{(4.5)*\InterFeuilles}) {$S$};
		\node[feuille] (Rbaa) at ({\NiveauD},{(4)*\InterFeuilles}) {$S$};
		\node[feuille] (Rbab) at ({\NiveauD},{(5)*\InterFeuilles}) {$\overline{S}$};
		\node[noeud] (Rbb) at ({\NiveauC},{(6.5)*\InterFeuilles}) {$\overline{S}$};
		\node[feuille] (Rbba) at ({\NiveauD},{(6)*\InterFeuilles}) {$S$};
		\node[feuille] (Rbbb) at ({\NiveauD},{(7)*\InterFeuilles}) {$\overline{S}$};
		% Arcs (MODIFIABLES : Styles)
		\draw[fleche] (R)--(Ra) node[etiquette] {$\cdots$};
		\draw[fleche] (Ra)--(Raa) node[etiquette] {$\cdots$};
		\draw[fleche] (Raa)--(Raaa) node[etiquette] {$\cdots$};
		\draw[fleche] (Raa)--(Raab) node[etiquette] {$\cdots$};
		\draw[fleche] (Ra)--(Rab) node[etiquette] {$\cdots$};
		\draw[fleche] (Rab)--(Raba) node[etiquette] {$\cdots$};
		\draw[fleche] (Rab)--(Rabb) node[etiquette] {$\cdots$};
		\draw[fleche] (R)--(Rb) node[etiquette] {$\cdots$};
		\draw[fleche] (Rb)--(Rba) node[etiquette] {$\cdots$};
		\draw[fleche] (Rba)--(Rbaa) node[etiquette] {$\cdots$};
		\draw[fleche] (Rba)--(Rbab) node[etiquette] {$\cdots$};
		\draw[fleche] (Rb)--(Rbb) node[etiquette] {$\cdots$};
		\draw[fleche] (Rbb)--(Rbba) node[etiquette] {$\cdots$};
		\draw[fleche] (Rbb)--(Rbbb) node[etiquette] {$\cdots$};
		\end{tikzpicture}
	\end{center}
	%:-+-+-+-+- Fin	Il y a 8 chemins possibles au total.
	On considère un schéma de Bernoulli avec n =4. La représentation de la situation avec un arbre est:\\		
%:-+-+-+- Engendré par : http://math.et.info.free.fr/TikZ/Arbre/
\begin{center}
	% Racine à Gauche, développement vers la droite
	\begin{tikzpicture}[xscale=1,yscale=1]
	% Styles (MODIFIABLES)
	\tikzstyle{fleche}=[->,>=latex,thick]
	\tikzstyle{noeud}=[fill=yellow,circle,draw]
	\tikzstyle{feuille}=[fill=yellow,circle,draw]
	\tikzstyle{etiquette}=[midway,fill=white,draw]
	% Dimensions (MODIFIABLES)
\def\DistanceInterNiveaux{2.5}
\def\DistanceInterFeuilles{1.5}
	% Dimensions calculées (NON MODIFIABLES)
	\def\NiveauA{(0)*\DistanceInterNiveaux}
	\def\NiveauB{(1.75)*\DistanceInterNiveaux}
	\def\NiveauC{(3.25)*\DistanceInterNiveaux}
	\def\NiveauD{(4.5)*\DistanceInterNiveaux}
	\def\NiveauE{(5.5)*\DistanceInterNiveaux}
	\def\InterFeuilles{(-1)*\DistanceInterFeuilles}
	% Noeuds (MODIFIABLES : Styles et Coefficients d'InterFeuilles)
	\node[noeud] (R) at ({\NiveauA},{(7.5)*\InterFeuilles}) {$Racine$};
	\node[noeud] (Ra) at ({\NiveauB},{(3.5)*\InterFeuilles}) {$S$};
	\node[noeud] (Raa) at ({\NiveauC},{(1.5)*\InterFeuilles}) {$S$};
	\node[noeud] (Raaa) at ({\NiveauD},{(0.5)*\InterFeuilles}) {$S$};
	\node[feuille] (Raaaa) at ({\NiveauE},{(0)*\InterFeuilles}) {$S$};
	\node[feuille] (Raaab) at ({\NiveauE},{(1)*\InterFeuilles}) {$\bar{S}$};
	\node[noeud] (Raab) at ({\NiveauD},{(2.5)*\InterFeuilles}) {$\bar{S}$};
	\node[feuille] (Raaba) at ({\NiveauE},{(2)*\InterFeuilles}) {$S$};
	\node[feuille] (Raabb) at ({\NiveauE},{(3)*\InterFeuilles}) {$\bar{S}$};
	\node[noeud] (Rab) at ({\NiveauC},{(5.5)*\InterFeuilles}) {$\bar{S}$};
	\node[noeud] (Raba) at ({\NiveauD},{(4.5)*\InterFeuilles}) {$S$};
	\node[feuille] (Rabaa) at ({\NiveauE},{(4)*\InterFeuilles}) {$S$};
	\node[feuille] (Rabab) at ({\NiveauE},{(5)*\InterFeuilles}) {$\bar{S}$};
	\node[noeud] (Rabb) at ({\NiveauD},{(6.5)*\InterFeuilles}) {$\bar{S}$};
	\node[feuille] (Rabba) at ({\NiveauE},{(6)*\InterFeuilles}) {$S$};
	\node[feuille] (Rabbb) at ({\NiveauE},{(7)*\InterFeuilles}) {$\bar{S}$};
	\node[noeud] (Rb) at ({\NiveauB},{(11.5)*\InterFeuilles}) {$\bar{S}$};
	\node[noeud] (Rba) at ({\NiveauC},{(9.5)*\InterFeuilles}) {$S$};
	\node[noeud] (Rbaa) at ({\NiveauD},{(8.5)*\InterFeuilles}) {$S$};
	\node[feuille] (Rbaaa) at ({\NiveauE},{(8)*\InterFeuilles}) {$S$};
	\node[feuille] (Rbaab) at ({\NiveauE},{(9)*\InterFeuilles}) {$\bar{S}$};
	\node[noeud] (Rbab) at ({\NiveauD},{(10.5)*\InterFeuilles}) {$\bar{S}$};
	\node[feuille] (Rbaba) at ({\NiveauE},{(10)*\InterFeuilles}) {$S$};
	\node[feuille] (Rbabb) at ({\NiveauE},{(11)*\InterFeuilles}) {$\bar{S}$};
	\node[noeud] (Rbb) at ({\NiveauC},{(13.5)*\InterFeuilles}) {$\bar{S}$};
	\node[noeud] (Rbba) at ({\NiveauD},{(12.5)*\InterFeuilles}) {$S$};
	\node[feuille] (Rbbaa) at ({\NiveauE},{(12)*\InterFeuilles}) {$S$};
	\node[feuille] (Rbbab) at ({\NiveauE},{(13)*\InterFeuilles}) {$\bar{S}$};
	\node[noeud] (Rbbb) at ({\NiveauD},{(14.5)*\InterFeuilles}) {$\bar{S}$};
	\node[feuille] (Rbbba) at ({\NiveauE},{(14)*\InterFeuilles}) {$S$};
	\node[feuille] (Rbbbb) at ({\NiveauE},{(15)*\InterFeuilles}) {$\bar{S}$};
	% Arcs (MODIFIABLES : Styles)
	\draw[fleche] (R)--(Ra) node[etiquette] {$\cdots\cdots$};
	\draw[fleche] (Ra)--(Raa) node[etiquette] {$\cdots\cdots$};
	\draw[fleche] (Raa)--(Raaa) node[etiquette] {$\cdots\cdots$};
	\draw[fleche] (Raaa)--(Raaaa) node[etiquette] {$\cdots\cdots$};
	\draw[fleche] (Raaa)--(Raaab) node[etiquette] {$\cdots\cdots$};
	\draw[fleche] (Raa)--(Raab) node[etiquette] {$\cdots\cdots$};
	\draw[fleche] (Raab)--(Raaba) node[etiquette] {$\cdots\cdots$};
	\draw[fleche] (Raab)--(Raabb) node[etiquette] {$\cdots\cdots$};
	\draw[fleche] (Ra)--(Rab) node[etiquette] {$\cdots\cdots$};
	\draw[fleche] (Rab)--(Raba) node[etiquette] {$\cdots\cdots$};
	\draw[fleche] (Raba)--(Rabaa) node[etiquette] {$\cdots\cdots$};
	\draw[fleche] (Raba)--(Rabab) node[etiquette] {$\cdots\cdots$};
	\draw[fleche] (Rab)--(Rabb) node[etiquette] {$\cdots\cdots$};
	\draw[fleche] (Rabb)--(Rabba) node[etiquette] {$\cdots\cdots$};
	\draw[fleche] (Rabb)--(Rabbb) node[etiquette] {$\cdots\cdots$};
	\draw[fleche] (R)--(Rb) node[etiquette] {$\cdots\cdots$};
	\draw[fleche] (Rb)--(Rba) node[etiquette] {$\cdots\cdots$};
	\draw[fleche] (Rba)--(Rbaa) node[etiquette] {$\cdots\cdots$};
	\draw[fleche] (Rbaa)--(Rbaaa) node[etiquette] {$\cdots\cdots$};
	\draw[fleche] (Rbaa)--(Rbaab) node[etiquette] {$\cdots\cdots$};
	\draw[fleche] (Rba)--(Rbab) node[etiquette] {$\cdots\cdots$};
	\draw[fleche] (Rbab)--(Rbaba) node[etiquette] {$\cdots\cdots$};
	\draw[fleche] (Rbab)--(Rbabb) node[etiquette] {$\cdots\cdots$};
	\draw[fleche] (Rb)--(Rbb) node[etiquette] {$\cdots\cdots$};
	\draw[fleche] (Rbb)--(Rbba) node[etiquette] {$\cdots\cdots$};
	\draw[fleche] (Rbba)--(Rbbaa) node[etiquette] {$\cdots\cdots$};
	\draw[fleche] (Rbba)--(Rbbab) node[etiquette] {$\cdots\cdots$};
	\draw[fleche] (Rbb)--(Rbbb) node[etiquette] {$\cdots\cdots$};
	\draw[fleche] (Rbbb)--(Rbbba) node[etiquette] {$\cdots\cdots$};
	\draw[fleche] (Rbbb)--(Rbbbb) node[etiquette] {$\cdots\cdots$};
	\end{tikzpicture}
\end{center}
\end{document}