\documentclass[12pt,a4paper]{article}
\usepackage[utf8]{inputenc}
\usepackage[T1]{fontenc}
\usepackage{amsmath}
\usepackage{xcolor}
\usepackage{color}
\usepackage{pifont}
\everymath{\displaystyle}
\usepackage[francais]{babel}
\usepackage{fancyhdr}
\usepackage{fancybox}
\usepackage{hyperref}
%\renewcommand{\thefootnote}{\Alph{footnote}}
\setlength{\parindent}{10px}
\usepackage{fancyhdr}
\usepackage[left=1.3cm,right=1.2cm,top=2cm,bottom=1.5cm]{geometry}
\usepackage{array}
\pagestyle{fancy}
\fancyhead[L]{Florentin Lespinasse}
\fancyhead[C]{Histoire Chapitre 1}
\fancyhead[R]{Ann\'ee 2019/2020}
\colorlet{greeen}{green!50!black}



\begin{document}
\begin{center}
        \shadowbox{\begin{large}
                \textcolor{black}{La metropolisation un processus mondial differencie}
        \end{large}}
    \end{center}
    \vspace{0.5 cm}
Introduction \\ \par
\ding{226}
	En 2018, on à 4 $\overline{M}$ d'urbains ce qui représente 55\% de la population mondiale.\\
\ding{227}
	La concentration urbaine des habitants est un phénomène mondial est un phénomène en augmentation en effet cela touche tout les types de villes, (celles des pays du Nord, du Sud, les petites villes, Métropoles et même les mégapoles).\\
	L'urbanisation représentera 68\% de la population mondiale en 2050.\\
\ding{227}
	Les villes sont egalement très diverses. Elles ont des paysages, des fonctions et des influances différentes.\\

Métropolisation : Concentration des Hommes et des activitées dans les grandes villes influantes (métropoles).\\
Métropoles : Grande ville qui concentre les habitants, les activitées et les pouvoirs de commendement et exerce une influance sur un téritoire qui l'entoure.\\
Mégapoles : plus de 10 $\overline{M}$ d'habitants\\
Urbanisation : Augmentaton de la population urbaine et de l'extention spatiale de la ville.\\\\

\ding{226} Problématique \\
\textcolor{greeen}{En quoi la métropolisation influe-t-elle sur l'organisation des territoires?}\\


\begin{dingautolist}{192}

\item \textit{\textbf{Le poids croissant des métropoles : la métropolisation}}

\item La diversité des paysages des métropoles 
	Les espaces d'une métropole sont structurées par les réseaux de transports et s'organisent entre le centre et la périférie, on parle aussi d'aglomération
Agglomération: Enssemble urbanisé en continue constitué d'une ville centre et de sa banlieue.\\
Centre: espace de décisions qui concentre les différents aspects de la puissance d'une métropole.\\



\item \textit{\textbf{Les Métropoles puissantes à différentes échelles (multiscalaire)}} 

\ding{227}
	Les métropoles sont hierarchisées en fonction de leur influance et de leurs atractivitées.\\
	Mégapoles : 3 dans le monde (Nord Américaine, Japon, européenne)\par
	Constituent un espace urbanisé continu et concentre une forte densité de population. 
	Elle rassemblent aussi de nombreuses fonctions économique et politique au niveau mondial.	

\ding{227}
	Les villes mondiales : Elles se carractérisent par un population importante, par une atractivitée internationnale dans les domaines culturels, politiques...\par
New York, Tokyo, Paris, Londres.

\ding{227}
Villes à l'échelle nationale et régionales qui sont les capitales politiques (Washinton, Brasilia) Les métropoles régionales ont une influance réduite sur le téritoire qui les entoure.



\end{dingautolist}

\end{document}

