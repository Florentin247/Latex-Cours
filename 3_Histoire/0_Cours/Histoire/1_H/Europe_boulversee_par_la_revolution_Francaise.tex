\documentclass[12pt,a4paper]{article}
\usepackage[utf8]{inputenc}
\usepackage[T1]{fontenc}
\usepackage{amsmath}
\usepackage{xcolor}
\usepackage{color}
\usepackage{pifont}
\everymath{\displaystyle}
\usepackage[francais]{babel}
\usepackage{fancyhdr}
\usepackage{fancybox}
\usepackage{hyperref}
%\renewcommand{\thefootnote}{\Alph{footnote}}
\setlength{\parindent}{10px}
\usepackage{fancyhdr}
\usepackage[left=1.3cm,right=1.2cm,top=2cm,bottom=1.5cm]{geometry}
\usepackage{array}
\pagestyle{fancy}
\fancyhead[L]{Florentin Lespinasse}
\fancyhead[C]{Histoire Chapitre 1}
\fancyhead[R]{Ann\'ee 2019/2020}
\colorlet{greeen}{green!50!black}

\renewcommand{\arraystretch}{1.1} 
\colorlet{gris}{gray}{0.80} 
\newcounter{lignetab} 
\setcounter{lignetab}{11} 
\newcommand{\lignetab}{% 
\emph{\stepcounter{lignetab} \arabic{lignetab}}} 




\begin{document}
\begin{center}
        \shadowbox{\begin{large}
                \textcolor{black}{L'europe boulversée par la révolution Fran\c caise $(1789-1815)$}
        \end{large}}
    \end{center}
    \vspace{0.5 cm}
Introduction \\ \par
	Au milieu de la France, les transformations sont surtout politiques et sociales et avec l'expériance de nouveaux régimes politiques et socieaux.
	L'Europe, les idéesde France entra\^inent des gueres, mais ces idées se diffusent et entra\^inet changements dans tout l'europe. \\

\ding{226} Problématique \\
\textcolor{greeen}{En quoi la Révolution francaise boulverse-t-elle l'Europe?}


\begin{dingautolist}{192}

\item 1789 : Un été Révolutionnaire \\ 
\'Egalitée devant la loi: Principe selon lequel chacun est traité de la même facon par la loi sans privilèges \\
Souveraineté Nationale: La nation est l'origine des pouvoirs politiques (votes/suffrage).

\item 1792-1794 : L'Europe menace la Révolution Fran\c caise \\
République: Régime politique o\'u le pouvoir est détenu par un groupe de personne élue par le peuple.

\item 1799-1815 : L'Eaurope et Napoléon \\
Empire: Régime politique dirigé par un empereur et territoire sur lequel il exerce son pouvoir.

\end{dingautolist}

\end{document}

