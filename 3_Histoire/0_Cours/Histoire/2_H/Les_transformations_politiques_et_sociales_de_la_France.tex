\documentclass[12pt,a4paper]{article}
\usepackage[utf8]{inputenc}
\usepackage[T1]{fontenc}
\usepackage{amsmath}
\usepackage{xcolor}
\usepackage{color}
\usepackage{pifont}
\everymath{\displaystyle}
\usepackage[francais]{babel}
\usepackage{fancyhdr}
\usepackage{fancybox}
\usepackage{hyperref}
%\renewcommand{\thefootnote}{\Alph{footnote}}
\setlength{\parindent}{10px}
\usepackage{fancyhdr}
\usepackage[left=1.3cm,right=1.2cm,top=2cm,bottom=1.5cm]{geometry}
\usepackage{array}
\pagestyle{fancy}
\fancyhead[L]{Florentin Lespinasse}
\fancyhead[C]{Histoire Chapitre 2}
\fancyhead[R]{Ann\'ee 2019/2020}

\begin{document}
\begin{center}
        \shadowbox{\begin{large}
                \textcolor{black}{Les transformations politiques et sociales de la France $(1848-1870)$ }
        \end{large}}
    \end{center}
    \vspace{0.5 cm}
Introduction

	Entre 1848 et 1870, plusieurs régimes politiques se succèdent. 
	La France passe d'une république avec des avancées démocratiques importantes à un empire $\to$ régime politique autoritaire\footnote{Régime politique dans lequel le chef d'état concentre tout les pouvoirs, les libertées et les droits sont supprimés ou menacés}. 
	Cette période est aussi un moment de boulversement politique, économique, et social.\\ \\

\ding{226} Problématique \\
	Comment la démocratie\footnote{Régime politique fondé sur la souverainetée du peuple, la séparation des pouvoirs, le pluralisme politique, le respect des loies et des libertées} progresse-t-elle en France? \\ \\



\begin{dingautolist}{192}

\item La Seconde République (1848-1852) \\ 
	Suffrage Universel Masculin : Systeme électoral dans lequel tout les citoyens majeurs de sexe masculin peuvent voter.



\item Le Second Empire (1852-1870) \\
	Droit de grève : Droit pour les salariés de cesser collectivement le travail afin d'obtenir des ameliorations de leurs condition de vie et de travail (1864).

\item L'Industrialisation de la France \\
	La France se modernise sous le 2$^{nd}$	Empire avec les développement des moyens de transports (chemins de fer, canneaux...) et la mécanisation des machines dans les usines.
	L'Industrialisation\footnote{Passage d'une économie reposant sur l'artisanat et l'agriculture à une économie qui fonde sa richesse sur la production industrielle réalisée dans des usines grâce à des machines}, modifie les paysages et la société (paysans $\to$ ouvriers dans les usines à cause de l'urbanisation). 
	Les villes se transforment avec l'apparition de banlieues industrielles et les ouvriers deviennent une classe sociale importante.
	La France connait une forte urbaniation\footnote{Mouvement de concentration des populations dans les villes}.\par
	Les conditions de vie des ouvriers sont difficiles (les logements sont insalubre, et le travail en usine est épuisant (forte chaleur, bruit...).





\end{dingautolist}

\end{document}

