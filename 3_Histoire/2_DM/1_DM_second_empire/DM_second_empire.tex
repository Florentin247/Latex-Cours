\documentclass[12pt,a4paper]{article}
\usepackage[utf8]{inputenc}
\usepackage[T1]{fontenc}
\usepackage{amsmath}
\usepackage{xcolor}
\usepackage{color}
\usepackage{pifont}
\everymath{\displaystyle}
\usepackage[francais]{babel}
\usepackage{fancyhdr}
\usepackage{fancybox}
\usepackage{hyperref}

\usepackage{fancyhdr}
\usepackage[left=1.3cm,right=1.2cm,top=2cm,bottom=1.5cm]{geometry}
\usepackage{array}
\pagestyle{fancy}
\fancyhead[L]{Lespinasse Florentin~~~~~~1STI2D4}
\fancyhead[C]{Devoir Maison}
\fancyhead[R]{09 d\'ecembre 2019}

\begin{document}
\begin{center}
        \shadowbox{\begin{large}
                \textcolor{black}{Le Second Empire}
        \end{large}}
    \end{center}
    \vspace{0.5 cm}
	\begin{paragraphe}\par
	Louis Napol\'eon Bonaparte prends tous les pouvoirs d\`es son coup d'\'etat le 2 d\'ecembre 1851. 
	Il attends le 2 d\'ecembre 1852 pour se faire sacrer Empereur, la date du sacrement de Napol\'eon \textsc{\romannumeral 1}$^{er}$, qui est son oncle.  
	Napol\'eon \textsc{\romannumeral 3} a le pouvoir executif, nome les ministres, le conseil d'\'tat, le s\'enat, le pr\'efet, le sous-pr\'efet et aussi les maires des communes. 
	Napol\'eon \textsc{\romannumeral 3} garde le pouvoir l\'egislatif et nome les candidats qui doivent se pr\'esenter.
	\end{paragraphe}\\
	
	\begin{paragraphe}\par
	L'empereur prends aussi le contr\^ole des libert\'es. 
	La presse, expression, r\'eunion... 
	Les oeuvres litt\'eraires sont soumises a la censure si elles attaquent le pouvoir, l'empreur, sa fa\c con de gouverner... 
	Victor Hugo \`a beaucoup contest\'e le pouvoir du point de vue religieux comme social. 
	Deplus, il revendique le r\'egime autoritaire. 
	Donc tout ses \'crits sont censur\'es.
	Mais il arrive quand m\^eme \`a les publiers.
	Napol\'eon \textsc{\romannumeral 3}, rassemble la foule et pl\'ebiscite. Cela les emp\^echent donc de se prononcer sur leur point de vue.
	Il interdt certaines libert\'es comme celle d'association.
    \end{paragraphe}\\

	\begin{paragraphe}\par
	Il existe des contestations politiques.
	Victor Hugo en fait parti, il conteste le r\'egime autoritaire comme L\'eon Gambetta. 
	Avec son texte sur les libert\'es, il d\'efend le << Programme de Belleville >>. 
	Lors des \'elections l\'egislatives de 1869.
	Le suffrage universel masculin est gard\'e de la 2$^{nd}$ R\'epublique, mais ne sont autoris\'es \`a voter seulement les hommes de 21 ans ou plus et doivent en plus \^etre fran\c cais. 
	Deplus, les scrutins doivent \^etre secret.
	\end{paragraphe}\\

\end{document}

