\documentclass[11pt]{lettre}

\usepackage[utf8]{inputenc}
\usepackage[T1]{fontenc}
\usepackage{lmodern}
\usepackage{eurosym}
\usepackage[francais]{babel}
\usepackage{numprint}
\makeatletter
\newcommand*{\NoRule}{\renewcommand*{\rule@length}{0.5}}
\makeatother


\begin{document}
\begin{tt}
\begin{letter}{Famille Cavignac \\1440 route de villette\\38380 Miribel Les \'Echelles }
\NoRule
\name{Lespinasse Florentin}
\address{Lespinasse Florentin\\2 Rue André Marie Ampère\\71530 Champforgeuil}
\lieu{Creusot,}
\date{ le 19 ao\^ut 1845}
\notelephone%{07 66 24 85 76}
\noemail%{floles38@gmail.com}
\nofax

\def\concname{Objet :~} % On définit ici la commande 'objet'
\conc{Ma nouvelle vie}
\opening{Chère M\`ere, cher P\`ere,}
\begin{paragraphe}\par
Eug\`ene Schneider est le fondateur de cette entreprise, il nous acceuille avec le plus grands respect qu'il peut donner \`a ses ouvriers, il a la «rosette» sur son veston car il est d\'ecor\'e de la L\'egion d'honneur. 
Mes coll\`egues et moi sommes log\'es. 
Le patron nous paye en fonction de ce que nous faisont dans la journ\'ee. 
Notre supp\'erieur est un homme politique, qui a tennu plusieurs postes durant sa vie, Maire, chef du corps l\'egislatif, d\'eput\'e mais aussi ministre. 
Il s'implante en 1836 au Creusot pour fonder avec son fr\`ere l'entreprise Schneider. 
\end{paragraphe}

\begin{paragraphe}\par
Je vis dans une ville avec des usines \`a ne jamais en finir, des chemin\'ees tr\`es hautes, qui fonctionnent sans rel\^ache. 
La suie qui est rejett\'ee par ces chemin\'ees, retombe sur la terre, de nos potager, sur les toits des habitations. 
Des \'ecoles se cr\'ee pour que les familles ne sois pas \'eloign\'es de ceux qui travaillent dans l'entreprise. 
Dans l'entreprise, il y a une infirmerie pour soigner les bless\'es. 
L'entreprise Schneider fonctionne gr\^ace \`a son implentation, avec le charbon \`a fleur \`a proximit\'e. 
Le transport de cette mati\`ere principale est \'effectu\'e par des trains mais aussi par les voies fluviales. 
Deplus, Eug\`ene fait \`evoluer son entreprise avec la cr\'eation d'acier \`a canon. 
\end{paragraphe}

\begin{paragraphe}\par
Maintenant, je suis ouvrier d'une tr\`es grande entreprise, nous sommes une cinquantaine a travailler en m\^eme temps. 
L'atelier est tr\`es grand mais nous sommes au chaud toute la journ\'e. 
Quand je rentre le soir dans mon appartement, j'arrose mon petit jardin, manges mes r\'coltes, je n'ai pas beaucoup de temps pour faire autre choses. 
Mon nouveau travail est tr\`es \'epuisant.
Il y a certains de mes coll\`egues qui sont tomb\'es malade et que l'on a jamais revu. 
Quand une personne pars, il y en a d'autres qui arrivent. 
Des cit\'es ouvri\`eres sont cr\'e\'ees, la famille Schneider prend soin de nous, il nous paternalise. 
\end{paragraphe}

\closing{Je vous embrasse fort et je vous promet de passer vous voir bient\^ot. Je vous aimes et penses beaucoup \`a vous.}
\end{tt}
\noname
\end{letter}
\end{document} 


