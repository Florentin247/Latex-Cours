%%%%%%
%BASE
%%%
\documentclass[12pt,a4paper]{article}
\usepackage[utf8]{inputenc}
\usepackage[T1]{fontenc}
\usepackage[francais]{babel}

%%%%%
%MATHS
%%%
\usepackage{amsmath}
\usepackage{array}

%%%%%
%COULEURS
%%%
\usepackage{xcolor}
\usepackage{color}

%%%%%
%PUCES
%%%
\usepackage{pifont}

\everymath{\displaystyle}
\usepackage{hyperref}
\setlength{\parindent}{10px}



%%%%%
%Haut de Page
%%%
\usepackage{fancybox}
\usepackage{fancyhdr}
\usepackage[left=1.3cm,right=1.2cm,top=2cm,bottom=1.5cm]{geometry}

\begin{document}
\begin{center}
        \shadowbox{\begin{large}
                \textcolor{black}{Chaine de moto}
        \end{large}}
    \end{center}
    \vspace{0.5 cm}
\begin{enumerate}
\item 
\item 
\item 
\item 
\item 
\item 
\item 
\item $\frac{F}{(S2)}$ $\leq$ $\frac{Re}{K}$ $\Rightarrow$ $\frac{6500}{h\times1.2}$ $\leq$ $\frac{620}{2}$ $\Rightarrow$ $6500 $$\leq$$ 310\times1.2h$ $\Rightarrow$ $6500 $$\leq$$ 372h$ $\Rightarrow$ $17.47 $$\leq$$ h$
\item La formule est $\frac{h}{2} = \frac{17.5}{2} = 8.75$\par
	Donc $h=2h'$ et $S1=h$
\item
\item On voit que la pièce est déformée surtout au niveau de l'endroit ou la force est appliquée,\par
La limite d'elasticitée est à $$
on modifie le diamètre interieur, $(7)$ et l'épaisseur $(4)$ la limite d'élasticitée est à ${6.204e+008}$



\end{enumerate}
\end{document}

