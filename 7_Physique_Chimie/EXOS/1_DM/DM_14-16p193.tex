%%%%%%
%BASE
%%%
\documentclass[12pt,a4paper]{article}
\usepackage[utf8]{inputenc}
\usepackage[T1]{fontenc}
\usepackage[francais]{babel}

%%%%%
%MATHS
%%%
\usepackage{amsmath}
\usepackage{array}
\newtheorem{exo}{Exercice}

%%%%%
%COULEURS
%%%
\usepackage{xcolor}
\usepackage{color}

%%%%%
%PUCES
%%%
\usepackage{pifont}

\everymath{\displaystyle}
\usepackage{hyperref}
\setlength{\parindent}{10px}



%%%%%
%Haut de Page
%%%
\usepackage{fancybox}
\usepackage{fancyhdr}
\usepackage[left=1.3cm,right=1.2cm,top=2cm,bottom=1.5cm]{geometry}
\pagestyle{fancy}
\fancyhead[L]{Lespinasse Florentin}
\fancyhead[C]{}
\fancyhead[R]{1STI2D4~~~~~~~~\today}

\begin{document}
\begin{center}
        \shadowbox{\begin{large}
                \textcolor{black}{Exercices }
        \end{large}}
    \end{center}
    \vspace{0.5 cm}
\textbf{Exercice (14p193)}
\begin{enumerate}\par
			\item $Fe = Fe^3^+ \+ 3e ^-$
			\item $I_2+2e^- = 2I^-$\par
				  $SO_4^2^- + 4H^+ = SO_2+e^- + 2H_2O$\par
				  $I_2+SO_4^2^-+4H^++2e^- = 2I^-+SO_2+2H_20$
			\item $ZN^2^+ + 2e^- = ZN$\par
				  $NO_3^-+4H^++3e^- = No +2H_2O$\par
				  $ZN^2^+ +NO_3 +4H^++5e^- = NO+2H_2O+ZN$
\end{enumerate} 

\textbf{Exercice (16p193)}
\begin{enumerate}\par 
			\item $2ClO^-+4H^++2e^+ = Cl_2+2H_2O$\par
				  $Cl_2+2e^-= 2Cl^-$
			\item $2ClO^-+4H^++Cl_2+5e^- = Cl_2+2H_2O+2Cl^-$
			\item Le gaz toxique qui s'echape c'est du dichlore
\end{enumerate}

\end{document}

