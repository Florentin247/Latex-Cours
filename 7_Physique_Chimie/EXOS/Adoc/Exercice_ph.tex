%%%%%%
%BASE
%%%
\documentclass[12pt,a4paper]{article}
\usepackage[utf8]{inputenc}
\usepackage[T1]{fontenc}
\usepackage[francais]{babel}

%%%%%
%MATHS
%%%
\usepackage{amsmath}
\usepackage{array}
\newtheorem{exo}{Exercice}

%%%%%
%COULEURS
%%%
\usepackage{xcolor}
\usepackage{color}

%%%%%
%PUCES
%%%
\usepackage{pifont}

\everymath{\displaystyle}
\usepackage{hyperref}
\setlength{\parindent}{10px}



%%%%%
%Haut de Page
%%%
\usepackage{fancybox}
\usepackage{fancyhdr}
\usepackage[left=1.3cm,right=1.2cm,top=2cm,bottom=1.5cm]{geometry}
\pagestyle{fancy}
\fancyhead[L]{Lespinasse Florentin}
\fancyhead[C]{}
\fancyhead[R]{1STI2D4~~~~~~~~\today}

\begin{document}
\begin{center}
        \shadowbox{\begin{large}
                \textcolor{black}{Exercices }
        \end{large}}
    \end{center}
    \vspace{0.5 cm}
\begin{enumerate}
\item La perturbation générée par la ola est un mouvement au départ de quelques spéctateurs voisins qui se mettent debout et lèvent les bras au ciel.
\item Elle se propage dans l'air.
\item Sans transport de matière.
\item La direction de la propagation est longitudinale.
\item Les spectaturs ont un mouvement perpendiculaire à la ola.
\item Une onde transversale est un type d'onde pour lequel la déformation du milieu est perpendiculaire à la direction de propagation de l'onde.
\item UNE CORDE, l'air, non, transversales, horisontale,onde mecanique.
\item LE RESSORT, l'air, non, longitudinale, verticale, onde mecanique.
\item LE HAUT PARLEUR, l'air, oui,longitudinale,verticale, onde mecanique progressive.
\item Une onde mécanique progressive est le phénomène de propagation d'une perturbation locale dans un milieu matériel.

\end{enumerate}
\end{document}

