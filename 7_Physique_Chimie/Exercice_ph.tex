%%%%%%
%BASE
%%%
\documentclass[12pt,a4paper]{article}
\usepackage[utf8]{inputenc}
\usepackage[T1]{fontenc}
\usepackage[francais]{babel}

%%%%%
%MATHS
%%%
\usepackage{amsmath}
\usepackage{array}
\newtheorem{exo}{Exercice}

%%%%%
%COULEURS
%%%
\usepackage{xcolor}
\usepackage{color}

%%%%%
%PUCES
%%%
\usepackage{pifont}

\everymath{\displaystyle}
\usepackage{hyperref}
\setlength{\parindent}{10px}



%%%%%
%Haut de Page
%%%
\usepackage{fancybox}
\usepackage{fancyhdr}
\usepackage[left=1.3cm,right=1.2cm,top=2cm,bottom=1.5cm]{geometry}
\pagestyle{fancy}
\fancyhead[L]{Lespinasse Florentin}
\fancyhead[C]{}
\fancyhead[R]{1STI2D4~~~~~~~~\today}

\begin{document}
\begin{center}
        \shadowbox{\begin{large}
                \textcolor{black}{Exercices }
        \end{large}}
    \end{center}
    \vspace{0.5 cm}
\textbf{Exercice (4p193)}
\begin{enumerate}\par
			\item $2H^++2e^-$$\Rightarrow$$H_2$
			\item L'oxydé est le $H_2(g)$ et le réduit est $H^+(g)$
\end{enumerate}

\textbf{Exercice (5p193)}
\begin{enumerate}\par
			\item $Zn^2^+(aq)+2e^-$$\Rightarrow$$Zn(s)$
			\item $I_2(aq)+2e^-$$\Rightarrow$$2I^-(aq)$
			\item $Al^3^+(aq)+3e^-$$\Rightarrow$$Al(s)$
			\item $Cl_2(aq)+2e^-$$\Rightarrow$$2Cl^-(aq)$
			\item $O_2(g)+4H + 4e^-$$\Rightarrow$$2H_2O(l)$
			\item $NO_3^-(aq)+e^+$$\Rightarrow$$NO(g)$
\end{enumerate}

\textbf{Exercice (7p193)}
\begin{enumerate}\par
			\item $O_2 + 2H^+ + 2e^-$$\Rightarrow$$H_2O_2$\par
				  $MnO_4^- + 8H^+ + 5e^-$$\Rightarrow$$Mn^2^+ + 4H_2O$
			\item Oxydant $O_2$ et $MnO_4^- $\par
				  Réducteurs $H_2O_2$ et $Mn^2^+ $  
			\item pas compris la question
\end{enumerate}

\textbf{Exercice (9p193)}
\begin{enumerate}\par
			\item
			\item
\end{enumerate}

\end{document}

