\documentclass[12pt,a4paper]{article}
\usepackage[utf8]{inputenc}
\usepackage[T1]{fontenc}
\usepackage{xcolor}
\usepackage{color}
\usepackage{pifont}
\everymath{\displaystyle}
\usepackage[francais]{babel}
\usepackage{hyperref}
\setlength{\parindent}{10px}

\usepackage{fancyhdr}
\usepackage{fancybox}
\usepackage[left=1.3cm,right=1.2cm,top=2cm,bottom=1.5cm]{geometry}
\usepackage{array}
%\pagestyle{fancy}

\begin{document}
\begin{center}
        \shadowbox{\begin{Large}
                \textcolor{black}{\textbf{Règles de vie dans la Maison Des Lycéens}}
        \end{Large}}
    \end{center}
    \vspace{0.5 cm}
\begin{Jeux}{\textcolor{blue}{\Large{\underline{\textbf{JEUX :}}}}} \\ \par
\ding{229} Noter le nom de l'adhérent concerné sur le cahier, lui rappeler de faire attention car il est responssable du matériel qu'il emprunte. Si une dégradation est remarquée, il sera sanctionné.\par
\ding{229} Si une perte / dégradation est remarquée, merci d'en informer le bureau immédiatement.\par
⇧
\ding{229} Le temps maximum d'emprunt pour un jeu : \par
		~~~~~~~	-Durant la journée pendant les heures de cours.\par
		~~~~~~~	-Durant la nuit pour les internes. \par
~~~~\ding{161} Sanction --> Interdiction d'emprunter des jeux pour la semaine.\\
\end{Jeux}\\ 
    %\vspace{0.5 cm}
\begin{PC}{\textcolor{blue}{\Large{\underline{\textbf{PC :}}}}} \\ \par
\ding{229} Faire attention au matériel informatique, faire attention aux rageux. (Lancé de manettes, de souris, énervement sur le clavier...)\par
~~~~\ding{161} Sanction --> l'emmener s'expliquer avec la CPE (M$^m^e$ Montessuit) sinon l'emener voir les surveillants.\par
\ding{229} Si l'adérent fait autre chose que des jeux sur le PC, noter son nom sur le cahier puis lui interdire les PC toute la journée. Si repris à plusieurs reprises prévenir le Bureau puis l'emmener s'expliquer avec la CPE (M$^m^e$ Montessuit).\par
\ding{229} Interdit durant la récré.\par
\ding{229} Pas plus de 4 personnes sur les PC pour les joueurs manettes.\par
\ding{229} 2 chaises par PC sinon ils restent debout derrière.\par
\ding{229} Pas d'autres périphériques que ceux de la MDL n'ont le droit d'\^etre branchés sur les PC. \par
\ding{229} Si le joueur crie trop derrière le PC :\par
~~~~\ding{161} Sanction --> interdiction de retourner sur les PC de la journée. \par
\ding{229} Les "NEXT" peuvent \^etre prises mais pas les "SUR NEXT". \par
\ding{229} 1 Game dure 20 minutes maximum. \par
\ding{229} Il est interdit de déplacer les PC. \par
\ding{229} Les Casques audio sont sur autorisation gérant. \par
\end{PC} 
    \vspace{0.5 cm}
\begin{Restauration}{\textcolor{blue}{\underline{\Large{\textbf{Restauration :}}}}} \\ \par
\ding{229}Boissons et nourriture du lycée (self, cafet) sont acceptées, mais pas la nourriture extérieure ainsi que toute nourriture produisant une odeur incommodante (Burger, Tacos, Kebab...)\par
\ding{229}Pas de nourriture et boissons  près des PC.\par
\ding{229}Tous les papiers doivent \^etre mis à la poubelle pour faciliter le ménage.\par
\end{Restauration} \\ 
    \vspace{0.5 cm}
\begin{Autre}
\ding{229} Le niveau sonore doit rester correct.\\
\ding{229} Les tables ne doivent pas \^etre bougées.\\
\end{Autre} \\ \par
\end{document}

